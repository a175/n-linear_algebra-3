% !TeX root =./x2.tex
% !TeX program = pdfpLaTeX
\chapter*{注意など}
\section{注意}
このメモは随時追記, 修正する.
必ずしも末尾に追記するとは限らない.

講義で解説する順番と必ずしも一致しない.
また, 講義で解説する内容をすべてここにまとめるわけでもなく,
ここでまとめた内容をすべて講義で解説するわけでもない.

忙しくなった場合には,
このメモは,
更新しない予定である.

\section{シラバス}
シラバスに挙げた項目は以下の通り:
\begin{enumerate}
\item
  ベクトル空間,
\item 線形写像,
\item 行列式,
\item 固有値$\cdot$固有ベクトル,
\item 行列の対角化
\end{enumerate}

\section{参考書}
%% このノートの末尾に挙げた参考文献のうち,
%% \cite{978-4-7806-0772-7}
%% はシラバスに教科書として挙げたものである.
%% \cite{978-4-7806-0164-0,978-4-535-78682-0}
%% はシラバスに参考書として挙げたものである.

%% \cite{978-4-535-78682-0}は大学の数学に関する講義全般について解説している本である.
%% 講義を受ける上での注意や,
%% 講義の前提となる集合や論理の知識についても解説がしてある.
%% 本講義とは直接関係はないが,
%% 図書館で借りるなどして一度最初の数章を読んでみることを勧める.


\section{講義で使われる見出し}
講義に出てくる言明には大きく分けて二つあります:
\begin{itemize}
 \item 証明が必要ではないもの
 \item 証明が必要なもの
\end{itemize}
また講義の板書では,
言明に見出しをつけることがしばしばあります.
以下では使われる見出しについて説明します.
\subsection{証明が必要ではない言明}
言葉の持つ意味を決めることが,
`定義'です.
つまり, ルールを決めるということです.
ルールを設定するということですので,
`定義'自身に証明は必要ではありません.

数学書では定義される用語をイタリック/太文字/下線で
書く習慣があります.
重要だからという理由で強調されているわけではありません.

Def/Definition/定義 などの見出しが使われます.

\subsection{証明が必要な言明}
証明をして真である確認する必要がある言明は,
`命題'と呼ばれます.
`主張' や `事実' と呼ぶこともあると思います.

講義では, 
気分や役割により次の様に分類することが多いと思います.
これらのうち, どの見出しを使うかというのは, 
かなり主観的です.
\begin{description}
\item[定理]
重要なもの, まとめ的なもの など 比較的重いものに使われます.

Thm/Theorem/定理  などの見出しを使います.

\item[補題]
他の命題を示すために使うものに使われます.

Lem/Lemma/補題  などの見出しを使います.

\item[系]
他からすぐ示すことができるものに使われます.

Cor/Corollary/系  などの見出しを使います.
\item[その他の命題]

Prop/Porposition/命題  などの見出しを使います.
\end{description}

\subsection{補足}
また見出しとしては次のようなものも使こともあります:
\begin{itemize}
 \item 
実例, 具体例 (`喩え'ではない) を述べる際に,
例/Example/E.g.\   などの見出しを使います.
\item
補足的なことを述べる際に,
注/Rem/Remark   などの見出しを使います.
\end{itemize}

\tableofcontents

\chapter{線形代数以前の話}


\section{集合や論理について}

集合や論理についての基本的なことについては,
既知とします.
もしそれらについて不安な人は,
高校の教科書の該当箇所や,
\cite{978-4-535-78682-0}
を参考に復習をすること.

基本的な概念:
\begin{enumerate}
  \item 集合 $\in$, $\subset$
  \item 写像 $\circ$
  \item 恒等写像$\id_V$.
  \item 演算, 作用.
  \item 行列の和, スカラー倍, 積, 転置.
\end{enumerate}


以下の記号を用いる:
\begin{itemize}
\item 複素数全体のなす集合を$\CC$とおく.
\item 実数全体のなす集合を$\RR$とおく.
\item 有理数全体のなす集合を$\QQ$とおく.
\item 整数全体のなす集合を$\ZZ$とおく.
\item 非負整数全体のなす集合を$\NN$とおく.
\end{itemize}
つまり,
`$x\in\CC$'は
`$x$は複素数である'
ということを意味する.

演算と作用について.

\section{体}
$K$を集合とする.
$K$に四則演算(加減乗除)が定まっているとき,
$K$は
\defit{体}\footnote{タイと読む.}
\Defit{field}
であるという.
ただし,
``四則演算が定まっている''
とは, 次を満たしていることとする:
\begin{enumerate}
\item
  $s,t\in K$に対し$s+t$という$K$の元が定まる.
  (この演算を加法と呼ぶ.)
  さらに, $0$という(特別な)元がある.

  $a,b,c\in K$なら, 次が成り立つ:
  \begin{enumerate}
  \item
    $(a+b)+c=a+(b+c)$.
  \item
    $a+b=b+a$.
  \item
    $0+a=a$.
  \item
    $x$に関する方程式
    \begin{align*}
      a+x=0
    \end{align*}
    が($K$の元の)解を持つ.
  \end{enumerate}
\item
  このとき, $a+x=0$の解は,
  $a$が定まればただ一つに定まるので,
  この解を$-a$と書く.

  $b+(-a)$を$b-a$と略記する。
  (この演算を加法と呼ぶ.)

\item
  $s,t\in K$に対し$st$という$K$の元が定まる.
  (この演算を乗法と呼ぶ.)
  さらに, $1$という(特別な)元がある.

  $a,b,c\in K$なら, 次が成り立つ:
  \begin{enumerate}
  \item
    $(ab)c=a(bc)$.
  \item
    $ab=ba$.
  \item
    $1a=a$.
  \item
    $a\neq 0$ならば,
    $x$に関する方程式
    \begin{align*}
      ax=1
    \end{align*}
    が($K$の元の)解を持つ.
  \end{enumerate}

\item
  $a\neq 0$とする.
  このとき, $ax=1$の解は,
  $a$が定まればただ一つに定まるので,
  この解を$a^{-1}$とか$\frac{1}{a}$と書く.

  $b(\frac{1}{a})$を$\frac{b}{a}$と略記する。
  (この演算を除法と呼ぶ.)

\item
  $a,b,c\in K$なら, 以下が成り立つ:
  \begin{enumerate}
  \item $a(b+c)=ab+ac$.
  \item $0\neq 1$.
  \end{enumerate}
\end{enumerate}

\begin{remark}
  つまり体とは,
  結合則, 分配則, 可換則などが成り立つ
  四則演算が備わった数の集合のことである.
\end{remark}


\begin{example}
  $\CC$は体である.
\end{example}
\begin{example}
  $\RR$は体である.
\end{example}
\begin{example}
  $\QQ$は体である.
\end{example}
\begin{remark}
  証明は省略する.
  以後これらは事実として使う.
\end{remark}
\begin{remark}
  掛け算は$ab$または$a\cdot b$のように書き表す.
  ($\times$は本原稿では別の意味で使うので,
  掛け算の意味では用いない)
\end{remark}


\begin{example}
  $2\in\ZZ$ではあるが$\frac{1}{2}\not\in \ZZ$である.
  したがって$2x=1$は$\ZZ$の元の解を持たない.
  $2\neq 0$であるが$2x=1$は$\ZZ$の元の解を持たないので,
  $\ZZ$は体ではない.
\end{example}

%% 本講義では,
%% 体$K$の元のことを数と呼ぶ.
%% つまり,
%% 数に対しては四則演算ができる.
%% 特に断らない限り,
%% 体としては何を考えても構わない.


\sectionX{章末問題}

\begin{quiz}
  %\solvelater{quiz:0:1}
  $\KK$を体とし, $V$を$\KK$を成分とする$(2,1)$行列の集合とする.
  つまり,
  \begin{align*}
    V=\Set{\begin{pmatrix}a_1\\a_2\end{pmatrix}|a_1,a_2\in \KK}
  \end{align*}
  とする. 通常の和とスカラー倍を考える.  つまり,
  \begin{align*}
    a=\begin{pmatrix}a_1\\a_2\end{pmatrix}, & b=\begin{pmatrix}b_1\\b_2\end{pmatrix} \in V
  \end{align*}
  $\alpha\in\KK$に対し,
  \begin{align*}
    a+b&=\begin{pmatrix}a_1+b_1\\a_2+b_2\end{pmatrix}, & \alpha a=\begin{pmatrix}\alpha a_1\\\alpha a_2\end{pmatrix} \in V
  \end{align*}
  とする. また,
  \begin{align*}
  \zzero=
  \begin{pmatrix}
    0\\0
  \end{pmatrix}
  \end{align*}
  とする.
  このとき, $\alpha,\beta\in\KK$, $a,b,c\in V$に対し, 以下を示せ:
  \begin{enumerate}
    \item $a+b=b+a$.
    \item $(a+b)+c=a+(b+c)$.
    \item $a+\zzero=a$.
    \item $a+(-1 a)=a$.
    \item $(\alpha\beta)a=\alpha(\beta a)$.
    \item $1a=a$.
    \item $\alpha(a+b)=\alpha a+\alpha b$.
    \item $(\alpha+\beta)a=\alpha a+\beta a$.
  \end{enumerate}
\end{quiz}

\begin{quiz}
  %\solvelater{quiz:0:1}
  $\KK$を体とし, $V$を$\KK$を成分とする$(2,1)$行列の集合とする.
  $A$を$\KK$を成分とする$(2,2)$行列とし,
  \begin{align*}
    \shazo{\varphi}{V}{V}{x}{Ax}
  \end{align*}
  とする.
  このとき, 以下を示せ:
  \begin{enumerate}
    \item $a,b\in V\implies \varphi(a+b)=\varphi(a)+\varphi(b)$.
    \item $\alpha\in\KK a\in V\implies \varphi(\alpha a)=\alpha\varphi(a)$.
  \end{enumerate}
\end{quiz}


\chapter{線形空間と線形写像}
ここでは, 線形空間と線形写像を定義し,
いくつかの例を挙げる.
また, 線形空間が同型であるということを定義する.

\section{線形空間の定義とその例}
線形空間の定義と例を挙げる.

\begin{definition}
  $(\KK,+,\cdot,0,1)$を体とする.
  $V$を集合とする.
  $\pplus\colon V\times V \to V$を$V$上の二項演算とする.
  $\act \colon \KK\times V \to V$を$\KK$の$V$への作用とする.
  $0_V \in V$とする.

  以下の条件を満たすとき,
  $(V,\pplus,\act ,0_V)$が$\KK$上のベクトル空間とか$\KK$-線形空間であるいう:
  \begin{enumerate}
  \item 
    \begin{enumerate}
    \item $u,w\implies  u\pplus w=w\pplus u$.
    \item
      \label{def:vecsp:item:sum:ass}
      $v,w,u\in V \implies (v\pplus w)\pplus u=v\pplus (w\pplus u)$.
    \item $w\in V \implies 0_V\pplus w=w$.
    \item $w\in V\implies $ `$w\pplus x=0_V$を満たす$x\in V$が存在する'.
    \end{enumerate}
  \item
    \begin{enumerate}
    \item
      \label{def:vecsp:item:prod:ass}
      $a,b\in\KK, w\in V \implies a\act (b\act w)=(a\cdot b)\act w$
    \item $w\in V \implies 1\act w=w$.
    \end{enumerate}
  \item
    \begin{enumerate}
    \item $a\in \KK,u,w\in V \implies a\act (u\pplus w)=(a\act u)\pplus (a\act w)$.
    \item $a,b\in\KK, w\in V\implies (a+b)\act w=(a\act w)\pplus (b\act w)$.
    \end{enumerate}
  \end{enumerate}
\end{definition}
\begin{remark}
  $(V,+,\act ,0_V)$が$\KK$-線形空間
  であるとき,
  $+$を加法, $\act $をスカラー倍と呼ぶ.
  また, $0_V$を$V$の零元とか零ベクトルと呼ぶ.
  また, 
  $(V,+,\act ,0_V)$が$\KK$-線形空間であることを,
  $V$は加法$+$とスカラー倍$\act $で$0_V$を零元とする$\KK$-線形空間であるということもある.
  また, $+$, $\act $, $0_V$が文脈上明らかなときには,
  単に$V$を$\KK$-線形空間と呼ぶこともある.
\end{remark}
\begin{remark}
  $(V,+,\act ,0_V)$が$\KK$上のベクトル空間
  であるとき,
  $V$の元をベクトルと呼ぶ.
  また, 零元$0_V$のことを, 零ベクトルと呼ぶこともある.
\end{remark}

まず, ベクトル空間の例をいくつか挙げる.

\begin{example}
  $\KK$を体とし, $n\geq 1$とする.
  このとき,
  \begin{align*}
    \KK^n
    =\Set{\begin{pmatrix}a_1\\\vdots\\a_n\end{pmatrix}|a_1,\ldots,a_n\in \KK}
  \end{align*}
  とおく.
  このとき, $a_i,b_i,c\in\KK$に対し,
  \begin{align*}
    \begin{pmatrix}a_1\\\vdots\\a_n\end{pmatrix}+\begin{pmatrix}b_1\\\vdots\\b_n\end{pmatrix}
      &=\begin{pmatrix}a_1+b_1\\\vdots\\a_n+b_n\end{pmatrix},\\
    c\begin{pmatrix}a_1\\\vdots\\a_n\end{pmatrix}
    &=\begin{pmatrix}ca_1\\\vdots\\ca_n\end{pmatrix}
    \intertext{とし,}
    \zzero_n&=\begin{pmatrix}0\\\vdots\\a_n\end{pmatrix}
  \end{align*}
  とおくと,
  $(\KK^n,+,\act ,\zzero_n$は$\KK$-線形空間.
  これを$\KK$上の$n$次元数ベクトル空間と呼ぶ.
\end{example}

\begin{example}
  $\KK$を体とし, $n,m\geq 1$とし,
  $I=\Set{1,\ldots,m}$, $J=\Set{1,\ldots, n}$とする.
  このとき,
  \begin{align*}
    \KK^{m \times n}
    =\Set{(a_{i,j})_{i\in I,j\in J}|a_{i,j}\in \KK}
  \end{align*}
  とおく.
  このとき, $a_{i,j},b_{i,j},c\in\KK$に対し,
  \begin{align*}
    (a_{i,j})_{i\in I,j\in J}
    +(b_{i,j})_{i\in I,j\in J}
      &=(a_{i,j}+b_{i,j})_{i\in I,j\in J},\\
    c(a_{i,j})_{i\in I,j\in J}
    &=(ca_{i,j})_{i\in I,j\in J}
    \intertext{とし,}
    O_{m,n}&=(0)_{i\in I,j\in J}
  \end{align*}
  とおくと,
  $(\KK^{m \times n},+,\act ,O_{m,n})$は$\KK$-線形空間.
\end{example}

\begin{example}
  $\NN$で添字付けられた数列$a_0, a_1,\ldots$を
  $(a_i)_{i\in \NN}$
  で表す.
  $\KK$を体とする.
  \begin{align*}
    \KK^\NN = \Set{(a_i)_{i\in \NN}|a_i\in \KK}
  \end{align*}
  とおく.
  次の和とスカラー倍で, $(0)_{i\in \NN}$を零元とする$\KK$-線形空間:
  $c\in\KK$, $(a_i)_{i\in \NN},(b_i)_{i\in \NN}\in \KK^\NN$に対し,
  \begin{align*}
    (a_i)_{i\in \NN}+(b_i)_{i\in \NN} &= (a_i+b_i)_{i\in \NN},\\
    c(a_i)_{i\in \NN} &= (ca_i)_{i\in \NN}.
  \end{align*}
\end{example}

\begin{example}
  $\KK$を体とする.
  $X$を集合とする.
  \begin{align*}
    \KK^X = \Set{f\colon X \to \KK \text{; 写像}}
  \end{align*}
  とする.
  $c\in\KK$, $f,g\in \KK^X$に対し,
  $f+g\in \KK^X$と$c\act f\in\KK^X$を以下で定める:
  $x\in X$に対し,
  \begin{align*}
    (f+g)(x) &= f(x)+g(x),\\
    (c\act f)(x) &= c(f(x)).
  \end{align*}
  また $\underline{0}\in\KK^X$を以下で定める:
  $x\in X$に対し,
  \begin{align*}
    \underline{0}(x)=0.
  \end{align*}
  このとき, $(\KK^X,+,\act ,\underline{0})$は,
  $\KK$-線形空間.
\end{example}

\begin{example}
  $\KK$を体とする.
  $X$を集合とし, $(V,+,\act,0_V)$を$\KK$-線形空間とする.
  \begin{align*}
    V^X = \Set{f\colon X \to V \text{; 写像}}
  \end{align*}
  とする.
  $c\in\KK$, $f,g\in V^X$に対し,
  $f+g\in V^X$と$c\act f\in V^X$を以下で定める:
  $x\in X$に対し,
  \begin{align*}
    (f+g)(x) &= f(x)+g(x),\\
    (c\act f)(x) &= c\act (f(x)).
  \end{align*}
  また $\underline{0_V}\in\KK^X$を以下で定める:
  $x\in X$に対し,
  \begin{align*}
    \underline{0_V}(x)=0_V.
  \end{align*}
  このとき, $(\KK^X,+,\act ,\underline{0_V})$は,
  $\KK$-線形空間.
\end{example}

\begin{example}
  $(\KK,+,\cdot,0)$を体とする.
  体としての和と積で$\KK$-線形空間.
  つまり, $(\KK,+,\cdot,0)$は, $\KK$-線形空間.
\end{example}

\begin{example}
  $\CC=\Set{a+b\sqrt{-1}|a,b\in \RR}$は,
  通常の和と積で$\CC$-線形空間.
  また,
  $\CC$は,
  通常の和と積で$\RR$-線形空間でもある.
\end{example}

\begin{example}
  \begin{align*}
    V=\Set{a+b\sqrt{2}|a,b\in\QQ}
  \end{align*}
  とすると,
  通常の和と積で$\QQ$-線形空間.
\end{example}

\begin{example}
  $\zeta_n= e^{\frac{2\pi\sqrt{-1}}{n}}=\cos(\frac{2\pi}{n})+\sqrt{-1}\sin(\frac{2\pi}{n})$
  とする.
  \begin{align*}
    V=\Set{\sum_{i=0}^{n}a_i\zeta_n^i|a_i\in\QQ}
  \end{align*}
  とすると,
  通常の和と積で$\QQ$-線形空間.
\end{example}

\begin{example}
  $\CC$は,
  通常の和と積で$\RR$-線形空間.
\end{example}



表記方法などについてのいくつかコメントする.
\begin{remark}
\Cref{def:vecsp:item:sum:ass}
があるので, $v+w+u$を, $(v+w)+u$と思っても,
$v+(w+u)$と思っても差し支えない.
そこで, $(v+w)+u$を$v+w+u$と略記する.
\end{remark}
\begin{remark}
\Cref{def:vecsp:item:prod:ass}
があるので, $abw$を, $a\act (b\act w)$と思っても,
$(a\cdot b)\act w$と思っても差し支えない.
そこで, $a\act (b\act w)$を$abw$と略記する.
もっと一般に, $n\geq 1$, $a_1,\ldots, a_n\in\KK$, $w\in V$に対し,
$a_1\act (a_2\act \ldots\act (a_n\act w)\cdots)$のことを
$a_1\cdots a_n w$と略記する.
\end{remark}
\begin{remark}
  $\KK$上のベクトル空間の定義において,
  割り算は条件として現れない.
  したがって,
  これらの条件を, ある可換環に対して満たすものという概念を定義することができる.
\end{remark}
\begin{prop}
  \label{lem:uniq:zero}
  $(V,+,\act ,0_V)$を$\KK$-線形空間とする.
  $o\in V$が, 次の条件を満たすとする:
  \begin{align*}
    w\in V \implies w+o=w.
  \end{align*}
  このとき, $o=0_V$.
\end{prop}
\begin{proof}
  $0_V$の満たす条件から$0_V+o=o$.
  一方, $o$の満たす条件から$0_V+o=0_V$.
  よって, $0_V=0_V+o=o$.
\end{proof}
\begin{remark}
  $u,w\in V$に対し, $u+w=w+u$であるので,
  次の条件は同値である:
  \begin{enumerate}
  \item $w\in V \implies w+o=w$.
  \item $w\in V \implies o+w=w$.
  \end{enumerate}
  したがって, \cref{lem:uniq:zero}から,
  零ベクトルの条件を満たす元はただ一つしかないことがわかる.
  何が$V$の零ベクトルであるかを明示しなくともよい場合は,
  単に$(V,+,\act )$を$\KK$-線形空間と呼ぶこともある.
\end{remark}
\begin{prop}
  \label{prop:zeroveciszeroscalar}
  $(V,+,\act ,0_V)$を$\KK$-線形空間とする.
  このとき,
  \begin{align*}
    w\in V \implies 0\act w=0_V. 
  \end{align*}
\end{prop}
\begin{proof}
  $x\in x$は
  $0\act w+x=0_V$を満たすとする.
  このとき,
  \begin{align*}
    (0\act w+0\act w)+x&=(0+0)\act w+x=0\act w+x=0_V\\
    0\act w+(0\act w+x)&=0\act w+0_V=0\act w
  \end{align*}
  よって, $0\act w=0_V$.  
\end{proof}

\begin{lemma}
  \label{lem:uniq:inv}
  $(V,+,\act ,0_V)$を$\KK$-線形空間とする.
  $w\in V$とする.
  $x,y\in V$が以下を満たすなら$x=y$:
  \begin{align*}
    w+x&=0_V\\
    y+w&=0_V
  \end{align*}
\end{lemma}
\begin{proof}
  $x$の満たす条件から,
  $y+(w+x)=y+0_V=y$.
  $y$の満たす条件から,
  $(y+w)+x=0_V+x=x$.
  よって$x=y$.
\end{proof}
\begin{remark}
  $u,w\in V$に対し, $u+w=w+u$であるので,
  次の条件は同値である:
  \begin{enumerate}
  \item $w+x=0_V$.
  \item $x+w=0_V$.
  \end{enumerate}
  したがって, \cref{lem:uniq:inv}から,
  各$w\in V$に対し, $w+x=0_V$を満たす$x\in V$は
  ただ一つしかないことがわかる.
\end{remark}
\begin{prop}
  $(V,+,\act ,0_V)$を$\KK$-線形空間とする.
  $w,x\in V$に対し,
  以下は同値:
  \begin{enumerate}
  \item
    \label{prop:inverse:-1:item:1}
    $w+x=0_V$.
  \item
    \label{prop:inverse:-1:item:2}
    $x=(-1)\act w$.    
  \end{enumerate}
\end{prop}
\begin{proof}
  \Cref{prop:zeroveciszeroscalar}より,
  \begin{align}
    w+(-1)\act w=(1-1)\act w=0\act w=0_V.
  \end{align}
  したがって,
  $w+x=0_V$を満たす
  $x$は$(-1)\act w$に等しいことがわかる.
\end{proof}

\begin{remark}
  $(-1)\act w$のことを$-w$と略記する.
  また, $u+(-w)$のことを$u-w$と略記する.
\end{remark}




\section{線形写像の定義とその例}
\begin{definition}
  $(V,+,\act)$, $(W,\pplus,\aact)$を$\KK$-線形空間とする.
  $\varphi$を$V$から$W$への写像とする.
  $\varphi$が以下の条件を満たすとき,
  $\varphi$は$(V,+,\act)$から$(W,\pplus,\aact)$への$\KK$-線形写像であるという:
  \begin{enumerate}
    \item $v,u\in V\implies \varphi(v+u)=\varphi(v)\pplus\varphi(u)$.
    \item $c\in \KK, u\in V\implies \varphi(c\act v)=c\aact\varphi(v)$.
  \end{enumerate}
\end{definition}

\begin{definition}
  $\varphi$は$(V,+,\act)$から$(W,\pplus,\aact)$への$\KK$-線形写像であることを,
  写像$\varphi$は$\KK$-線形であるということもある.
\end{definition}

線形写像の例をいくつか挙げる.
\begin{example}
  $\KK$を体とする.
  $A\in \KK^{m\times n}$とし,
  $\varphi$を次の写像とする:
  \begin{align*}
    \shazo{\varphi}{\KK^m}{\KK^n}{w}{Aw}.
  \end{align*}
  このとき, $\varphi$は$\KK$-線形である.
\end{example}

\begin{example}
  $\KK$を体とする.
  $A\in \KK^{m\times n}$とし,
  $\varphi$を次の写像とする:
  \begin{align*}
    \shazo{\varphi}{\KK^{k\times m}}{\KK^{k\times n}}{X}{AX}.
  \end{align*}
  このとき, $\varphi$は$\KK$-線形である.
\end{example}

\begin{example}
  $\KK$を体とする.
  $\varphi$を次の写像とする:
  \begin{align*}
    \shazo{\varphi}{\KK^{m\times n}}{\KK^{n\times m}}{A}{\transposed{A}}.
  \end{align*}
  このとき, $\varphi$は$\KK$-線形である.
\end{example}

\begin{example}
  $\varphi$を次の写像とする:
  \begin{align*}
    \shazo{\varphi}{\CC}{\CC}{z}{\overline{z}},
  \end{align*}
  ただし, 
  $x,y\in\RR$に対し$\overline{x+y\sqrt{-1}}=x-y\sqrt{-1}$, つまり,
  $\overline{z}$は$z$の複素共軛とする.
  このとき, $\varphi$は$\RR$-線形である.
  しかし, $\CC$-線形ではない.
\end{example}

\begin{example}
  $a\in\RR$とする.
  $\varphi$を次の写像とする:
  \begin{align*}
    \shazo{\varphi}{\RR}{\RR}{x}{ax}.
  \end{align*}
  これは$\RR$-線形写像である.
  $\psi$を次の写像とする:
  \begin{align*}
    \shazo{\psi}{\RR}{\RR}{x}{ax+1}.
  \end{align*}
  これは$\RR$-線形写像ではない.
  $\phi$を次の写像とする:
  \begin{align*}
    \shazo{\phi}{\RR}{\RR}{x}{x^2}.
  \end{align*}
  これは$\RR$-線形写像ではない.
\end{example}

\begin{example}
  $\KK$を体とする.
  $\varphi$を次の写像とする:
  \begin{align*}
    \shazo{\varphi}{\KK^\NN}{\KK^\NN}{(a_i)_{i\in \NN}}{(a_{i+1})_{i\in \NN}}.
  \end{align*}
  このとき, $\varphi$は$\KK$-線形である.
\end{example}

\begin{example}
  $(V,+,\act)$を$\KK$-線形空間とする.
  恒等写像$\id_V$は$\KK$-線形である.
\end{example}
\begin{lemma}
  $(V,+,\act)$, $(W,\pplus,\aact)$を$\KK$-線形空間とし,
  $\varphi\colon V\to W$は全単射であるとする.
  $\varphi$が$\KK$-線形なら,
  逆写像$\varphi^{-1}$は
  $(W,\pplus,\aact)$から
  $(V,+,\act)$への$\KK$-線形写像.
\end{lemma}
\begin{proof}\end{proof}

線形写像の性質をいくつか紹介する.
\begin{prop}
  $V$, $U$, $W$を$\KK$-線形空間とし,
  $\varphi\colon V\to U$,
  $\psi\colon U\to W$を$\KK$-線形写像とする.
  このとき, $\psi\circ\varphi\colon V\to W$は$\KK$-線形写像である.
\end{prop}
\begin{proof}\end{proof}

\begin{prop}
  $(V,+,\act,0_V)$, $(W,\pplus,\aact,0_W)$を$\KK$-線形空間とし,
  $\varphi\colon V\to W$を$\KK$-線形写像とする.
  このとき,
  \begin{enumerate}
    \item $\varphi(0_V)=0_W$.
    \item $\varphi(-x)=-\varphi(x)$.
  \end{enumerate}
\end{prop}
\begin{proof}\end{proof}


\section{同型写像}

\begin{definition}
  $V$, $W$を$\KK$-線形空間とする.
  以下の条件を満たす
  $\varphi\colon V\to W$を,
  $V$から$W$への($\KK$-線形空間としての)同型写像と呼ぶ:
  \begin{enumerate}
  \item $\varphi\colon V\to W$は$\KK$-線形写像.
  \item 以下の条件を満たす$\KK$-線形写像$\psi\colon W \to V$が存在する:
    \begin{enumerate}
      \item $\varphi\circ \psi=\id_W$.
      \item $\psi\circ \varphi=\id_V$.
    \end{enumerate}
  \end{enumerate}
\end{definition}
\begin{definition}
  $V$, $W$を$\KK$-線形空間とする.
  $V$から$W$への($\KK$-線形空間としての)同型写像が存在するとき,
  $V$と$W$は($\KK$-線形空間として)同型であるといい,  
  $V\simeq W$と書く.
\end{definition}

同型写像の例を挙げる:
\begin{example}
  $\KK$を体とする.
  $\KK$-線形写像$\varphi$を以下で定める:
  \begin{align*}
    &\shazo{\varphi}{\RR^{m\times n}}{\RR^{n\times m}}{A}{\transposed{A}}.
  \end{align*}
  このとき,
  $\psi$を
  \begin{align*}
    &\shazo{\psi}{\RR^{n\times m}}{\RR^{m\times n}}{A}{\transposed{A}}
  \end{align*}
  とすると,
  $\psi$は$\KK$-線形であり, $\varphi\circ \psi=\id_{\KK^{n\times m}}$, $\psi\circ\varphi=\id_{\KK^{m\times n}}$である.
  よって, $\varphi$は同型写像である.
  したがって, $\KK^{m\times n}\simeq\KK^{n\times m}$である.
\end{example}


\begin{example}
  $\RR$-線形写像$\varphi$を以下で定める:
  \begin{align*}
    &\shazo{\varphi}{\RR^2}{\CC}{\begin{pmatrix}x\\y\end{pmatrix}}{x+y\sqrt{-1}}.
  \end{align*}
  このとき,
  $\psi$を
  \begin{align*}
    &\shazo{\varphi}{\RR^2}{\CC}{\begin{pmatrix}x\\y\end{pmatrix}}{x+y\sqrt{-1}},\\
   &\shazo{\psi}{\CC}{\RR^2}{x+y\sqrt{-1}}{\begin{pmatrix}x\\y\end{pmatrix}}.
  \end{align*}
  とすると,
  $\RR$-線形であり,
  $\varphi\circ \psi=\id_\CC$,
  $\psi\circ\varphi=\id_{\RR^2}$である.
  よって, 同型写像であり,
  $\RR$-線形空間として$\RR^2\simeq\CC$である.
\end{example}

\begin{example}
  $I={1,\ldots,m}$, $J={1,\ldots,n}$, $\Lambda=\Set{1,\ldots,mn}$とする.
  このとき,
  \begin{align*}
    &\shazo{\lambda}{I\times J}{\Lambda}{(i,j)}{i+(m-1)j}
  \end{align*}
  は全単射である.  $\lambda$の逆写像を$\kappa$とする.
  $\KK$を体とし,
  $\KK$線形写像$\varphi$を以下で定める:
  \begin{align*}
    &\shazo{\varphi}{\KK^{mn}}{\KK^{m\times n}}{\begin{pmatrix}a_1\\\vdots\\a_{mn}\end{pmatrix}}{(a_{\lambda(i,j)})_{i\in I, j\in J}}.
  \end{align*}
  $\psi$を
  \begin{align*}
    &\shazo{\psi}{\KK^{m\times n}}{\KK^{mn}}
    {(a_{i,j})_{i\in I, j\in J}}{\begin{pmatrix}a_{\kappa(1)}\\a_{\kappa(2)}\\\vdots\\a_{\kappa(mn)}\\\end{pmatrix}}
  \end{align*}
  とすると,
  $\KK$-線形であり,
  $\varphi\circ \psi=\id_{\KK^{m\times n}}$,
  $\psi\circ\varphi=\id_{\KK^{mn}}$である.
  よって, 同型写像であり,
  $\KK^{mn}\simeq \KK^{m\times n}$.
\end{example}


\begin{example}
  $V$を$\KK$-線形空間とする.
  このとき, $\id_{V}$は$\KK$-線形であり,
  $\id_{V}\circ\id_{V}=\id_{V}$であるので, 同型写像である.
  したがって, $V\simeq V$である.
\end{example}

\begin{example}
  $\varphi$を次の写像とする:
  \begin{align*}
    \shazo{\varphi}{\CC}{\CC}{z}{\overline{z}},
  \end{align*}
  このとき, $\varphi$は$\RR$-線形であり, $\varphi\circ\varphi=\id_{\CC}$である.
  $\varphi$は, $\CC$から$\CC$への$\RR$-線形空間としての同型写像である.
\end{example}


\begin{example}
  $\KK$を体とし, $I=\Set{1,\dots,n}$とする.
  $\varphi$を以下で定義する:
  \begin{align*}
    \shazo{\varphi}{\KK^I}{\KK^n}{f}{\begin{pmatrix}f(1)\\\vdots\\f(n)\end{pmatrix}}.
  \end{align*}
  $\varphi$は$\KK$-線形写像である.
  逆に,
  \begin{align*}
    a=
    \begin{pmatrix}
      a_1\\\vdots\\a_n
    \end{pmatrix}
  \end{align*}
  に対し,
  $I$から$\KK$への
  写像$f_a$を
  \begin{align*}
    \shazo{f_a}{I}{\KK}{i}{a_i}
  \end{align*}
  で定める.
  次の写像を考える:
  \begin{align*}
    \shazo{\psi}{\KK^n}{\KK^I}{a}{f_a}.
  \end{align*}
  つまり$\psi$以下で定義する.
  \begin{align*}
    \shazo{\psi}{\KK^n}{\KK^I}{a=
    \begin{pmatrix}
      a_1\\\vdots\\a_n
    \end{pmatrix}
  }{\left(\shazo{f_a}{I}{\KK}{i}{a_i}\right)}.
  \end{align*}
  このとき$\psi$も$\KK$-線形写像であり,
  $\varphi\circ\psi=\id_{\KK^n}$,
  $\psi\circ\varphi=\id_{\KK^I}$である.
  よって$\varphi$は同型写像であり,
  $\KK^I\simeq \KK^n$.
\end{example}

\begin{example}
  $\KK$を体とし, $I=\Set{1,\dots,m}$, $J=\Set{1,\ldots,n}$とする.
  $\varphi$を以下で定義する:
  \begin{align*}
    \shazo{\varphi}{\KK^{I\times J}}{\KK^{m\times n}}{f}{(f(i,j))_{i\in I,j\in J}}.
  \end{align*}
  $\varphi$は$\KK$-線形写像である.
  逆に,
  \begin{align*}
    A=
      (a_{i,j})_{i\in I,j\in J}
  \end{align*}
  に対し,
  $I\times J$から$\KK$への
  写像$f_a$を
  \begin{align*}
    \shazo{f_A}{I\times J}{\KK}{(i,j)}{a_{i,j}}
  \end{align*}
  で定め,
  次の写像を考える:
  \begin{align*}
    \shazo{\psi}{\KK^{m\times n}}{\KK^{I\times J}}{A}{f_A}.
  \end{align*}
  つまり$\psi$以下で定義する.
  \begin{align*}
    \shazo{\psi}{\KK^{m\times n}}{\KK^{I\times J}}
          {A=(a_{i,j})_{i\in I,j\in J}}{\left(\shazo{f_A}{I\times J}{\KK}{(i,j)}{a_{i,j}}\right)}.
  \end{align*}
  このとき$\psi$も$\KK$-線形写像であり,
  $\varphi\circ\psi=\id_{\KK^{m\times n}}$,
  $\psi\circ\varphi=\id_{\KK^{I\times J}}$である.
  よって$\varphi$は同型写像であり,
  $\KK^{I\times J}\simeq \KK^{m\times n}$.
\end{example}

\begin{example}
  $\KK$を体とし,
  $\NN$から$\KK$への写像を集めた集合を$V$,
  $\NN$で添字付された数列を集めた集合を$W$とする.
  どちらも$\KK$-線形空間であった.
  $\varphi$を以下で定義する:
  \begin{align*}
    \shazo{\varphi}{V}{W}{f}{(f(i))_{i\in \NN}}.
  \end{align*}
  $\varphi$は$\KK$-線形写像である.
  逆に,
  \begin{align*}
    a=
      (a_{i})_{i\in \NN}
  \end{align*}
  に対し,
  $\NN$から$\KK$への
  写像$f_a$を
  \begin{align*}
    \shazo{f_a}{\NN}{\KK}{i}{a_{i}}
  \end{align*}
  で定め,
  次の写像を考える:
  \begin{align*}
    \shazo{\psi}{W}{V}{a}{f_a}.
  \end{align*}
  つまり$\psi$以下で定義する.
  \begin{align*}
    \shazo{\psi}{W}{V}{a=(a_{i})_{i\in \NN}}{\left(\shazo{f_a}{\NN}{\KK}{i}{a_{i}}\right)}.
  \end{align*}
  このとき$\psi$も$\KK$-線形写像であり,
  $\varphi\circ\psi=\id_{W}$,
  $\psi\circ\varphi=\id_{V}$である.
  よって$\varphi$は同型写像であり,
  $V\simeq W$.
\end{example}


\begin{remark}
  $V$と$W$を$\KK$-線形空間とする.
  $\varphi\colon V\to W$が$\KK$-線形写像であるとき,
  $V$において和やスカラー倍に関してなりたつことは,
  $\varphi$を通して, $W$における命題に翻訳できる.
  $\varphi\colon V\to W$が同型写像であるなら,
  $\KK$-線形写像$\psi\colon W\to V$で
  $\varphi\circ \psi=\id_W$と
  $\psi\circ \varphi=\id_V$を満たすものが存在する.
  $W$において和やスカラー倍に関してなりたつことは,
  $\psi$を通して, $V$における命題に翻訳できる.
  $\varphi$と$\psi$を使い, 双方を自由に行き来できるので,
  $V$と$W$は$\KK$-線形空間として同一視できる.
\end{remark}

\begin{prop}
  $V$, $W$, $U$を$\KK$-線形空間とする.
  このとき, 以下が成り立つ:
  \begin{enumerate}
  \item $V\simeq W$.
  \item $V\simeq W \implies W\simeq V$.
  \item $V\simeq U, U\simeq W \implies V\simeq W$.
  \end{enumerate}
\end{prop}
\begin{proof}\end{proof}

\begin{prop}
  $V$, $W$を$\KK$-線形空間とし, $\varphi\colon V\to W$を$\KK$-線形写像とする.
  このとき, 以下は同値:
  \begin{enumerate}
  \item $\varphi$は同型写像.
  \item $\varphi$は全単射.
  \end{enumerate}
\end{prop}
\begin{proof}\end{proof}

\sectionX{章末問題}
\begin{quiz}
  %\solvelater{quiz:1:1}
\end{quiz}

\chapter{部分空間}

\sectionX{章末問題}
\begin{quiz}
  %\solvelater{quiz:1:1}
\end{quiz}

\chapter{商空間}
\sectionX{章末問題}
\begin{quiz}
  %\solvelater{quiz:1:1}
\end{quiz}
