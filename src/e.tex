
\chapter{練習問題}

ここでは,
練習問題を挙げてある.
本原稿を前から順番に読み進める場合に,
どのあたりまで読んだら解けるかという目安ごとに,
問題を分けてある.
問題Aは, 次につながる問題であるので,
読んだ直後に目を通すとよいと思う.

中には, 本質的には同じ問題が何度も出てくることもある.

\section{\Cref{chap:prelim}の後の問題}
主に集合と写像, 体, それから行列に関する基本的な事項について知っていれば,
解けるであろう問題.
%\section{1/15回目: \schoolCalender{1}}
% 講義: 集合と写像, 体.
% \subsection{レポート課題 (締め切り: \schoolCalender{2})}
\subsection{問題A}
\begin{quiz}
  $\KK$を体とし, $V$を$\KK$を成分とする$(2,1)$行列の集合とする.
  つまり,
  \begin{align*}
    V=\Set{\begin{pmatrix}a_1\\a_2\end{pmatrix}|a_1,a_2\in \KK}
  \end{align*}
  とする. 通常の和とスカラー倍を考える.  つまり,
  \begin{align*}
    a&=\begin{pmatrix}a_1\\a_2\end{pmatrix}\in V,&  b&=\begin{pmatrix}b_1\\b_2\end{pmatrix} \in V,
    &\alpha\in\KK&
  \end{align*}
  に対し,
  \begin{align*}
    a+b&=\begin{pmatrix}a_1+b_1\\a_2+b_2\end{pmatrix}\in V, & \alpha a=\begin{pmatrix}\alpha a_1\\\alpha a_2\end{pmatrix} \in V
  \end{align*}
  とする. また,
  \begin{align*}
  \zzero=
  \begin{pmatrix}
    0\\0
  \end{pmatrix}
  \end{align*}
  とする.

  このとき, $\alpha,\beta,1,-1\in\KK$, $a,b,c\in V$に対し, 以下を示せ:
  \begin{enumerate}
    \item 
    \begin{enumerate}
    \item $a+b=b+a$.
    \item $(a+b)+c=a+(b+c)$.
    \item $a+\zzero=a$.
    \item
      $a+(-1 a)=\zzero$.
    \end{enumerate}
  \item
    \begin{enumerate}
    \item $(\alpha\beta)a=\alpha(\beta a)$.
    \item $1a=a$.
    \end{enumerate}
  \item
    \begin{enumerate}
    \item $\alpha(a+b)=\alpha a+\alpha b$.
    \item $(\alpha+\beta)a=\alpha a+\beta a$.
    \end{enumerate}
  \end{enumerate}
\end{quiz}



\subsection{問題B}
$\KK$を体とし, $\KK$の元を成分とする行列について考える.

\begin{quiz}
  次の命題の真偽を判定し, 証明を与えよ:
  \begin{enumerate}
  \item
    $A$, $B$がともに正則な$n$次正方行列である
    とする.
    このとき, $AB$も正則で, その逆行列は$B^{-1}A^{-1}$.
  \item
    $A$を正則な$n$次正方行列とする.
    このとき, $\transposed{A}$も正則で, その逆行列は$\transposed{(A^{-1})}$.
  \item
    $A$を正則な$n$次正方行列とし,
    $c$は逆数を持つ$\KK$の元とする.
    このとき, $cA$も正則で, その逆行列は$\frac{1}{c}A^{-1}$.
  \end{enumerate}
\end{quiz}

\begin{quiz}
  次の命題の真偽を判定し, 証明を与えよ:
  \begin{enumerate}
  \item
    単位行列$E_n$は正則で, その逆行列は$E_n$.
  \item
    $A$は正則であるとする.
    このとき,
    $A^{-1}$は正則で, その逆行列は$A$.
  \end{enumerate}
\end{quiz}


\begin{quiz}
  次の命題の真偽を判定し, 証明を与えよ:
  \begin{enumerate}
  \item
    $A$, $B$がともに正則な$n$次正方行列である
    とする.
    このとき, $A+B$は正則であるとは限らない.
  \end{enumerate}
\end{quiz}

\begin{quiz}
  $v$, $w$を$(2,1)$-行列 (2項数ベクトル) とし,
  $A$を2次正方行列とする.
  $v$, $w$を並べて得られる2次正方行列を$P=(v|w)$とおく.
  $Av=\lambda v$, $Aw=\mu w$を満たす$\lambda,\mu\in\KK$が存在するとする.
  このとき,
  次の命題の真偽を判定し, 証明を与えよ:
  \begin{enumerate}
  \item $AP=PD$, ただし,
    \begin{align*}
      D=\begin{pmatrix}\lambda&0\\0&\mu\end{pmatrix}.
    \end{align*}
  \end{enumerate}
\end{quiz}

\begin{quiz}
  $A$を$(m,n)$行列とし,
  $V$を$(n,1)$行列 ($n$項数ベクトル)を集めた集合とする.
  \begin{align*}
    \KKK=\Set{\xx\in V | A\xx=\zzero}
  \end{align*}
  とする.
  このとき,
  次の命題の真偽を判定し, 証明を与えよ:
  \begin{enumerate}
  \item
    $\aaa,\bbb\in\KKK \implies\aaa+\bbb\in\KKK$.
  \item
    $\aaa\in\KKK, \alpha\in\KK \implies \alpha\aaa\in\KKK$.
  \end{enumerate}
\end{quiz}

\begin{quiz}
  $A$を$(m,n)$行列とし,
  $\bbb$を$(m,1)$行列($m$項数ベクトル)とする.
  $V$を$(n,1)$行列 ($n$項数ベクトル)を集めた集合とし,
  \begin{align*}
    \KKK=\Set{\xx\in V | A\xx=\zzero}
  \end{align*}
  とする.
  $\vv\in V$が
  $A\vv =\bbb$を満たすとする.
  \begin{align*}
    \FFF&=\Set{\xx\in V | A\xx=\bbb}\\
    \FFF'&=\Set{\vv+\aaa | \aaa\in \KKK}
  \end{align*}
  とする.
  
  このとき,
  次の命題の真偽を判定し, 証明を与えよ:
  \begin{enumerate}
  \item
    $\FFF=\FFF'$.
  \end{enumerate}
   
\end{quiz}


\begin{quiz}
  次の行列が正則かどうか判定し正則なら逆行列を求めよ:
  \begin{enumerate}
  \item
    $A=
    \begin{pmatrix}
      1&1&0&1\\
      1&-1&1&2\\
      1&1&1&4\\
      1&-1&1&8
    \end{pmatrix}$.
  \end{enumerate}
\end{quiz}


\section{\Cref{sec:linspace:def}の後の問題}
主に線形空間の定義を知っていれば解けるであろう問題.
%\endinput
%\newpage
%\section{2/15回目: \schoolCalender{2}}
% 講義: 線形空間の定義と例.
%\subsection{レポート課題 (締め切り: \schoolCalender{3})}
\subsection{問題A}
\begin{quiz}
  %\solvelater{quiz:0:1}
  $\KK$を体とし, $V$を$\KK$を成分とする$(2,1)$行列の集合とする.
  このとき, $V$は$\KK$線形空間であった.
  $A$を$\KK$を成分とする$(2,2)$行列とし,
  \begin{align*}
    \shazo{\varphi}{V}{V}{x}{Ax}
  \end{align*}
  とする.
  このとき, 以下を示せ:
  \begin{enumerate}
    \item $a,b\in V\implies \varphi(a+b)=\varphi(a)+\varphi(b)$.
    \item $\alpha\in\KK, a\in V\implies \varphi(\alpha a)=\alpha\varphi(a)$.
  \end{enumerate}
  (ただし, 行列の演算に関する基本的な性質は既知としてよい.)
\end{quiz}

\subsection{問題B}

\begin{quiz}
  $V$を$\KK$-線形空間とする.
  $o,o'\in V$が以下を満たすとする:
  \begin{enumerate}
    \item $x\in V \implies  o+x=x+o=x$.
    \item $x\in V \implies  o'+x=x+o'=x$.
  \end{enumerate}
  このとき, $o=o'$となることを示せ.
\end{quiz}

\begin{quiz}
  $0_V$を
  $\KK$-線形空間$V$の零元(零ベクトル)とする.
  $a\in V$とする.
  このとき, $x,x'\in V$が以下を満たすとする:
  \begin{enumerate}
    \item $a+x=x+a=0_V$.
    \item $a+x'=x'+a=0_V$.
  \end{enumerate}
  このとき, $x=x'$となることを示せ.
\end{quiz}

\begin{quiz}
  $\KK$を体とし,
  $V$を$\KK$ベクトル空間とする.
  $V$の零元を$0_V$とする.
  $v\in V$, $\alpha\in \KK$に対し次が成り立つことを示せ:
  \begin{enumerate}
    \item $\alpha v=0_V\implies \alpha=0$ または$v=0_V$.
  \end{enumerate}
\end{quiz}

\begin{quiz}
  $V$, $W$を
  $\KK$-線形空間とする.
  このとき,
  $V$と$W$の外部直和$V\boxplus W$が
  $\KK$-線形空間であることを示せ.
\end{quiz}

\begin{quiz}
  $V=\Set{x\in\RR|x>0}$とする.
  $x,y\in V$に対し,
  $x \pplus y =x\cdot y$
  とし,
  $x\in V$, $a\in \RR$に対し,
  $a\aact x=x^a$とする.
  このとき,
  $V$は和$\pplus$とスカラー倍$\aact$で
  $\RR$線形空間であることを示せ.
\end{quiz}

\begin{quiz}
  次の命題の真偽を判定し証明を与えよ:
  \begin{enumerate}
  \item $\CC$は通常の和と積で$\RR$線形空間である.
  \item $\RR$は通常の和と積で$\CC$線形空間ではない.
  \end{enumerate}
\end{quiz}

\begin{quiz}
$\NN$で添字付けられた数列$a_0, a_1,\ldots$を$(a_i)_{i\in \NN}$で表す.
  $\ell(\KK) = \Set{(a_i)_{i\in \NN}|a_i\in \KK}$とする.
  $\alpha\in \KK$, $(a_i)_{i\in \NN},(b_i)_{i\in \NN}\in \ell(\KK)$
  に対し和を
  $(a_i)_{i\in \NN}+(b_i)_{i\in \NN} = (a_i+b_i)_{i\in \NN}$で定義し,
  $\alpha\in \KK$, $(a_i)_{i\in \NN}\in \ell(\KK)$
  に対しスカラー倍を
  $\alpha (a_i)_{i\in \NN} = (\alpha a_i)_{i\in \NN}$
  で定義する.
  このとき, $\ell(\KK)$は
  この和とスカラー倍で$\KK$-線形空間となることを示せ.
\end{quiz}

\begin{quiz}
  $S$を集合とし, $V$を$\KK$-線形空間とする.
  $V^S= \Set{f\colon S \to V \text{; 写像}}$とおく.
  $\alpha\in\KK$, $f,g\in V^S$に対し,
  和$f+g\in V^S$とスカラー倍$\alpha f\in V^S$を以下で定める:
  $x\in S$に対し,
  \begin{align*}
    (f+g)(x) &= f(x)+g(x),\\
    (\alpha f)(x) &= \alpha.(f(x)).
  \end{align*}
  このとき,
  $\KK^S$はこの和とスカラー倍で
  $\KK$-線形空間となることを示せ.
\end{quiz}



\section{\Cref{sec:linmap:def}のあとの問題}
主に線形写像の定義を知っていれば解けるであろう問題.
%\endinput
%\newpage
%\section{3/15回目: \schoolCalender{3}}
% 講義: 線形写像の定義と例.
%\subsection{レポート課題 (締め切り: \schoolCalender{4})}
\subsection{問題A}
\begin{quiz}
  $\KK$を体とし, $V$を$\KK$を成分とする$(2,1)$行列の集合とする.
  $A$を$\KK$を成分とする$(2,2)$行列とし,
  \begin{align*}
    \shazo{\varphi}{V}{V}{x}{Ax}
  \end{align*}
  とする. $A$が正則であるとする.
  このとき, 以下の条件を満たす線形写像$\psi\colon V\to V$が存在することを示せ:
  \begin{enumerate}
    \item $\varphi\circ \psi=\id_V$.
    \item $\psi\circ \varphi=\id_V$.
  \end{enumerate}
\end{quiz}
\begin{quiz}
  $\KK$を体とし,
  $V$を$\KK$を成分とする$(3,1)$行列のなす$\KK$-線形空間とし,
  $W$を$\KK$を成分とする$(2,1)$行列のなす$\KK$-線形空間とする.
  $A$を$\KK$を成分とする$(3,2)$行列とし,
  $\zzero$を$W$の零元とする.
  このとき, 次は$\KK$-線形空間であることを示せ:
  \begin{align*}
    \Set{x\in V|Ax=\zzero}.
  \end{align*}
\end{quiz}

\subsection{問題B}
\begin{quiz}
  次の命題の真偽を判定し証明を与えよ:
  $A\in \KK^{n\times n}$とし,
  $\varphi$を次の写像とする:
  \begin{align*}
    \shazo{\varphi_A}{\KK^{n\times n}}{\KK^{n\times n}}{X}{AX-XA}.
  \end{align*}
  このとき, $\varphi_A$は$\KK$-線形である.
\end{quiz}

\begin{quiz}
  次の命題の真偽を判定し証明を与えよ:
  $I=\Set{1,2,\ldots, n}$する.
  $\tr$を次の写像とする:
  \begin{align*}
    \shazo{\tr}{\KK^{n\times n}}{\KK}{(a_{i,j})_{i\in I, j\in I}}{\sum_{i\in I}a_{i,i}}.
  \end{align*}
  この写像は$\KK$-線形である.
\end{quiz}


\begin{quiz}
  次の命題の真偽を判定し証明を与えよ:
  $V$を$\KK$-線形空間とする.
  $a\in A$, $b\in V$とする.
  $\varphi$を次の写像とする:
  \begin{align*}
    \shazo{\varphi}{V}{V}{x}{ax+b}.
  \end{align*}
  このとき,
  $\varphi$が線形写像であることと$b$が$V$の零元であることは同値.
\end{quiz}

\begin{quiz}
  $\varphi$を次の写像とする:
  \begin{align*}
    \shazo{\varphi}{\CC}{\CC}{z}{\overline{z}},
  \end{align*}
  ただし, 
  $x,y\in\RR$に対し$\overline{x+y\sqrt{-1}}=x-y\sqrt{-1}$, つまり,
  $\overline{z}$は$z$の複素共軛とする.
  以下の命題の真偽を判定し証明を与えよ:
  \begin{enumerate}
  \item $\varphi$は$\RR$-線形である.
  \item $\varphi$は$\CC$-線形ではない.
  \end{enumerate}
\end{quiz}

\begin{quiz}
  次の命題の真偽を判定し証明を与えよ:
  $V$を$\KK$線形空間とし,
  $w_1,\ldots,w_r\in V$とする.
    \begin{align*}
      \shazo{\nu_{(w_1,\ldots,w_r)}}{\KK^r}{V}
      {\begin{pmatrix}a_1\\\vdots\\a_r\end{pmatrix}}{a_1 w_1+\cdots+ a_r w_r}
    \end{align*}
    は$\KK$-線形写像である.
\end{quiz}

\begin{quiz}
  次の命題の真偽を判定し証明を与えよ:
  $V$, $W$
  を$\KK$線形空間とし,
  $W$の零元を$0_W$とする.
    \begin{align*}
      \shazo{\underline{0_W}}{V}{W}
      {x}{0_W}
    \end{align*}
  は$\KK$-線形写像である.
\end{quiz}

\begin{quiz}
  次の命題の真偽を判定し証明を与えよ:
  $V$を$\KK$-線形空間とする.
  恒等写像$\id_V$は$\KK$-線形である.
\end{quiz}

\begin{quiz}
  次の命題の真偽を判定し証明を与えよ:
  $V$, $U$, $W$を$\KK$-線形空間とし,
  $\varphi\colon V\to U$,
  $\psi\colon U\to W$を$\KK$-線形写像とする.
  このとき, $\psi\circ\varphi\colon V\to W$は$\KK$-線形写像である.
\end{quiz}

\begin{quiz}
  次の命題の真偽を判定し証明を与えよ:
  $V$, $W$を$\KK$-線形空間とし,
  $\varphi\colon V \to W$を線形写像とする.
  $\varphi$が全単射なら,
  その逆写像
  $\varphi^{-1}$
  も線形写像である.
\end{quiz}

\begin{quiz}
  \label{quiz:hom:sc}
  次の命題の真偽を判定し証明を与えよ:
  $V$, $V$を$\KK$-線形空間とし,
  $\varphi\colon V\to W$を線形写像とする.
  このとき, $\alpha \in \KK$に対し,
  \begin{align*}
  \shazo{\alpha\varphi}{V}{W}
  {x}{\alpha (\varphi(x))}
  \end{align*}
  は$V$から$W$への線形写像.
\end{quiz}

\begin{quiz}
  \label{quiz:hom:sum}
  次の命題の真偽を判定し証明を与えよ:
  $V$, $W$を$\KK$-線形空間とし,
  $\varphi\colon V\to W$,
  $\psi\colon V\to W$
  を線形写像とする.
  このとき, 
  \begin{align*}
  \shazo{\varphi+\psi}{V}{W}
  {x}{\varphi(x) + \psi(x)}
  \end{align*}
  は$V$から$W$への線形写像.
\end{quiz}

\begin{quiz}
  $V$, $W$を$\KK$-線形空間とし,
  $0_V$, $0_W$をそれぞれの零元とする.
  $\varphi\colon V\to W$を$\KK$-線形写像とする.
  次の命題の真偽を判定し証明を与えよ:
  \begin{enumerate}
    \item $\varphi(0_V)=0_W$.
    \item $\varphi(-x)=-\varphi(x)$.
  \end{enumerate}
\end{quiz}

\begin{quiz}
  次の命題の真偽を判定し証明を与えよ:
  $A\in \KK^{m\times n}$に対し,
  $\mu_A$を次の写像とする:
  \begin{align*}
    \shazo{\mu_A}{\KK^n}{\KK^m}{w}{Aw}.
  \end{align*}
$H=\Set{\varphi\colon \KK^n\to\KK^m\text{; 線形写像}}$,
$M=\Set{\mu_A|A\in\KK^{m\times n}}$
  とするとき, $H=M$.
\end{quiz}

\begin{quiz}
  次の命題の真偽を判定し証明を与えよ:
  $\aaa,\bbb\in\RR^n$に対し,
  $\Braket{\aaa,\bbb}$を$\aaa$と$\bbb$の内積とする.
  $\aaa\in\RR^n$に対し
  \begin{align*}
    \shazo{\varphi_{\aaa}}{\RR^n}{\RR}
    {\xx}{\braket{\aaa,\xx}}
  \end{align*}
  は$\RR$-線形写像である.
\end{quiz}

\begin{quiz}
  $\aaa,\bbb\in\RR^n$に対し,
  $\Braket{\aaa,\bbb}$を$\aaa$と$\bbb$の内積とする.
  $\|\aaa\|^2=\Braket{\aaa,\aaa}=1$を満たす
  $\aaa\in\RR^n$に対し,
  \begin{align*}
    \shazo{\varphi_{\aaa}}{\RR^n}{\RR^n}
    {\xx}{\braket{\aaa,\xx}\aaa}
  \end{align*}
  とおく.
  以下の命題の真偽を判定し証明を与えよ:
  \begin{enumerate}
  \item $\varphi_{\aaa}$は$\RR$-線形写像である.
  \item $\varphi_{\aaa}\circ \varphi_{\aaa}=\varphi_{\aaa}$.
  \end{enumerate}
\end{quiz}
\begin{quiz}
  $\aaa,\bbb\in\RR^n$に対し,
  $\Braket{\aaa,\bbb}$を$\aaa$と$\bbb$の内積とする.
  $\|\aaa\|^2=\Braket{\aaa,\aaa}=1$を満たす
  $\aaa\in\RR^n$に対し,
  \begin{align*}
    \shazo{\varphi_{\aaa}}{\RR^n}{\RR^n}
    {\xx}{\xx-\braket{\aaa,\xx}\aaa}
  \end{align*}
  とおく.
  以下の命題の真偽を判定し証明を与えよ:
  \begin{enumerate}
  \item $\varphi_{\aaa}$は$\RR$-線形写像である.
  \item $\varphi_{\aaa}\circ \varphi_{\aaa}=\varphi_{\aaa}$.
  \end{enumerate}
\end{quiz}

\begin{quiz}
  $\aaa,\bbb\in\RR^n$に対し,
  $\Braket{\aaa,\bbb}$を$\aaa$と$\bbb$の内積とする.
  $\|\aaa\|^2=\Braket{\aaa,\aaa}=1$を満たす
  $\aaa\in\RR^n$に対し,
  \begin{align*}
    \shazo{\varphi_{\aaa}}{\RR^n}{\RR^n}
    {\xx}{\xx-2\braket{\aaa,\xx}\aaa}
  \end{align*}
  とおく.
  以下の命題の真偽を判定し証明を与えよ:
  \begin{enumerate}
  \item $\varphi_{\aaa}$は$\RR$-線形写像である.
  \item $\varphi_{\aaa}\circ \varphi_{\aaa}=\id_{\RR^n}$.
  \end{enumerate}
\end{quiz}

\begin{quiz}
\begin{align*}
  \aaa=\begin{pmatrix}a_1\\\vdots\\a_n\end{pmatrix}\in\RR^n
\end{align*}
に対し,
\begin{align*}
  \psi(\aaa)&=\frac{1}{n}\sum_{i=1}^n a_i\\
  \phi(\aaa)&=\frac{1}{n}\sum_{i=1}^n (a_i - \psi(\aaa))^2\\
  \varphi(\aaa)&=\sqrt{\phi(\aaa)}
\end{align*}
このとき,
以下の命題の真偽を判定し証明を与えよ:
\begin{enumerate}
  \item $\psi\colon \RR^n\to \RR$は$\RR$線形写像である.
  \item $\phi\colon \RR^n\to \RR$は$\RR$線形写像ではない.
  \item $\varphi\colon \RR^n\to \RR$は$\RR$線形写像ではない.
\end{enumerate}
\end{quiz}


\section{\Cref{sec:iso:def,sec:subspace:def,sec:subspace:example}の後の問題}
主に同型写像の定義, 部分空間の定義を知っていれば解けるであろう問題.
%\endinput
%\newpage
%\section{4/15回目: \schoolCalender{4}}
% 講義: 同型写像の定義と例. 部分空間の定義と例(1). 
%\subsection{レポート課題 (締め切り: \schoolCalender{5})}
\subsection{問題A}
\begin{quiz}
  %\solvelater{quiz:1:1}
  \begin{align*}
    V&=\Set{\begin{pmatrix}x\\x\\0\end{pmatrix}|x\in \KK}&
    V'&=\Set{\begin{pmatrix}0\\x\\x\end{pmatrix}|x\in \KK}
  \end{align*}
  とする.
  \begin{enumerate}
    \item $V\cap V'$を求めよ.
    \item $\Set{v+v'|v\in V,v'\in V}$は$\KK^3$の部分線形空間であることを示せ.
    \item $V\cup V'$は$\KK^3$の部分線形空間では無いことを示せ.
  \end{enumerate}
\end{quiz}



\subsection{問題B}

\begin{quiz}
  次の命題の真偽を判定し証明を与えよ:
  $[n]=\Set{1,\ldots,n}$とし,
  $\KK^{[n]}=\Set{f\colon [n]\to \KK}$とする.
  \begin{align*}
    \aaa=\begin{pmatrix}a_1\\\vdots\\a_n\end{pmatrix}\in
    \KK^{n}=\Set{\begin{pmatrix}x_1\\\vdots\\x_n\end{pmatrix}|x_i\in\KK}
  \end{align*}
  に対し,
  \begin{align*}
    \shazo{f}{[n]}{\KK}{i}{a_i}
  \end{align*}
  とする.  
  $\varphi$を次の写像とする:
  \begin{align*}
    \shazo{\varphi}{\KK^{n}}{\KK^{[n]}}{\aaa}{f_{\aaa}}.
  \end{align*}
  このとき, $\varphi$は同型写像である.
\end{quiz}

\begin{quiz}
  $[k]=\Set{1,\ldots,k}$とする.
  次の命題の真偽を判定し証明を与えよ:
  \begin{enumerate}
  \item
    $\KK^{m\times n}\simeq \KK^{[m]\times [n]}$.
  \item
    $\KK^\NN\simeq \ell(\KK)$.
  \end{enumerate}
\end{quiz}

\begin{quiz}
  $V=\Set{x\in\RR|x>0}$とする.
  $x,y\in V$に対し,
  $x \pplus y =x\cdot y$
  とし,
  $x\in V$, $a\in \RR$に対し,
  $a\aact x=x^a$とする.
  このとき,
  $V$は和$\pplus$とスカラー倍$\aact$で
  $\RR$線形空間であった.
  また$\RR$は通常の和と積で$\RR$線形空間であった.
  次の命題の真偽を判定し証明を与えよ:
  $\RR\simeq V$.
\end{quiz}

\begin{quiz}
  次の命題の真偽を判定し証明を与えよ:
  $A\in \KK^{m\times n}$に対し,
  $\mu_A$を次の写像とする:
  \begin{align*}
    \shazo{\mu_A}{\KK^n}{\KK^m}{w}{Aw}.
  \end{align*}
  このとき,
  \begin{align*}
    \shazo{\varphi}{\KK^{m\times n}}{\Hom_{\KK}(\KK^n,\KK^m)}
    {A}{\mu_A}
  \end{align*}
  は同型写像である.
\end{quiz}

\begin{quiz}
  次の命題の真偽を判定し証明を与えよ:
  $V$, $W$を$\KK$線形空間とし,
  $\varphi\colon V\to W$を同型写像とする.
  $c\in\KK$が逆数を持つとする.
  このとき,
  $c\varphi\colon V\to W$は同型写像.
\end{quiz}

\begin{quiz}
  $V$, $W$, $U$を$\KK$線形空間とする.
  次の命題の真偽を判定し証明を与えよ:
  \begin{enumerate}
  \item
    $\id_V\colon V\to V$は同型写像.
  \item
    $\varphi\colon V\to W$が同型写像なら
    $\varphi^{-1}\colon W\to V$も同型写像.
  \item
    $\varphi\colon V\to W$, $\psi\colon W \to U$がともに同型写像なら
    $\psi\circ\varphi\colon V\to U$も同型写像.
  \end{enumerate}
\end{quiz}



\begin{quiz}
  $\zeta = e^{\frac{2\pi\sqrt{-1}}{n}}=\cos(\frac{2\pi}{n})+\sin(\frac{2\pi}{n})\sqrt{-1}\in \CC$とする.
  $V=\Set{\sum_{i=0}^{n-1}a_i \zeta^i|a_i\in \QQ}$とする.
  このとき, $V$は通常の和と積で
  $\QQ$-線形空間であることを示せ.
\end{quiz}

\begin{quiz}
  次の命題の真偽を判定し証明を与えよ:
  \begin{enumerate}
  \item $\Set{\begin{pmatrix}x\\2x \end{pmatrix} | x\in\RR}$は,
    $\RR^2$の部分空間である.
  \item $\Set{\begin{pmatrix}x\\x+2 \end{pmatrix} | x\in\RR}$は,
    $\RR^2$の部分空間ではない.
  \item $\Set{\begin{pmatrix}x\\x^2 \end{pmatrix} | x\in\RR}$は,
    $\RR^2$の部分空間ではない.
  \end{enumerate}
\end{quiz}

\begin{quiz}
  次の命題の真偽を判定し証明を与えよ:
  \begin{align*}
    V=\Set{\begin{pmatrix}x_1\\x_2\end{pmatrix}\in\RR^2|x_1x_2>0}
  \end{align*}
  は$\RR^2$の部分空間ではない.
\end{quiz}

\begin{quiz}
  (\Cref{quiz:hom:sc,quiz:hom:sum}と本質的に同じ)
  次の命題の真偽を判定し証明を与えよ:
  $V$, $W$を$\KK$-線形空間とする.
  このとき, $\Hom_{\KK}(V,W)$は$W^V$の部分$\KK$-線形空間である.
\end{quiz}

\begin{quiz}
  次の命題の真偽を判定し証明を与えよ:
  $V$を漸化式$a_i=a_{i-1}+a_{i-2}$で定義される数列全体の集合, つまり,
  \begin{align*}
    V=\Set{(a_i)_{i\in \NN}\in \ell(\KK)|
      \begin{array}{c}
        a_0,a_1\in\KK\\
        i>1\implies a_{i}-a_{i-1}-a_{i-2}=0
      \end{array}
    }
  \end{align*}
  とする.
  $V$は$\ell(\RR)$の部分空間である.
\end{quiz}

\begin{quiz}
  次の命題の真偽を判定し証明を与えよ:
  \begin{align*}
    V=\Set{(a_i)_{i\in \NN}\in \ell(\RR)|\text{次を満たす$n$が存在する: }i>n\implies a_i=0}
  \end{align*}
  とする.
  $V$は$\ell(\RR)$の部分空間である.
\end{quiz}

\begin{quiz}
  $X\subset\RR$に対し,
  次の条件を満たす$M$を$X$の上界と呼ぶ: $x\in X\implies x<M$.
  $X$の上界が存在することを,
  $\sup(X)<\infty$と書く.
  また,
  次の条件を満たす$m$を$X$の下界と呼ぶ: $x\in X\implies m<x$.
  $X$の下界が存在することを,
  $\inf(X)>-\infty$と書く.
  \begin{align*}
    V=\Set{(a_i)_{i\in \NN}\in \ell(\RR)|
      \begin{array}{l}
        \sup\Set{a_i|i\in\NN}<\infty\\
        \inf\Set{a_i|i\in\NN}>-\infty
      \end{array}
    }
  \end{align*}
  とする.
  $V$は$\ell(\RR)$の部分空間であることを,
  $\sup(X)<\infty$, $\inf(X)>-\infty$の定義に基づき証明せよ.
\end{quiz}

\begin{quiz}
  $(a_i)_{i\in\NN}$が次の条件をみたすとき, $\lim_{i\to\infty}a_i =0$と書く:
  \begin{quote}
    $\varepsilon > 0$ ならば,
      次の条件を満たす$N\in \NN$が存在する:
      \begin{align*}
        n>N \implies -\varepsilon < a_n<\varepsilon.
      \end{align*}
  \end{quote}
  このとき,
  \begin{align*}
    V=\Set{(a_i)_{i\in \NN}\in \ell(\RR)|\lim_{i\to\infty}a_i = 0}
  \end{align*}
  とする.
  $V$は$\ell(\RR)$の部分空間であることを,
  $\lim_{i\to\infty}a_i =0$の定義に基づき証明せよ.
\end{quiz}

\begin{quiz}
  次の命題の真偽を判定し証明を与えよ:
  $U$を$\KK$-線形空間とする.
  $V\subset U$に対し,
  以下の2つは同値である:
  \begin{enumerate}
  \item 次の$2$条件を満たす:
    \begin{enumerate}
    \item $v,w\in V$, $a,b\in \KK\implies av+bw\in V$.
    \item $0_U\in V$.
    \end{enumerate}
  \item 次の$2$条件を満たす:
    \begin{enumerate}
    \item $v,w\in V$, $a,b\in \KK\implies av+bw\in V$.
    \item $V\neq \emptyset$.
    \end{enumerate}
  \end{enumerate}
\end{quiz}

\newcommand{\Mod}[1]{\mathrel{\underset{#1}{\equiv}}}
\begin{quiz}
  $U$を$\KK$-線形空間とし, $V$を$U$の部分空間とする.
  $u,u'\in U$に対し,
  $u-u'\in V$となることを$u\Mod{V} u'$と書くことにする.
  このとき,
  次の命題の真偽を判定し証明を与えよ:
  \begin{enumerate}
  \item $u\in U \implies u \Mod{V} u$.
  \item $u,u'\in U, u \Mod{V} u' \implies u' \Mod{V} u$.    
  \item $u,u',u''\in U, u \Mod{V} u', u' \Mod{V} u'' \implies u \Mod{V} u''$.
  \end{enumerate}
\end{quiz}

\begin{quiz}
  $U$を$\KK$-線形空間とし, $V$を$U$の部分空間とする.
  $u,u'\in U$に対し,
  $u-u'\in V$となることを$u\Mod{V} u'$と書くことにする.
  このとき,
  次の命題の真偽を判定し証明を与えよ:
  \begin{enumerate}
  \item $u,u', w,w'\in U, u \Mod{V} u', w \Mod{V} w' \implies u+w \Mod{V} u'+w'$.
  \item $c\in \KK,u,u'\in U, u \Mod{V} u'\implies cu \Mod{V} cu'$.
  \end{enumerate}
\end{quiz}

\section{\Cref{sec:subspace:mor,sec:subspace:sub,sec:subspace:non}の後の問題}
主に, 核, 像, 共通部分, 和空間などを知っていれば,
解けるであろう問題.

\subsection{問題A}
\begin{quiz}
  列ベクトル表示された
  $(m,n)$行列
  $A=(\aaa_1|\cdots|\aaa_n)$
  を考える.
  このとき, 以下を(連立方程式の解の自由度に帰着させることで)示せ:
  \begin{enumerate}
  \item
    $\rank(A)\geq m$ならば,
   次が成り立つ:
   \begin{enumerate}
   \item $\bbb\in\KK^m$とする.
     このとき, $\bbb=c_1\aaa_1+\cdots+c_n\aaa_n$を満たす
     $c_1,\ldots,c_n\in\KK$が存在する.
   \end{enumerate}
  \item
    $\rank(A)< n$ならば,
   次が成り立つ:
   \begin{enumerate}
   \item 
     $c_1\aaa_1+\cdots+c_n\aaa_n=\zzero$かつ
     $(c_1,\ldots,c_n)\neq (0,\ldots,0)$を満たす
     $c_1,\ldots,c_n\in\KK$が存在する.
   \end{enumerate}
  \end{enumerate}
\end{quiz}

\subsection{問題B}



\begin{quiz}
  次の命題の真偽を判定し証明を与えよ:
  $U$を$\KK$-線形空間とし,
  $V,V', W$を$U$の部分空間とする.
  このとき,
  $(V+V')\cap W=(V\cap W)+(V'\cap W)$となるとは限らない.
\end{quiz}

\begin{quiz}
  次の命題の真偽を判定し証明を与えよ:
  $U$を$\KK$-線形空間とし,
  $V,V', W$を$U$の部分空間とする.
  このとき,
  $(V\cap V')+ W=(V+W)\cap (V'+W)$となるとは限らない.
\end{quiz}


\begin{quiz}
  次の命題の真偽を判定し証明を与えよ:
  $U$を$\KK$-線形空間とし,
  $V,W$を$U$の部分空間とする.
  このとき,
  \begin{align*}
    \tilde V&=\Set{(v,-v)\in U\boxplus U|v\in V} \\
    \check W&=\Set{(w,0_U)\in U\boxplus U|w\in W} \\
    \hat U&=\Set{(0_U,u)\in U\boxplus U|u\in U}
  \end{align*}
  とおく.
  また,
  \begin{align*}
    \check X &= (\tilde V + \check W) \cap \hat U \\
    X&=\Set{x|(0_U,x) \in\check X}
  \end{align*}
  とする.
  このとき, $X=V\cap W$.
\end{quiz}

\begin{quiz}
  $V$, $W$を$\KK$-線形空間とし,
  $\varphi\colon V\to W$を$\KK$線形写像とする.
  このとき,
  以下の命題の真偽を判定し証明を与えよ:
  \begin{enumerate}
  \item
    $V$の部分空間$V'$に対し,
    $W'=\Set{\varphi(x)\in W|x\in V'}$は$W$の部分空間である.
  \item
    $W$の部分空間$W''$に対し,
    $V''=\Set{x\in V|\varphi(x)\in W''}$は$V$の部分空間である.
  \end{enumerate}
\end{quiz}

\begin{quiz}
  $\varphi\colon \RR[x]\to \RR[x]$を,
  $x^n \in \KK[x]$に対して$nx^{n-1}$を対応させる$\RR$-線形写像,
  つまり, 多項式$f\in\RR[x]$に対し$f$の微分を対応させる写像とする.
  このとき, $\Img(\varphi)$と$\Ker(\varphi)$を求めよ.
\end{quiz}

\begin{quiz}
  $V$を$\KK$-線形空間とし,
  $\varphi\colon V\to V$を$\KK$線形写像とする.
  $\varphi^{i}=\underbrace{\varphi\circ\cdots\circ\varphi}_{i}$とし,
  $\varphi^{0}=\id_V$とする.
  このとき,
  以下の命題の真偽を判定し証明を与えよ:
  \begin{enumerate}
    \item $i\in\NN$に対し, $\Ker(\varphi^i)\subset \Ker(\varphi^{i+1})$.
    \item $\Ker(\varphi^N) =\Ker(\varphi^{N+1})$かつ$i\geq N\implies \Ker(\varphi^i) =\Ker(\varphi^{i+1})$.
  \end{enumerate}
\end{quiz}

\begin{quiz}
  $V$を$\KK$-線形空間とし,
  $\varphi\colon V\to V$を$\KK$線形写像とする.
  $\varphi^{i}=\underbrace{\varphi\circ\cdots\circ\varphi}_{i}$とし,
  $\varphi^{0}=\id_V$とする.
  このとき,
  以下の命題の真偽を判定し証明を与えよ:
  \begin{enumerate}
    \item $i\in\NN$に対し, $\Img(\varphi^i)\supset \Img(\varphi^{i+1})$.
    \item $\Img(\varphi^N) =\Img(\varphi^{N+1})$かつ$i\geq N\implies \Img(\varphi^i) =\Img(\varphi^{i+1})$.
  \end{enumerate}
\end{quiz}


\begin{quiz}
  $V$, $W$を$\KK$-線形空間とし,
  $\varphi\colon V\to W$を$\KK$線形写像とする.
  \begin{align*}
    F&=\Set{(v,\varphi(v))\in V\boxplus W| v\in V }\\
    K&=\Set{(x,y) \in F | y=0_W}
  \end{align*}
  とおく.
  また
  \begin{align*}
    \shazo{\pi_1}{V\boxplus W}{V}
          {(v,w)}{v}
          &&
    \shazo{\pi_2}{V\boxplus W}{W}
          {(v,w)}{w}
  \end{align*}
  とする.
  このとき,
  以下の命題の真偽を判定し証明を与えよ:
  \begin{enumerate}
  \item
    $F$は$V\boxplus W$の部分空間である.
  \item
    $\Ker(\varphi)=\Set{\pi_1(a) | a\in K}$.
  \item
    $\Img(\varphi)=\Set{\pi_2(a) | a\in F}$.
  \end{enumerate}
\end{quiz}

\begin{quiz}
  $V$, $W$, $V'$, $W'$を$\KK$-線形空間とする.
  $\varphi\colon V\to V'$,
  $\phi\colon W\to W'$を同型写像とし,
  $\psi\colon V\to W$を$\KK$-線形写像とする.
  $\psi'=\phi\circ\psi\circ\varphi^{-1}$とおく.
  このとき,
  以下の命題の真偽を判定し証明を与えよ:
  \begin{enumerate}
  \item $\Ker(\psi')\simeq\Ker(\psi)$.
  \item $\Img(\psi')\simeq\Img(\psi)$.
  \end{enumerate}
\end{quiz}

\begin{quiz}
  \begin{align*}
    \aaa=\begin{pmatrix}a_1\\\vdots\\a_n\end{pmatrix},\quad
    \bbb=\begin{pmatrix}b_1\\\vdots\\b_n\end{pmatrix}\in\RR^n
  \end{align*}
  に対し,
  \begin{align*}
    B(\aaa,\bbb)=\sum_{i=1}^n a_ib_i
  \end{align*}
  とする.
  つまり, $B(\aaa,\bbb)$を$\aaa$と$\bbb$の内積とする.
  $V$を$\RR^n$の部分空間とし,
  \begin{align*}
    W=\Set{\aaa\in \RR^n|\bbb\in V\implies B(\aaa,\bbb)=0}
  \end{align*}
  とおく.
  このとき,
  以下の命題の真偽を判定し証明を与えよ:
  \begin{enumerate}
  \item
    $W$は$\RR^n$の部分空間である.
  \item
    $\RR^n=V\oplus W$と内部直和に分解できる.
  \end{enumerate}
\end{quiz}

\begin{quiz}  
  \begin{align*}
    \aaa=\begin{pmatrix}a_1\\a_2\end{pmatrix},\quad
    \bbb=\begin{pmatrix}b_1\\b_2\end{pmatrix}\in\CC^2
  \end{align*}
  に対し,
  \begin{align*}
    B(\aaa,\bbb)= a_1 \overline{b_1}+ a_2 \overline{b_2}
  \end{align*}
  とする.
  ただし, $\overline{x}$は$x$の複素共軛を表す.
  $V$を$\CC^2$の部分空間とし,
  \begin{align*}
    W=\Set{\aaa\in \CC^2|\bbb\in V\implies B(\aaa,\bbb)=0}
  \end{align*}
  とおく.
  このとき,
  以下の命題の真偽を判定し証明を与えよ:
  \begin{enumerate}
  \item
    $W$は$\CC^2$の部分空間である.
  \item
    $\CC^2=V\oplus W$と内部直和に分解できる.    
  \end{enumerate}
\end{quiz}

\begin{quiz}  
  \begin{align*}
    \aaa=\begin{pmatrix}a_1\\a_2\end{pmatrix},\quad
    \bbb=\begin{pmatrix}b_1\\b_2\end{pmatrix}\in\CC^2
  \end{align*}
  に対し,
  \begin{align*}
    B(\aaa,\bbb)= a_1 b_1+ a_2 b_2
  \end{align*}
  とする.
  $V$を$\CC^2$の部分空間とし,
  \begin{align*}
    W=\Set{\aaa\in \CC^2|\bbb\in V\implies B(\aaa,\bbb)=0}
  \end{align*}
  とおく.
  このとき,
  以下の命題の真偽を判定し証明を与えよ:
  \begin{enumerate}
  \item
    $W$は$\CC^2$の部分空間である.
  \item
    $\CC^2=V\oplus W$と内部直和に分解できないこともある.    
  \end{enumerate}
\end{quiz}

\begin{quiz}
  $\aaa,\bbb\in\RR^n$に対し,
  $\Braket{\aaa,\bbb}$を$\aaa$と$\bbb$の内積とする.
  $\|\aaa\|^2=\Braket{\aaa,\aaa}=1$を満たす
  $\aaa\in\RR^n$に対し,
  \begin{align*}
    \shazo{\varphi_{\aaa}}{\RR^n}{\RR^n}
    {\xx}{\xx-\braket{\aaa,\xx}\aaa}
  \end{align*}
  とおく.
  このとき,
  以下の命題の真偽を判定し証明を与えよ:
  $\Ker(\varphi_{\aaa})=\Set{c\aaa|c\in \RR}$.
\end{quiz}


\begin{quiz}
  次の命題の真偽を判定し証明を与えよ:
  $V$を$\KK$-線形空間とする.
  $\varphi\colon V\to V$は次を満たす$\KK$-線形写像であるとする:
  $\varphi\circ\varphi = \varphi$.
  このとき,
  $V=\Img(\varphi)\oplus \Ker(\varphi)$と内部直和分解する.
\end{quiz}

\begin{quiz}
  次の命題の真偽を判定し証明を与えよ:
  $U$を$\KK$-線形空間とし,
  $I$を集合とする.
  $i\in I$に対し, $V_i$を$U$の部分空間とする.
  このとき,
  $\bigcap_{i\in I}V_i$は$U$の部分空間である.
\end{quiz}


\begin{quiz}
  $U$を$\KK$-線形空間とする.
  $i\in \NN$に対し,
  $V_i$を$U$の部分空間とし,
  $V_{i}\subset V_{i+1}$
  を満たしているとする.
  $V=\bigcup_{i\in \NN}V_i$
  とおく.
  このとき,
  $V$は$U$の部分空間である.
\end{quiz}

\begin{quiz}
  $U$を$\KK$-線形空間とし,
  $v_1,\ldots, v_r \in U$とする.
  \begin{align*}
    V=\Set{a_1v_1+\cdots +a_rv_r|a_i\in \KK}    
  \end{align*}
  とおく. また,
  \begin{align*}
    \SSS=\Set{W|v_1,\ldots,v_r \in W, \text{$W$は$U$の部分空間}}    
  \end{align*}
  とし,
  \begin{align*}
    W_0=\bigcap_{W\in\SSS}W
  \end{align*}
  とおく.
  このとき,
  次の命題の真偽を判定し証明を与えよ:
  \begin{enumerate}
  \item $V=W_0$.
  \item $W\in \SSS \implies V\subset W$.
  \end{enumerate}
\end{quiz}


\begin{quiz}
  次の命題の真偽を判定し証明を与えよ:
  $U$を$\KK$-線形空間とし,
  $V_1, V_2$を$U$の部分空間とする.
  \begin{align*}
    V=\Set{v_1+v_2|v_1\in V_1,v_2\in V_2}    
  \end{align*}
  とおく.
  また,
  \begin{align*}
    \SSS=\Set{W|V_1\cup V_2 \subset W, \text{$W$は$U$の部分空間}}    
  \end{align*}
  とし,
  \begin{align*}
    W_0=\bigcap_{W\in\SSS}W
  \end{align*}
  とおく.
  このとき,
  次の命題の真偽を判定し証明を与えよ:
  \begin{enumerate}
  \item $V=W_0$.
  \item $W\in \SSS \implies V\subset W$.
  \end{enumerate}
\end{quiz}


\begin{quiz}
  $U$を$\KK$-線形空間とし,
  $i\in\NN$に対し,
  $v_i \in U$とする.
  \begin{align*}
    V_r=\Set{a_1v_1+\cdots +a_rv_r|a_i\in \KK}    
  \end{align*}
  とし,
  \begin{align*}
    V=\bigcup_{r\in\NN}V_r
  \end{align*}
  とおく. また,
  \begin{align*}
    \SSS=\Set{W|\Set{v_i|i\in \NN}\subset W, \text{$W$は$U$の部分空間}}    
  \end{align*}
  とし,
  \begin{align*}
    W_0=\bigcap_{W\in\SSS}W
  \end{align*}
  とおく.
  このとき,
  次の命題の真偽を判定し証明を与えよ:
  \begin{enumerate}
  \item $V=W_0$.
  \item $W\in \SSS \implies V\subset W $.
  \end{enumerate}
\end{quiz}

\begin{quiz}
  次の命題の真偽を判定し証明を与えよ:
  $U$を$\KK$-線形空間とし,
  $I$を集合とする.
  $i\in \NN$に対し,
  $V_i$を$U$の部分空間とする.
  \begin{align*}
    V=\bigcup_{i\in \NN}(V_0+V_1+\cdots+V_i)
  \end{align*}
  とおく.
  また,
  \begin{align*}
    \SSS=\Set{W|\bigcup_{i\in\NN}V_i \subset W, \text{$W$は$U$の部分空間}}    
  \end{align*}
  とし,
  \begin{align*}
    W_0=\bigcap_{W\in\SSS}W
  \end{align*}
  とおく.
  このとき,
  次の命題の真偽を判定し証明を与えよ:
  \begin{enumerate}
  \item $V=W_0$.
  \item $W\in \SSS \implies V\subset W$.
  \end{enumerate}
\end{quiz}


\begin{quiz}
  $V$と$W$を$\KK$-線形空間とし,
  $0_V$を$V$の零元,
  $0_W$を$W$の零元とする.
  $V\boxplus W=\Set{(v,w)|v\in V,w\in W}$とし,
  \begin{align*}
    \shazo{\pi}{V\boxplus W}{V}
    {(v,w)}{v}&&
    \shazo{\iota}{V}{V\boxplus W}
    {v}{(v,0_W)}\\
    \shazo{\varpi}{V\boxplus W}{W}
    {(v,w)}{w}&&
    \shazo{\nu}{W}{V\boxplus W}
    {w}{(0_V,w)}
  \end{align*}
  とする.
  以下の命題の真偽を判定し証明を与えよ:
  \begin{enumerate}
  \item
    $U$を$\KK$-線形空間とし,
    $\varphi\colon U\to V$,
    $\phi\colon U\to W$
    を$\KK$-線形写像とする.
    このとき,
    $\psi\colon U\to V\boxplus W$で以下の条件を満たすものが,
    ただ一つ存在する:
    \begin{align*}
      \varphi&=\pi\circ\psi\\
      \phi&=\varpi\circ\psi.
    \end{align*}
  \item
    $U$を$\KK$-線形空間とし,
    $\varphi\colon V\to U$,
    $\phi\colon W\to U$
    を$\KK$-線形写像とする.
    このとき,
    $\psi\colon V\boxplus W\to U$で以下の条件を満たすものが,
    ただ一つ存在する:
    \begin{align*}
      \varphi&=\psi\circ\iota\\
      \phi&=\psi\circ\nu.
    \end{align*}
  \end{enumerate}
\end{quiz}


\begin{quiz}
  次の命題の真偽を判定し証明を与えよ:
  $I$を集合とし, $i\in I$に対し$V_i$を$0_{i}$を零元とする$\KK$-線形空間とする.
  $i \in I$に対し$v_i\in V_i$とする列を$(v_i)_{i\in I}$で表し,
  \begin{align*}
    \prod_{i\in I} V_i &=\Set{(v_i)_{i\in I}|v_i\in V_i}\\
    \bigoplus_{i\in I} V_i &=\Set{(v_i)_{i\in I} \in \prod_{i\in I} V_i|\#\Set{i\in I|v_i\neq 0_{i}}<\infty}
  \end{align*}
  とおく.
  $\prod_{i\in I} V_i$は次の和とスカラー倍で$\KK$-線形空間である:
  $(v_i)_{i\in I}+(w_i)_{i\in I}=(v_i+w_i)_{i\in I}$,
  $c(v_i)_{i\in I}=(cv_i)_{i\in I}$.
  このとき,
  $\bigoplus_{i\in I} V_i$は$\prod_{i\in I} V_i$の部分空間である.
\end{quiz}


\begin{quiz}
  次の命題の真偽を判定し証明を与えよ:
  $I$を集合とし, $i\in I$に対し$V_i$を$0_{i}$を零元とする$\KK$-線形空間とする.
  $i \in I$に対し$v_i\in V_i$とする列を$(v_i)_{i\in I}$で表し,
  \begin{align*}
    \prod_{i\in I} V_i &=\Set{(v_i)_{i\in I}|v_i\in V_i}
%    \bigoplus_{i\in I} V_i &=\Set{(v_i)_{i\in I} \in \prod_{i\in I} V_i|\#\Set{i\in I|v_i\neq 0_{i}}<\infty}
  \end{align*}
  とおく.
  \begin{align*}
    \shazo{\pi_i}{\prod_{i\in I} V_i}{V_i}
    {(v_i)_{i\in I}}{v_i}
  \end{align*}

  次の命題の真偽を判定し証明を与えよ:
  \begin{enumerate}
  \item
    $U$を$\KK$-線形空間とし,
    $i\in I$に対し
    $\varphi_i\colon U\to V_i$
    を$\KK$-線形写像とする.
    このとき,
    $\psi\colon U\to \prod_{i\in I} V_i$で以下の条件を満たすものが,
    ただ一つ存在する:
    \begin{align*}
      i\in I\implies \varphi&=\pi_i\circ\psi
    \end{align*}
  \end{enumerate}
\end{quiz}


\begin{quiz}
  次の命題の真偽を判定し証明を与えよ:
  $I$を集合とし, $i\in I$に対し$V_i$を$0_{i}$を零元とする$\KK$-線形空間とする.
  $i \in I$に対し$v_i\in V_i$とする列を$(v_i)_{i\in I}$で表し,
  \begin{align*}
    \prod_{i\in I} V_i &=\Set{(v_i)_{i\in I}|v_i\in V_i}\\
    \bigoplus_{i\in I} V_i &=\Set{(v_i)_{i\in I} \in \prod_{i\in I} V_i|\#\Set{i\in I|v_i\neq 0_{i}}<\infty}
  \end{align*}
  とおく.
  $i\in I$,
  $v\in V_i$に対し,
  \begin{align*}
    \delta_{i,j}v=
    \begin{cases}
      v & (j=i)\\
      0_j& (j\neq i)
    \end{cases}
  \end{align*}
  と書くことにする.
  \begin{align*}
    \shazo{\nu_i}{V_i}{\bigoplus_{i\in I} V_i}
    {v}{(\delta_{i,j}v_i)_{j\in I}}
  \end{align*}
  とするとき,
  次の命題の真偽を判定し証明を与えよ:
  \begin{enumerate}
  \item
    $U$を$\KK$-線形空間とし,
    $i\in I$に対し
    $\varphi_i\colon  V_i \to U$
    を$\KK$-線形写像とする.
    このとき,
    $\psi\colon \bigoplus_{i\in I} V_i\to U$で以下の条件を満たすものが,
    ただ一つ存在する:
    \begin{align*}
      i\in I\implies \varphi_i&=\psi\circ\nu_i
    \end{align*}
  \end{enumerate}
\end{quiz}

\begin{quiz}
  次の命題の真偽を判定し証明を与えよ:
  $U$を$\KK$-線形空間とする.
  $i\in \NN$に対し,
  $V_i$を$U$の部分空間とし,
  $V_{i}\subset V_{i+1}$
  を満たしているとする.
  \begin{align*}
    \shazo{\nu_i}{V_i}{\bigcup_{i\in \NN}V_i}
    {v}{v}
  \end{align*}
  とするとき,
  次の命題の真偽を判定し証明を与えよ:
  \begin{enumerate}
  \item
    $U$を$\KK$-線形空間とし,
    $i\in I$に対し
    $\varphi_i\colon  V_i \to U$
    を$\KK$-線形写像とする.
    このとき,
    $\psi\colon \bigcup_{i\in \NN} V_i\to U$で以下の条件を満たすものが,
    ただ一つ存在する:
    \begin{align*}
      i\in \NN\implies \varphi_i&=\psi\circ\nu_i
    \end{align*}
  \end{enumerate}
\end{quiz}



%% \begin{quiz}
%% \begin{align*}
%%   \aaa=\begin{pmatrix}a_1\\\vdots\\a_n\end{pmatrix}\in\RR^n
%% \end{align*}
%% に対し,
%% \begin{align*}
%%   \psi(\aaa)&=\frac{1}{n}\sum_{i=1}^n a_i\\
%%   \phi(\aaa)&=\frac{1}{n}\sum_{i=1}^n (a_i - \psi(\aaa))^2\\
%%   \varphi(\aaa)&=\sqrt{\phi(\aaa)}
%% \end{align*}
%% このとき,
%% 以下の命題の真偽を判定し証明を与えよ:
%% \begin{enumerate}
%%   \item $\psi\colon \RR^n\to \RR$は$\RR$線形写像である.
%%   \item $\phi\colon \RR^n\to \RR$は$\RR$線形写像ではない.
%%   \item $\varphi\colon \RR^n\to \RR$は$\RR$線形写像ではない.
%% \end{enumerate}
%% \end{quiz}

%% \begin{quiz}
%%   次の命題の真偽を判定し証明を与えよ:
%%   \begin{align*}
%%     V=\Set{(a_i)_{i\in \NN}\in \ell(\RR)|\text{次を満たす$n$が存在する: }i>n\implies a_i=0}
%%   \end{align*}
%%   とする.
%%   $V$は$\ell(\RR)$の部分空間である.
%% \end{quiz}

\section{\Cref{sec:basis:standard,sec:basis:def,sec:basis:linmap}の後の問題}
主に基底の定義, 生成系の定義, 一次独立性の定義を知っていれば
解けるであろう問題.
\subsection{問題A}
\begin{quiz}
  $V=\KK^2$, $W=\KK^3$, $A$, $B$を$(3,2)$行列とする.
  \begin{align*}
    \shazo{\varphi}{V}{W}
    {x}{Ax} \\
    \shazo{\psi}{V}{W}
    {x}{Bx}
 \end{align*}
  とする.
  また$(\ee_1,\ee_2)$を$V$の標準基底とする.
  このとき, 以下は同値であることを示せ:
  \begin{enumerate}
  \item $\varphi=\psi$.
  \item $A=B$.
  \item $\varphi(\ee_1)=\psi(\ee_1)$, $\varphi(\ee_2)=\psi(\ee_2)$.
  \end{enumerate}
\end{quiz}

\begin{quiz}
  $U$を$\KK$線形空間であるとする.
  $U=V\oplus V'$と内部直和に分解されているとする.
  $(e_1,\ldots,e_n)$を$V$の基底とし,
  $(e'_1,\ldots,e'_m)$を$V'$の基底とする.
  このとき,
  $(e_1,\ldots,e_n,e'_1,\ldots,e'_m)$
  は$U$の基底であることを示せ.  
\end{quiz}
\begin{quiz}
  $(V,0_V)$, $(W,0_W)$を$\KK$線形空間であるとする.
  $(e_1,\ldots,e_n)$を$V$の基底とし,
  $(e'_1,\ldots,e'_m)$を$W$の基底とする.
  このとき,
  $((e_1,0_W),\ldots,(e_n,0_W),(0_V,e'_1),\ldots,(0_V,e'_m))$
  は外部直和$V\boxplus W$の基底であることを示せ.  
\end{quiz}

\subsection{問題B}
\begin{quiz}
  \begin{align*}
    A=
    \begin{pmatrix}
      1&2&3&4&5\\
      1&2&4&8&16\\
      1&2&7&12&21
    \end{pmatrix}
  \end{align*}
  とし,
  \begin{align*}
    \shazo{\mu_A}{\KK^5}{\KK^3}
    {x}{Ax}
  \end{align*}
  を考える.
  このとき,
  $\Ker(\mu_A)$
  および
  $\Img(\mu_A)$
  の基底を与えよ.
\end{quiz}

\begin{quiz}
  \begin{align*}
    \shazo{\varphi}{\KK^{n\times n}}{\KK}
    {A}{\tr(A)}
  \end{align*}
  を考える.
  このとき,
  $\Ker(\varphi)$
  および
  $\Img(\varphi)$
  の基底を与えよ.
\end{quiz}

\begin{quiz}
  $\KK$を体とし,
  \begin{align*}
    V=
    \Set{A\in\KK^{n\times n} | B\in \KK^{n\times n} \implies AB=BA}
  \end{align*}
  とする.
  \begin{enumerate}
  \item
    次の命題の真偽を判定し証明を与えよ:
    $V$は$\KK^{n\times n}$の部分空間である.
  \item
    $V$の基底を与えよ.
  \end{enumerate}
\end{quiz}

\begin{quiz}
  次の命題の真偽を判定し証明を与えよ:
  $\RR$は$\QQ$上の線形空間であった.
  $1,x\in\RR$に対し, 以下は同値である:
  \begin{enumerate}
  \item $(1,x)$は$\QQ$上一次独立である.
  \item $x$は無理数である.
  \end{enumerate}
\end{quiz}


\begin{quiz}
  次の命題の真偽を判定し証明を与えよ:
  $V,W$を$\KK$-線形空間とし,
  $v_1,\ldots,v_n\in V$とする.
  $\varphi\colon V\to W$を$\KK$線形写像する.
  このとき,
  $(\varphi(v_1),\ldots,\varphi(v_n))$が一次独立ならば
  $(v_1,\ldots,v_n)$も一次独立.
\end{quiz}

\begin{quiz}
  $V,W$を$\KK$-線形空間とし,
  $v_1,\ldots,v_n\in V$とする.
  $\varphi\colon V\to W$を$\KK$線形写像する.
  このとき,
  以下の命題の真偽を判定し証明を与えよ:
  \begin{enumerate}
  \item    
    $(v_1,\ldots,v_n)$が一次独立かつ$\varphi$が単射ならば
    $(\varphi(v_1),\ldots,\varphi(v_n))$も一次独立.
  \item
    $(v_1,\ldots,v_n)$が$V$の生成系かつ$\varphi$が全射ならば
    $(\varphi(v_1),\ldots,\varphi(v_n))$
    は$W$の生成系.
%%   \item
%%     $(v_1,\ldots,v_n)$が$V$の基底かつ$\varphi$が同型写像ならば
%%     $(\varphi(v_1),\ldots,\varphi(v_n))$
%%     は$W$の基底.
  \end{enumerate}
\end{quiz}

\begin{quiz}
  次の命題の真偽を判定し証明を与えよ:
  $V,W$を$\KK$-線形空間とし,
  $(v_1,\ldots,v_n)$を$V$の基底とする.
  $\varphi\colon V\to W$を$\KK$線形写像する.
  このとき, 以下は同値:
  \begin{enumerate}
  \item
    $\varphi$は同型写像である.
  \item
    $(\varphi(v_1),\ldots,\varphi(v_n))$
    は$W$の基底.
  \end{enumerate}
\end{quiz}


\begin{quiz}
$V$を$\KK$線形空間とし, $v_1,\ldots,v_n\in V$とする.
  このとき,
  以下の命題の真偽を判定し証明を与えよ:
  \begin{enumerate}
  \item
    $(v_1,\ldots,v_n)$が一次独立で$i\neq j$ならば$(v_i,v_j)$は一次独立.
  \item
    どの$i,j$に対しても$(v_i,v_J)$が一次独立であったとしても,
    $(v_1,\ldots,v_n)$が一次独立であるとは限らない.
  \end{enumerate}
\end{quiz}


\begin{quiz}
  次の命題の真偽を判定し証明を与えよ:
  $V$を$\KK$線形空間とし, $n>0$とする.
  以下は同値である:
  \begin{enumerate}
  \item $\dim_\KK(V)=n$
  \item $(e_1,\ldots,e_n)$が$V$の基底となるような$e_1,\ldots,e_n\in V$が取れる.
  \end{enumerate}
\end{quiz}

\begin{quiz}
  次の命題の真偽を判定し証明を与えよ:
  $V$を$\KK$線形空間とする.
  $\sigma\in S_n$とし,
  $B=(v_1,\ldots,v_n)$,
  $B'=(v_{\sigma(1)},\ldots,v_{\sigma(n)})$
  とする.
  このとき, 以下は同値:
  \begin{enumerate}
  \item
    $B$は$\KK$上一次独立.
  \item
    $B'$は$\KK$上一次独立.
  \end{enumerate}
\end{quiz}
\begin{quiz}
  $V$を$\KK$線形空間とする.
  $v_i\in V$とし,
  $(v_1,\ldots, v_r)$は$\KK$上一次独立であるとする.
  このとき,
  以下の命題の真偽を判定し証明を与えよ:
  \begin{enumerate}
  \item
    $c\in\KK$が逆数を持つならば,
    $(cv_1,\ldots, v_r)$は$\KK$上一次独立である.
  \item
    $c\in\KK$に対し,
    $(v_1+cv_2,\ldots, v_r)$は$\KK$上一次独立である.
  \end{enumerate}
\end{quiz}


\begin{quiz}
  次の命題の真偽を判定し証明を与えよ:
  $V$を$\KK$線形空間とする.
  $\sigma\in S_n$とし,
  $B=(v_1,\ldots,v_n)$,
  $B'=(v_{\sigma(1)},\ldots,v_{\sigma(n)})$
  とする.
  このとき, 以下は同値:
  \begin{enumerate}
  \item
    $B$は$V$の生成系である.
  \item
    $B'$は$V$の生成系である.
  \end{enumerate}
\end{quiz}

\begin{quiz}
  $V$を$\KK$線形空間とする.
  $v_i\in V$とし,
  $(v_1,\ldots, v_r)$は$V$の生成系であるとする.
  このとき,
  以下の命題の真偽を判定し証明を与えよ:
  \begin{enumerate}
  \item
    $c\in\KK$が逆数を持つならば,
    $(cv_1,\ldots, v_r)$は$V$の生成系である.
  \item
    $c\in\KK$に対し,
    $(v_1+cv_2,\ldots, v_r)$は$V$の生成系である.
  \end{enumerate}
\end{quiz}


\begin{quiz}
  次の命題の真偽を判定し証明を与えよ:
  $V$を$\KK$線形空間とする.
  $v_i\in V$とし,
  $(v_1,\ldots, v_r)$が$\KK$上一次独立であるとする.
  このとき,
  $(v_{1},\ldots, v_{k})$は$\KK$上一次独立.
\end{quiz}



\begin{quiz}
  次の命題の真偽を判定し証明を与えよ:
  $V$を$\KK$線形空間とする.
  $v_i\in V$とし,
  $(v_1,\ldots, v_r)$が$V$の生成系であるとする.
  このとき,
  $w_1,\ldots,w_l\in V$に対し,
  $(v_{1},\ldots, v_{r},w_1,\ldots,w_l)$は$V$の生成系.
\end{quiz}

\begin{quiz}
  次の命題の真偽を判定し証明を与えよ:
  $U$を$\KK$線形空間とし,
  $v_1,\ldots,v_n,w_1,\ldots,w_m \in U$とする.
  このとき, 以下は同値:  
  \begin{enumerate}
  \item $\Set{v_1,\ldots,v_n}\subset \Braket{w_1,\ldots,w_m}_\KK$.
  \item $\Braket{v_1,\ldots,v_n}_\KK\subset \Braket{w_1,\ldots,w_m}_\KK$.
  \end{enumerate}
\end{quiz}



\begin{quiz}
  次の命題の真偽を判定し証明を与えよ:
  $\KK$を体とし,
  $V$を$\KK$ベクトル空間とする.
  $v_1,\ldots, v_r\in V$が次の条件を満たすとする:
  \begin{enumerate}
  \item $(v_1,\ldots, v_r)$は$\KK$上一次独立.
  \item $v\in V\implies (v_1,\ldots, v_r,v)$は$\KK$上一次従属.
  \end{enumerate}
  このとき,
  $(v_1,\ldots, v_r)$は$V$の基底.
\end{quiz}


\begin{quiz}
  次の命題の真偽を判定し証明を与えよ:
  $\KK$を体とし,
  $V$を$\KK$ベクトル空間とする.
  $v_1,\ldots, v_r\in V$が次の条件を満たすとする:
  \begin{enumerate}
  \item $(v_1,\ldots, v_r)$は$V$の生成系.
  \item
    $i\in \Set{1,\ldots,r}\implies (v_1,\ldots, v_{i-1},v_{i+1},\ldots,v_r)$
    は$V$の生成系ではない.
  \end{enumerate}
  このとき,
  $(v_1,\ldots, v_r)$は$V$の基底.
\end{quiz}

\begin{quiz}
  次の命題の真偽を判定し証明を与えよ:
$\KK$を体とし,
  $V$を$\KK$ベクトル空間とする.
  $(v_1,\ldots, v_r)$は$\KK$上一次独立であるとする.
  $V\neq\Braket{v_1,\ldots, v_r}_\KK\neq \emptyset$とする.
  このとき, $w\in V\setminus\Braket{v_1,\ldots, v_r}_\KK$に対し,
  $(v_1,\ldots,v_{r},w)$は$\KK$上一次独立.
\end{quiz}

\begin{quiz}
  次の命題の真偽を判定し証明を与えよ:
$\KK$を体とし,
  $V$を$\KK$ベクトル空間とする.
  $(v_1,\ldots, v_r)$,
  $(w_1,\ldots, w_n)$はどちらも,
  $\KK$上一次独立であるとする.
$r<n$とする.
  このとき,
$(v_1,\ldots,v_{r},w_{i})$が一次独立となる$i$が存在する.
\end{quiz}



%
\section{\cref{sec:basis:ext,sec:basis:linmap}の後の問題}
% 講義: 双対基底やHomの基底.  基底の延長.
主に双対基底などについて知っていれば
解けるであろう問題.
\subsection{問題A}
\begin{quiz}
  $A\in\KK^{m\times n}$とし,
  \begin{align*}
    \shazo{\mu_A}{\KK^n}{\KK^m}
    {x}{Ax}
  \end{align*}
  について考える.
  $A$の階数を$r$とするとき以下を求めよ:
  \begin{enumerate}
  \item $\dim_\KK(\Ker(\mu_A))$を求めよ.
  \item $\dim_\KK(\Img(\mu_A))$を求めよ.
  \item $\dim_\KK(\Ker(\mu_A))+\dim_\KK(\Img(\mu_A))$を与えよ.
  \end{enumerate}
  ただし,
  連立一次方程式の解の自由度に関する公式は用いてよい.
  また,
  被役(行)階段行列の行ベクトルから零ベクトルであるものを除いたものが,
  一次独立であることや,
  被役(列)階段行列の列ベクトルから零ベクトルであるものを除いたものが,
  一次独立であること
  も用いて良い.
\end{quiz}



\subsection{問題B}
\begin{quiz}
  $\KK^n$の標準基底$E=(\ee^{(n)}_1,\ldots,e^{(n)}_n)$に対し,
  その双対基底$E^\ast=(\varepsilon_1^{E},\ldots,\varepsilon_m^{E})$
  を考える.
  このとき, 次を求めよ:
  \begin{enumerate}
    \item $\ee_j$に対し, $\varepsilon_i^{E}(\ee_j)$.
    \item $\aaa=\begin{pmatrix}a_1\\\vdots\\a_n\end{pmatrix}$に対し$\varepsilon_i^{E}(\aaa)$:
  \end{enumerate}
\end{quiz}

\begin{quiz}
  $n$を$2$以上の整数とし,
  $E=(\ee^{(n)}_1,\ldots,e^{(n)}_n)$を
  $\KK^n$の標準基底の標準基底とする.
  \begin{align*}
    f_i =
    \begin{cases}
      \ee^{(n)}_i+\ee^{(n)}_{i+1} &(i<n)\\
      \ee^{(n)}_1+\ee^{(n)}_n&(i=n)
    \end{cases}
  \end{align*}
  とし,
  $F=(f_1,\ldots,F_n)$とする.
  このとき,
  以下の命題の真偽を判定し証明を与えよ:  
  \begin{enumerate}
    \item $n$が奇数のとき, $F$は$\KK^n$の基底ではない.
    \item $n$が偶数のとき, $F$は$\KK^n$の基底ではない.
  \end{enumerate}
\end{quiz}

\begin{quiz}
  $E=(\ee^{(3)}_1,\ldots,e^{(3)}_3)$を
  $\KK^3$の標準基底の標準基底とする.
  \begin{align*}
    f_1 &=\ee^{(3)}_1+\ee^{(3)}_2\\
    f_2 &=\ee^{(3)}_2+\ee^{(3)}_3\\
    f_3 &=\ee^{(3)}_1+\ee^{(3)}_3
  \end{align*}
  とし, $F=(f_1,\ldots,F_3)$とする.
  このとき,
   \begin{enumerate}
    \item $F$の双対基底$F^\ast=(\varepsilon_1^{F},\ldots,\varepsilon_3^{F})$
      に対し, $\varepsilon_i^{F}(\ee_j)$を求めよ.
    \item $F$の双対基底$F^\ast=(\varepsilon_1^{F},\ldots,\varepsilon_3^{F})$
      と$\aaa=\begin{pmatrix}a_1\\\vdots\\a_n\end{pmatrix}$に対し,
      $\varepsilon_i^{F}(\aaa)$を求めよ.
   \end{enumerate}
\end{quiz}

\begin{quiz}
  $E=(\ee^{(n)}_1,\ldots,e^{(n)}_n)$を
  $\KK^n$の標準基底の標準基底とする.
  $f_i=\ee^{(n)}_1+\cdots+\ee^{(n)}_i$とし,
  $F=(f_1,\ldots,f_n)$とする.
  \begin{enumerate}
    \item $F$は$\KK^n$の基底であることを示せ.
    \item $F$の双対基底$F^\ast=(\varepsilon_1^{F},\ldots,\varepsilon_m^{F})$
      に対し, $\varepsilon_i^{F}(\ee_j)$を求めよ.
    \item $F$の双対基底$F^\ast=(\varepsilon_1^{F},\ldots,\varepsilon_m^{F})$
      と$\aaa=\begin{pmatrix}a_1\\\vdots\\a_n\end{pmatrix}$に対し,
      $\varepsilon_i^{F}(\aaa)$を求めよ.
  \end{enumerate}
\end{quiz}

\begin{quiz}
  $V$を$\KK$上の$n$次元数ベクトル空間とする.
  つまり,
  $V=\KK^{n \times 1}$とする.
  また,
  $H=\KK^{1 \times n}$とする.
  $\vec{a}=(a_1,\ldots,a_n)\in H$に対し,
  \begin{align*}
    \shazo{\varphi_{\vec{a}}}{V}{\KK}
          {\xx=\begin{pmatrix}x_1\\\vdots\\x_n\end{pmatrix}}
          {\vec{a}\xx=\sum_{i=1}^na_ix_i}
  \end{align*}
  とする.
  また,
  \begin{align*}
    \shazo{\varphi}{H}{V^\ast}
          {\vec{a}}{\varphi_a}
  \end{align*}
  とする.
  このとき,
  以下の命題の真偽を判定し証明を与えよ:
  \begin{enumerate}
  \item
    $\vec{a}$に対し,
    $\varphi_{\vec{a}}\in V^\ast$である.
    つまり
    $\varphi_{\vec{a}}\colon V\to \KK$は
    $\KK$-線形写像である.
  \item
    $\varphi\in \Hom_{\KK}(H,V^\ast)$である.
    つまり
    $\varphi\colon H\to V^\ast$は
    $\KK$-線形写像である.
  \end{enumerate}
\end{quiz}
\begin{quiz}
  $V$を$\KK$上の$n$次元数ベクトル空間とする.
  つまり,
  $V=\KK^{n \times 1}$とする.
  $\KK^n$の標準基底$E=(\ee^{(n)}_1,\ldots,\ee^{(n)}_m)$に対し,
  その双対基底$E^\ast=(\varepsilon_1^{E},\ldots,\varepsilon_m^{E})$
  を考える.
  また,
  $H=\KK^{1 \times n}$とする.
  $\vec{a}=(a_1,\ldots,a_n)\in H$に対し,
  \begin{align*}
    \shazo{\varphi_{\vec{a}}}{V}{\KK}
          {\xx=\begin{pmatrix}x_1\\\vdots\\x_n\end{pmatrix}}
          {\sum_{i=1}^na_ix_i}
  \end{align*}
  とし,
  $\KK$-線形写像
  \begin{align*}
    \shazo{\varphi}{H}{V^\ast}
          {\vec{a}}{\varphi_a}
  \end{align*}
  について考える.
  \begin{enumerate}
  \item
    $\varphi_{\vec{a}}\colon H\to V^\ast$は
    同型写像であることを示せ.
  \item
    $\varphi_{\vec{a}}^{-1}(\varepsilon^{E}_i)$を求めよ.
  \end{enumerate}
\end{quiz}

\begin{quiz}
  次の命題の真偽を判定し証明を与えよ:
  $V$と$W$を$\KK$線形空間であるとする.
  $D=(v_1,\ldots,v_n)$を$V$の基底,
  $B=(w_1,\ldots,w_m)$を$V$の基底とする.
  $\beta^{B,D}_{i,j}$は,
  \begin{align*}
    \beta^{B,D}_{i,j} (v_j)=w_i
  \end{align*}
  を満たす
  $\KK$-線形写像$\beta^{B,D}_{i,j}\colon V\to W$とする.
  このとき,
  \begin{align*}
    (\beta^{B,D}_{1,1},\ldots,\beta^{B,D}_{1,n};\beta^{B,D}_{2,1},\ldots,\beta^{B,D}_{2,n};\ldots;\beta^{B,D}_{m,1},\ldots,\beta^{B,D}_{m,n})
  \end{align*}
  は$\Hom_\KK(V,W)$の生成系である.
\end{quiz}

\begin{quiz}
  次の命題の真偽を判定し証明を与えよ:
  $V$と$W$を$\KK$線形空間であるとする.
  $D=(v_1,\ldots,v_n)$を$V$の基底,
  $B=(w_1,\ldots,w_m)$を$V$の基底とする.
  $\beta^{B,D}_{i,j}$は,
  \begin{align*}
    \beta^{B,D}_{i,j} (v_j)=w_i
  \end{align*}
  を満たす
  $\KK$-線形写像$\beta^{B,D}_{i,j}\colon V\to W$とする.
  このとき,
  \begin{align*}
    (\beta^{B,D}_{1,1},\ldots,\beta^{B,D}_{1,n};\beta^{B,D}_{2,1},\ldots,\beta^{B,D}_{2,n};\ldots;\beta^{B,D}_{m,1},\ldots,\beta^{B,D}_{m,n})
  \end{align*}
  は一次独立である.
\end{quiz}

\begin{quiz}
  $V$と$W$を$\KK$線形空間であるとする.
  $D=(v_1,\ldots,v_n)$を$V$の基底,
  $B=(w_1,\ldots,w_m)$を$V$の基底とする.
  $\beta^{B,D}_{i,j}\colon V\to W$,
  $\nu_{B}\colon \KK^n\to V$,
  $\nu_{D}\colon \KK^m\to W$,
  は,
  \begin{align*}
    \beta^{B,D}_{i,j} (v_j)&=w_i,\\
    \nu_{B}(\ee^{(n)}_i)&=v_i,\\
    \nu_{D}(\ee^{(m)}_i)&=w_i,
  \end{align*}
  を満たす
  $\KK$-線形写像とする.
  また,
  $A\in \KK^{m\times n}$に対し,
  \begin{align*}
    \shazo{\mu_A}{\KK^n}{\KK^m}
    {x}{Ax}
  \end{align*}
  とし,
  $B(i,j)\in \KK^{m\times n}$を行列単位とする.
  このとき, 
  以下の命題の真偽を判定し証明を与えよ:
  \begin{enumerate}
  \item $\nu_{B}\circ \mu_{B(i,j)} \circ \nu_{D}^{-1}$.
  \item $\Hom_{\KK}(V,W)\simeq \KK^{m\times n}$.
  \end{enumerate}
\end{quiz}


\begin{quiz}
  $V$を$\KK$線形空間であるとし,
  $D=(v_1,\ldots,v_n)$を$V$の基底とする.
  $\varepsilon^{D}_i\colon V\to \KK$,
  は,
  \begin{align*}
    \varepsilon^{D}_i(v_j)=
    \begin{cases}
      1&(i=j)\\
      0&(i\neq j)
    \end{cases}
  \end{align*}
  を満たす
  $\KK$-線形写像とする.
  $D^\ast=(\varepsilon^{D}_1,\ldots,\varepsilon^{D}_n)$
  とする.
  このとき, 
  以下の命題の真偽を判定し証明を与えよ:
  $D^\ast$は$V^\ast$の基底である.
  つまり, 以下が成り立つ:
  \begin{enumerate}
  \item $D^\ast$は$V$の生成系である.
  \item $D^\ast$は一次独立である.
  \end{enumerate}
\end{quiz}

\begin{quiz}
  $\RR^2$の基底$D=(v_1,v_2)$に対し,
  $\varphi_D$を次の様に定義する:
  $D^\ast=(\varepsilon^{D}_1,\ldots,\varepsilon^{D}_n)$を$D$の双対基底とし,
  $\nu^{D}_{D^\ast}\colon V\to V^\ast$を
  \begin{align*}
    \nu^{D}_{D^\ast}(v_i)=\varepsilon^{D}_i
  \end{align*}
  を満たす$\KK$-線形写像とし,
  $\varphi_D=\nu^{D}_{D^\ast}$とおく.

  このとき, $\varphi_D\neq \varphi_{D'}$を満たす基底$\RR^2$の基底$D$, $D'$を与えよ.
\end{quiz}

\begin{quiz}
  次の命題の真偽を判定し証明を与えよ:
  $V$を$\KK$-線形空間とする.
  $v\in V$に対し,
  \begin{align*}
    \shazo{\epsilon_v}{V^\ast}{\KK}
    {\varphi}{\varphi(v)}
  \end{align*}
  とする.
  このとき,
  $\epsilon_v\in (V^\ast)^\ast$である.
  つまり
  $\epsilon_v\colon V^\ast \to \KK$は$\KK$-線形写像である.    
\end{quiz}

\begin{quiz}
  次の命題の真偽を判定し証明を与えよ:
  $V$を$\KK$-線形空間とする.
  $v\in V$に対し,
  \begin{align*}
    \shazo{\epsilon_v}{V^\ast}{\KK}
    {\varphi}{\varphi(v)}
  \end{align*}
  とする.
  また, $\Phi$を,
  \begin{align*}
    \shazo{\Phi}{V}{(V^\ast)^\ast}
    {v}{\epsilon_v}
  \end{align*}
  とする.
  このとき, 
  $\Phi\in\Hom_\KK( V, (V^\ast)^\ast)$
  である.
  つまり,
  $\Phi\colon V\to (V^\ast)^\ast$
  は線形写像である.    
\end{quiz}

\begin{quiz}
  次の命題の真偽を判定し証明を与えよ:
  $V$を$\KK$-線形空間とする.
  $v\in V$に対し,
  \begin{align*}
    \shazo{\epsilon_v}{V^\ast}{\KK}
    {\varphi}{\varphi(v)}
  \end{align*}
  とする.
  また, $\Phi$を,
  \begin{align*}
    \shazo{\Phi}{V}{(V^\ast)^\ast}
    {v}{\epsilon_v}
  \end{align*}
  とする.
  このとき, 
  $\Phi\colon V\to (V^\ast)^\ast$
  は単射である.    
\end{quiz}

\begin{quiz}
  次の命題の真偽を判定し証明を与えよ:
  $V$, $W$を$\KK$-線形空間とする.
  $\varphi\colon V\to W$を
  $\KK$-線形写像とする.
  $f\in W^\ast$ならば,
  $f\circ \varphi\in V^\ast$.
\end{quiz}

\begin{quiz}
  次の命題の真偽を判定し証明を与えよ:
  $V$, $W$を$\KK$-線形空間とする.
  $\KK$-線形写像
  $\varphi\colon V\to W$に対し,
  \begin{align*}
    \shazo{\Phi}{W^\ast}{V^\ast}
    {f}{f\circ \varphi}
  \end{align*}
  とおく.
  このとき,
  $\Phi$は$\KK$線形写像.
  つまり,
   $\Phi \in \Hom_{\KK}(W^\ast,{V^\ast})$.
\end{quiz}


\section{\Cref{chap:quotient}の後の問題}
主に次元定理について知っていれば解けるであろう問題.

\subsection{問題A}
% 講義: 次元定理
%% \begin{quiz}
%%   列ベクトル表示された
%%   $n$次正方行列
%%   $A=(\aaa_1|\cdots|\aaa_n)$
%%   を考える.
%%   このとき, 以下が同値であることを示せ:
%%   \begin{enumerate}
%%   \item $A$が正則である.
%%   \item $(\aaa_1,\ldots,\aaa_n)$が$\KK^n$の基底である.
%%    \end{enumerate}
%% \end{quiz}


\begin{quiz}
  $\CC$は$\RR$-線形空間であり,
  \begin{align*}
    \shazo{\nu}{\RR^2}{\CC}
    {\begin{pmatrix}x\\y\end{pmatrix}}{x+y\sqrt{-1}}
  \end{align*}
  は同型写像であった.
  $\theta\in \RR$とし,
  $c=\cos(\theta)+\sin(\theta)\sqrt{-1}$とする.
  このとき, 
  \begin{align*}
    \shazo{\varphi}{\CC}{\CC}
    {z}{cz}
  \end{align*}
  とおく.
  \begin{enumerate}
  \item
    $\aaa=\begin{pmatrix}x\\y\end{pmatrix}$ に対し,
    $(\nu^{-1}\circ\varphi\circ\nu)(\aaa)$を求めよ.
  \item
    $(\nu^{-1}\circ\varphi\circ\nu)(\ee^{(2)}_1)$,
    $(\nu^{-1}\circ\varphi\circ\nu)(\ee^{(2)}_2)$
    を求めよ.
  \item
    次の条件を満たす
    $A\in \RR^{2\times 2}$
    を求めよ:
    \begin{align*}
      \aaa=\begin{pmatrix}x\\y\end{pmatrix}\in \RR^2
      \implies
      \nu(A\aaa)=\varphi(x+y\sqrt{-1}).
    \end{align*}
  \end{enumerate}
\end{quiz}


\subsection{問題B}

\begin{quiz}
  $A$を次の行列とする:
  \begin{align*}
    A=
    \begin{pmatrix}
      1&1&2&2&3&3&4&4\\
      1&2&3&4&5&6&7&8\\
      1&2&2&3&3&4&4&5
    \end{pmatrix}
  \end{align*}
  $\KK$線形写像
  \begin{align*}
    \shazo{\mu_A}{\KK^8}{\KK^3}
    {x}{Ax}
  \end{align*}
  に対し, $\dim_\KK(\Img(\mu_A))$を求めよ.
\end{quiz}

\begin{quiz}
  $V$を$3$次以下の$\RR$係数多項式を集めた集合とする.
  このとき,
  $f\in V$に対し, その微分$\frac{d}{dx}f$を対応させる
  $\RR$線形写像を$\varphi$とおく. つまり
  \begin{align*}
    \shazo{\varphi}{V}{V}
    {f(x)}{\frac{d}{dx}f}.
  \end{align*}
  このとき,
  $\dim_\KK(\Img(\varphi))$,
  $\dim_\KK(\Ker(\varphi))$
  を求めよ.
\end{quiz}


\begin{quiz}
  $V$, $W$を$\KK$-線形空間とし,
  $\varphi\colon V\to W$を$\KK$-線形写像とする.
  このとき,
  以下の命題の真偽を判定し証明を与えよ:
  \begin{enumerate}
  \item $\dim_\KK(\Img(\varphi))\leq \dim_\KK(W)$
  \item $\dim_\KK(\Img(\varphi))\leq \dim_\KK(V)$
  \end{enumerate}
\end{quiz}

\begin{quiz}
  $V$, $W$, $U$を$\KK$-線形空間とし,
  $\varphi\colon V\to W$,
  $\psi\colon W\to U$を$\KK$-線形写像とする. 
 このとき,
  以下の命題の真偽を判定し証明を与えよ:
  \begin{enumerate}
  \item $\dim_\KK(\Img(\psi\circ\varphi))\leq \dim_\KK(\Img(\varphi))$
  \item $\dim_\KK(\Img(\psi\circ\varphi))\leq \dim_\KK(\Img(\psi))$
  \end{enumerate}
\end{quiz}

\begin{quiz}
  $A=(\aaa_1|\cdots|\aaa_n)\in \KK^{m\times n}$, $B=(\bbb_1|\cdots|\bbb_l)\in\KK^{n\times l}$とする.
  \begin{align*}
    \shazo{\mu_B}{\KK^{n}}{\KK^{l}}
    {x}{Bx}
  \end{align*}
  とし, $\ccc_j=\mu_A(\aaa_j)$とする.
  このとき,
  以下の命題の真偽を判定し証明を与えよ:
  \begin{enumerate}
  \item $\dim_\KK(\Braket{\ccc_1,\ldots,\ccc_n}_\KK)\leq \dim_\KK(\Braket{\aaa_1,\ldots,\aaa_n}_\KK)$.
  \item $\dim_\KK(\Braket{\ccc_1,\ldots,\ccc_n}_\KK)\leq \dim_\KK(\Braket{\bbb_1,\ldots,\bbb_l}_\KK)$.
  \end{enumerate}
\end{quiz}

\section{\Cref{sec:repmat:enumesp,sec:repmat:linmap}の後の問題}
主に, 表現行列の定義がわかっていれば解ける問題.

\subsection{問題 A}
\begin{quiz}
  $\aaa=\ee^{(3)}_1+\ee^{(3)}_2\in\RR^3$とする.
  $\aaa$と$\zzero_3$を通る直線を$L$とする.
  $L$を軸として, $\aaa$から$\zzero_3$を見たときに半時計回りになる向きに,
  $\xx\in\RR^3$を, $\theta$だけ回転させた点を$\varphi(\xx)$とおく.
  このとき,
  \begin{align*}
    \shazo{\varphi}{\RR^3}{\RR^3}{\xx}{\varphi(\xx)}
  \end{align*}
  は線形写像である.
  $\bbb=\ee^{(3)}_1-\ee^{(3)}_2$,
  $\ccc=\sqrt{2}\ee{(3)}_3$とすると,
  $(\aaa,\bbb,\ccc)$は$\RR^3$の基底である.
  \begin{enumerate}
    \item $\varphi$の$(\aaa,\bbb,\ccc)$, $(\aaa,\bbb,\ccc)$に関する表現行列を求めよ.
    \item $\varphi$の$(\ee^{(3)}_1,\ee^{(3)}_2,\ee^{(3)}_3)$, $(\ee^{(3)}_1,\ee^{(3)}_2,\ee^{(3)}_3)$に関する表現行列を求めよ.
  \end{enumerate}
\end{quiz}
%(行列式の回).
\subsection{問題 B}
\begin{quiz}
  $z\in\CC$に対し, $\overline z$で$z$の複素共軛を表す.
  \begin{enumerate}
  \item $\RR$-線形空間としての$\CC$の基底を(1組)与えよ.
  \item 次の$\RR$-線形変換$\varphi$に対し, 前問の基底に関する表現行列を与えよ:
    \begin{align*}
      \shazo{\varphi}{\CC}{\CC}
      {z}{\overline{z}}.
    \end{align*}
  \item 次の$\RR$-線形変換$\varphi$に対し, 前問の基底に関する表現行列を与えよ:
    \begin{align*}
      \shazo{\varphi}{\CC}{\CC}
      {z}{\frac{\overline{z}+z}{2}}.
    \end{align*}
  \item $a,b\in\RR$とする.
    次の$\RR$-線形変換$\varphi$に対し, 前問の基底に関する表現行列を与えよ:
    \begin{align*}
      \shazo{\varphi}{\CC}{\CC}
      {z}{a\overline{z}+bz}.
    \end{align*}
  \end{enumerate}
\end{quiz}

\begin{quiz}
    $x,y\in\RR$とし,
    $\alpha=x+y\sqrt{-1}$とする.
    次の$\RR$-線形変換$\varphi$の表現行列を与えよ:
    \begin{align*}
      \shazo{\varphi}{\CC}{\CC}
      {z}{\alpha z}.
    \end{align*}
\end{quiz}


\begin{quiz}
  $\CC^2$は$\CC$-線形空間でもあるが,
  $\RR$-線形空間でもある.
  \begin{enumerate}
  \item
    $\CC^2$の$\RR$-線形空間としての基底を1組与え,
    $\dim_\RR(\CC^2)$を与えよ.
  \item
    $a,a',b,b',c,c'\in \RR$とし,
    $\alpha=a+a'\sqrt{-1}$,
    $\beta=b+b'\sqrt{-1}$,
    $\gamma=c+c'\sqrt{-1}$とする.
    \begin{align*}
      A=
      \begin{pmatrix}
        \alpha&\beta\\
        0&\gamma 
      \end{pmatrix}
    \end{align*}
    とする.
  全問で与えた基底に関する,
  次の$\RR$-線形変換$\varphi$の表現行列を与えよ:
  \begin{align*}
    \shazo{\varphi}{\CC^2}{\CC^2}
    {\xx}{A\xx}.
  \end{align*}
  \end{enumerate}
\end{quiz}

\begin{quiz}
  $V_n=\Set{f\in \KK[x] |\deg(x) \leq n}\cup\Set{0}$
  とおく.\footnote{$0\neq f=\sum_{i=0}^n a_n x^n \in \KK[x]$に対し, $\deg(f)=\max\Set{i|a_i \neq 0}$とする.}
  \begin{enumerate}
    \item
      $V_n$は$\KK[x]$の部分空間であることを示せ.
    \item
      $\dim_\KK (V_n)$を求めよ.
  \end{enumerate}
\end{quiz}

\begin{quiz}
  $V_n=\Set{f\in \KK[x] |\deg(x) \leq n}\cup\Set{0}$
  とおく.
  基底を与え,
  その基底に関する,
  次の線形写像$\varphi$の表現行列を与えよ:
  \begin{align*}
    \shazo{\varphi}{V_3}{V_4}
    {f}{(1+x)f}.
  \end{align*}
\end{quiz}

\begin{quiz}
  $V_n=\Set{f\in \KK[x] |\deg(x) \leq n}\cup\Set{0}$
  とおく.
  基底を与え,
  その基底に関する,
  次の線形変換$\varphi$の表現行列を与えよ:
  \begin{align*}
    \shazo{\varphi}{V_3}{V_3}
    {f}{\frac{d}{dx}f}.
  \end{align*}
  ただし, $\frac{d}{dx}f$は$f$の微分を表す.
\end{quiz}

\begin{quiz}
  $V_n=\Set{f\in \KK[x] |\deg(x) \leq n}\cup\Set{0}$
  とおく.
  基底を与え,
  その基底に関する,
  次の線形変換$\varphi$の表現行列を与えよ:
  \begin{align*}
    \shazo{\varphi}{V_3}{V_3}
    {f}{x\frac{d}{dx}f}.
  \end{align*}
  ただし, $\frac{d}{dx}f$は$f$の微分を表す.
\end{quiz}




\begin{quiz}
  $\KK$-線形空間$\KK^{2\times 2}$について考える.
  \begin{align*}
    A=\begin{pmatrix} a&b\\c&d\end{pmatrix}
  \end{align*}
  とする.
  \begin{enumerate}
  \item
    $\KK$-線形空間$\KK^{2\times 2}$の基底を与えよ.
  \item
  その基底に関する次の$\KK$線形変換$\varphi$の行現行列を与えよ:
  \begin{align*}
    \shazo{\varphi}{\KK^2}{\KK^2}
    {X}{AX}.
  \end{align*}
  \item
  その基底に関する次の$\KK$線形変換$\psi$の行現行列を与えよ:
  \begin{align*}
    \shazo{\psi}{\KK^2}{\KK^2}
    {X}{XA}.
  \end{align*}
  \end{enumerate}
\end{quiz}

\begin{quiz}
  $\KK$-線形空間$\KK^{2\times 2}$について考える.
  \begin{align*}
    A=\begin{pmatrix} a&b\\c&d\end{pmatrix}
  \end{align*}
  とする.
    $\KK$-線形空間$\KK^{2\times 2}$の基底を与え,
  その基底に関する次の$\KK$線形変換$\varphi$の行現行列を与えよ:
  \begin{align*}
    \shazo{\varphi}{\KK^2}{\KK^2}
    {X}{AX-XA}.
  \end{align*}
\end{quiz}

\begin{quiz}
  $A\in \KK^{m\times n}$に対し,
  $\transposed{A}$で$A$の転置を表す.
  \begin{enumerate}
  \item
  $\KK$-線形空間$\KK^{2\times 2}$の基底を与え,
  その基底に関する次の$\KK$線形変換$\varphi$の行現行列を与えよ:
  \begin{align*}
    \shazo{\varphi}{\KK^{2\times 2}}{\KK^{2\times 2}}
    {X}{\transposed{X}}.
  \end{align*}
\item
  $\KK$-線形空間$\KK^{3\times 2}$, $\KK^{2\times 3}$の基底をそれぞれ与え,
  その基底に関する次の$\KK$線形変換$\varphi$の行現行列を与えよ:
  \begin{align*}
    \shazo{\varphi}{\KK^{2\times 3}}{\KK^{3\times 2}}
    {X}{\transposed{X}}.
  \end{align*}
  \end{enumerate}
\end{quiz}

\begin{quiz}
  $\RR$-線形空間
  $\RR^4$について考える.
  $\aaa\in\RR$の第$i$成分を$a_i$,
  $\bbb\in\RR$の第$i$成分を$b_i$とするとき,  
  $\braket{\aaa,\bbb}=\sum_{i=1}^4a_ib_i$とする.
  \begin{align*}
    \ccc=
  \begin{pmatrix}
   a\\b\\c\\d
  \end{pmatrix}
  \end{align*}
  とする.
  このとき,
  $\RR^4$と$\RR$の基底を与え,
  その基底に関する次の$\RR$-線形写像$\varphi$の表現行列を与えよ:
  \begin{align*}
    \shazo{\varphi}{\RR^4}{\RR}
    {\xx}{\Braket{\aaa,\xx}}.
  \end{align*}
\end{quiz}

\begin{quiz}
  $e_0,\ldots,e_4\in \RR^\RR$を
  \begin{align*}
    &\shazo{e_0}{\RR}{\RR}{x}{1}
    \\
    &\shazo{e_1}{\RR}{\RR}{x}{\cos(x)}
    \\
    &\shazo{e_2}{\RR}{\RR}{x}{\sin(x)}
    \\
    &\shazo{e_3}{\RR}{\RR}{x}{\cos(2x)}
    \\
    &\shazo{e_4}{\RR}{\RR}{x}{\sin(2x)}
  \end{align*}
  とし,
  $V=\Braket{e_0,e_1,\ldots,e_4}_\RR$とする,
  \begin{enumerate}
  \item $(e_0,e_1,\ldots,e_4)$は$V$の基底であることを示せ.
  \item 次の$\RR$-線形変換の$(e_0,e_1,\ldots,e_4)$に関する
    表現行列を求めよ:
    \begin{align*}
    \shazo{\varphi}{V}{V}
      {f}{\frac{d}{dx}f}.
    \end{align*}
    ただし$\frac{d}{dx}f$は関数$f$の導関数を表す.
  \end{enumerate}
\end{quiz}

\begin{quiz}
  $n=4$,
  $m=4$とする.
  $V$を$(e_1,\ldots,e_n)$を基底とする$\KK$-線形空間とする.
  $W$を$(f_1,\ldots,f_m)$を基底とする$\KK$-線形空間とする.
  次の図形において,
  $k$と書かれた矢印が
  $e_i$から$e_j$に向かって引かれている
  (つまり$e_i\xrightarrow{k} e_j$となっている)
  とき,
  $v_k=e_i-e_j$とおく:
  \begin{align*}
    \begin{picture}(40,40)
      \put(0,0){\makebox(0,0){1}}
      \put(40,0){\makebox(0,0){2}}
      \put(40,40){\makebox(0,0){3}}
      \put(0,40){\makebox(0,0){4}}
      \put(5,0){\vector(1,0){30}}
      \put(40,5){\vector(0,1){30}}
      \put(35,40){\vector(-1,0){30}}
      \put(0,35){\vector(0,-1){30}}
      \put(20,-5){\makebox(0,0){1}}
      \put(45,20){\makebox(0,0){2}}
      \put(20,45){\makebox(0,0){3}}
      \put(-5,20){\makebox(0,0){4}}
    \end{picture}
  \end{align*}
  線形写像$\varphi\colon W\to V$を
  $\varphi(f_i)=v_i$で定義する.
  \begin{enumerate}
  \item $\varphi$の$(f_1,\ldots,f_m)$, $(e_1,\ldots,e_n)$に関する表現行列を求めよ.
  \item $\Img(\varphi)$の基底を求めよ.
  \item $\Ker(\varphi)$の基底を求めよ.
  \end{enumerate}
\end{quiz}

\begin{quiz}
  $n=4$,
  $m=5$とする.
  $V$を$(e_1,\ldots,e_n)$を基底とする$\KK$-線形空間とする.
  $W$を$(f_1,\ldots,f_m)$を基底とする$\KK$-線形空間とする.
  次の図形において,
  $k$と書かれた矢印が
  $e_i$から$e_j$に向かって引かれている
  (つまり$e_i\xrightarrow{k} e_j$となっている)
  とき,
  $v_k=e_i-e_j$とおく:
  \begin{align*}
    \begin{picture}(40,40)
      \put(0,0){\makebox(0,0){1}}
      \put(40,0){\makebox(0,0){2}}
      \put(40,40){\makebox(0,0){3}}
      \put(0,40){\makebox(0,0){4}}
      \put(5,0){\vector(1,0){30}}
      \put(40,5){\vector(0,1){30}}
      \put(35,40){\vector(-1,0){30}}
      \put(0,35){\vector(0,-1){30}}
      \put(4,4){\vector(1,1){32}}
      \put(20,-5){\makebox(0,0){1}}
      \put(45,20){\makebox(0,0){2}}
      \put(20,45){\makebox(0,0){3}}
      \put(-5,20){\makebox(0,0){4}}
      \put(23,17){\makebox(0,0){5}}
    \end{picture}
  \end{align*}
  線形写像$\varphi\colon W\to V$を
  $\varphi(f_i)=v_i$で定義する.
  \begin{enumerate}
  \item $\varphi$の$(f_1,\ldots,f_m)$, $(e_1,\ldots,e_n)$に関する表現行列を求めよ.
  \item $\Img(\varphi)$の基底を求めよ.
  \item $\Ker(\varphi)$の基底を求めよ.
  \end{enumerate}
\end{quiz}

\begin{quiz}
  $n=4$,
  $m=2$とする.
  $V$を$(e_1,\ldots,e_n)$を基底とする$\KK$-線形空間とする.
  $W$を$(f_1,\ldots,f_m)$を基底とする$\KK$-線形空間とする.
  次の図形において,
  $k$と書かれた矢印が
  $e_i$から$e_j$に向かって引かれている
  (つまり$e_i\xrightarrow{k} e_j$となっている)
  とき,
  $v_k=e_i-e_j$とおく:
  \begin{align*}
    \begin{picture}(40,40)
      \put(0,0){\makebox(0,0){1}}
      \put(40,0){\makebox(0,0){2}}
      \put(40,40){\makebox(0,0){3}}
      \put(0,40){\makebox(0,0){4}}
      \put(5,0){\vector(1,0){30}}      
      %\put(35,40){\vector(-1,0){30}}
      %\put(0,35){\vector(0,-1){30}}
      \put(40,5){\vector(0,1){30}}
      %\put(4,4){\vector(1,1){32}}
      \put(20,-5){\makebox(0,0){1}}
      \put(45,20){\makebox(0,0){2}}
      %\put(20,45){\makebox(0,0){3}}
      %\put(-5,20){\makebox(0,0){4}}
      %\put(23,17){\makebox(0,0){5}}
    \end{picture}
  \end{align*}
  線形写像$\varphi\colon W\to V$を
  $\varphi(f_i)=v_i$で定義する.
  \begin{enumerate}
  \item $\varphi$の$(f_1,\ldots,f_m)$, $(e_1,\ldots,e_n)$に関する表現行列を求めよ.
  \item $\Img(\varphi)$の基底を求めよ.
  \item $\Ker(\varphi)$の基底を求めよ.
  \end{enumerate}
\end{quiz}

\begin{quiz}
  次の$\RR$-線形写像$\varphi$の標準基底による表現行列を与えよ:
  \begin{align*}
    \shazo{\varphi}{\RR^3}{\RR^2}
    {\begin{pmatrix}x\\y\\z\end{pmatrix}}{2x+4y\\x+5z}
  \end{align*}
\end{quiz}

\begin{quiz}
  次を満たす$\RR$-線形写像$\varphi\colon \RR^2\to \RR^3$の標準基底による表現行列を与えよ:
  \begin{align*}
    \varphi(\begin{pmatrix}1\\1\end{pmatrix})=\begin{pmatrix}2\\4\\6\end{pmatrix}
    \varphi(\begin{pmatrix}1\\-1\end{pmatrix})=\begin{pmatrix}1\\3\\7\end{pmatrix}
  \end{align*}
\end{quiz}

\begin{quiz}
  $\xx\in \RR^2$に対し,
  $\xx$を
  $2$次元ユークリッド空間$\RR^2$において原点を中心に$\theta$だけ回転させた点
  $\xx'$に対応させる変換は,
  $\RR$-線形変換である.
  この線形変換の表示基底に関する表現行列を求めよ.
\end{quiz}

\begin{quiz}
  $\aaa\in\RR^3$を
  \begin{align*}
    \aaa=\begin{pmatrix}1\\0\\0\end{pmatrix}
  \end{align*}
  とする.
  $\xx\in \RR^3$に対し,
  $\xx$を
  $3$次元ユークリッド空間$\RR^2$において$x$軸
  (つまり原点と$\aaa$を通る直線)
  を軸に$\theta$だけ回転させた点
  $\xx'$
  (つまり,
  原点から$\aaa$を見たときに時計回りになる方向に$\theta$だけ回転させた点$\xx'$)
  に対応させる変換は,
  $\RR$-線形変換である.
  この線形変換の表示基底に関する表現行列を求めよ.
\end{quiz}

\begin{quiz}
  $\aaa\in\RR^3$を
  \begin{align*}
    \aaa=\begin{pmatrix}1\\1\\1\end{pmatrix}
  \end{align*}
  とする.
  $\xx\in \RR^3$に対し,
  $\xx$を
  $3$次元ユークリッド空間$\RR^2$において
  原点と$\aaa$を通る直線
  を軸に$\theta$だけ回転させた点
  $\xx'$
  (つまり,
  原点から$\aaa$を見たときに時計回りになる方向に$\theta$だけ回転させた点$\xx'$)
  に対応させる変換は,
  $\RR$-線形変換である.
  この線形変換の表示基底に関する表現行列を求めよ.
\end{quiz}


\section{\Cref{chap:det}の後の問題}
行列式に関連し, 交代性, 多重線型性の定義をわかっていれば解ける問題.
%\section{10/15回目: \schoolCalender{10}}
%\subsection{レポート課題 (締め切り: \schoolCalender{11})}
\subsection{問題 A}
\begin{quiz}
  \begin{align*}
    A=
    \begin{pmatrix}
      1&1&1\\
      1&1&1\\
      1&1&1
    \end{pmatrix}
  \end{align*}
  とする.
  \begin{enumerate}
  \item 次の$v$に対し, $Av=cv$となる$c$が存在することを示せ:
    \begin{align*}
      v=\begin{pmatrix}1\\1\\1\end{pmatrix}
    \end{align*}
  \item 次の$v$に対し, $Av=cv$となる$c$が存在することを示せ:
    \begin{align*}
      v=\begin{pmatrix}1\\-1\\0\end{pmatrix}
    \end{align*}
  \item 次の$v$に対し, $Av=cv$となる$c$が存在することを示せ:
    \begin{align*}
      v=\begin{pmatrix}0\\1\\-1\end{pmatrix}
    \end{align*}
  \item 次の$v$に対し, $Av=cv$となる$c$が存在しないことを示せ:
    \begin{align*}
      v=\begin{pmatrix}1\\0\\0\end{pmatrix}
    \end{align*}
  \end{enumerate}
\end{quiz}

\subsection{問題 B}

\begin{quiz}
  \begin{align*}
    A=
    \begin{pmatrix}
      1&1&1&1\\
      1&2&3&4\\
      2&3&4&5
    \end{pmatrix}
  \end{align*}
  とし,
  \begin{align*}
    \shazo{\mu_A}{\KK^4}{\KK^3}
    {x}{Ax}
  \end{align*}
  とする.
  また,
  \begin{align*}
    \shazo{\varphi}{\KK^4}{\Img(\mu_A)}
    {x}{\mu_A(x)}
  \end{align*}
  とする.
  基底を自分で与え,
  その基底に関する$\varphi$の表現行列を求めよ.
\end{quiz}

\begin{quiz}
  \begin{align*}
    A=
    \begin{pmatrix}
      1&1&1\\
      1&4&7\\
      2&5&6\\
      2&3&4
    \end{pmatrix}
  \end{align*}
  とし,
  \begin{align*}
    \shazo{\mu_A}{\KK^3}{\KK^4}
    {x}{Ax}
  \end{align*}
  とする.
  また,
  \begin{align*}
    V=\Braket{
      \begin{pmatrix}1\\1\\1\end{pmatrix},
      \begin{pmatrix}1\\2\\3\end{pmatrix},
      \begin{pmatrix}2\\3\\4\end{pmatrix},
    }_{\KK}
  \end{align*}
  とし,
  \begin{align*}
    \shazo{\varphi}{V}{\KK^4}
    {x}{\mu_A(x)}
  \end{align*}
  とする.
  基底を自分で与え,
  その基底に関する$\varphi$の表現行列を求めよ.
\end{quiz}

\begin{quiz}
  $\sigma\in S_n$とする.
  \begin{align*}
    \shazo{\varphi}{\KK^n}{\KK^n}
    {\begin{pmatrix}x_1\\\vdots\\v_n\end{pmatrix}}{\begin{pmatrix}x_{\sigma(n)}\\\vdots\\v_{\sigma(n)}\end{pmatrix}}
  \end{align*}
  とする.
  $\KK^n$の標準基底に関する$\varphi$の表現行列を求めよ.
\end{quiz}

\begin{quiz}
  $V$を$4$次以下の実係数多項式全体のなす$\RR$-線形空間とする.
  $\varphi$を次の線形変換とする:
  \begin{align*}
    \shazo{\varphi}{V}{V}
    {f}{x\frac{d}{dx}f}
  \end{align*}
  ただし$\frac{d}{dx}f$は$f$の微分を表す.
  \begin{enumerate}
  \item $V$の基底を与え, $\varphi$の表現行列を与えよ.
  \item $\varphi$が同型写像か判定し証明を与えよ.
  \end{enumerate}
\end{quiz}

\begin{quiz}
  $n$次元$\KK$線形空間$U$が,
  $U=V\oplus W$と内部直和に分解しているとする.
  線形変換
  $\varphi\colon U\to U$
  が
  $v\in V$に対して
  $\varphi(v)=v$,
  $w\in W$に対して
  $\varphi(w)=w$
  を満たしているとする.
  基底を自分で与え,
  その基底に関する$\varphi$の表現行列を求めよ. 
\end{quiz}


\begin{quiz}
  $\CC$は$\RR$-線形空間であった.
  $V$を$3$次以下の実多項式全体のなす$\RR$-線形空間とする.
  $\alpha=1+\sqrt{-1}$とし,
  $\varphi$を次の線形変換とする:
  \begin{align*}
    \shazo{\varphi}{V}{\CC}
          {f}{f|_{x=\alpha}},
  \end{align*}
  ただし$f|_{x=\alpha}$は$f$に現れる不定元$x$を$\alpha$に置き換えて得られる
  複素数を表す.
  基底を自分で与え,
  その基底に関する$\varphi$の表現行列を求めよ. 
\end{quiz}

\begin{quiz}
    $V$を$n$次元$\KK$-線形空間とし,
  $D$を$V$の基底とする.
  $W$を$m$次元$\KK$-線形空間とし,
  $B$を$W$の基底とする.
  $\varphi\in \Hom_\KK(V,W)$に対し,
  $D$, $B$に関する$\varphi$の表現行列を$A_\varphi$とかくことにする.
  \begin{align*}
    \shazo{\Phi}{\Hom_\KK(V,W)}{\KK^{m\times n}}
    {\varphi}{A_\varphi}
  \end{align*}
  とする.
  このとき,
  以下の命題の真偽を判定し証明を与えよ:
  \begin{enumerate}
  \item これは同型写像である.
  \end{enumerate}
\end{quiz}


\begin{quiz}
  次の命題の真偽を判定し証明を与えよ:
  $V$を$n$次元$\KK$-線形空間とし,
  $W$を$m$次元$\KK$-線形空間とする.
  $\varphi\colon V\to W$を$\KK$-線形写像であるとする.
  このとき,
  $\varphi$の
  表現行列が
  \begin{align*}
    \left(
    \begin{array}{c|c}
      E_r|O_{r,n-r}\\\hline
      O_{m-r}|O_{m-r,n-r}
    \end{array}
    \right)
  \end{align*}
  となるような
  $V$, $W$の基底が存在する.
\end{quiz}

\begin{quiz}
  $V$, $V'$を$n$次元$\KK$-線形空間とし,
  $W$, $W'$を$m$次元$\KK$-線形空間とする.
  $\varphi\colon V\to V'$を同型写像,
  $\phi\colon W\to W'$を同型写像とする.
  $\KK$-線形写像
  $\psi\colon V\to W$,
  $\psi'\colon V'\to W'$が
  $\psi'\circ\varphi=\phi\circ\psi$
  を満たしているとする.
  $D=(v_1,\ldots,v_n)$を$V$の基底,
  $B=(w_1,\ldots,w_m)$を$W$の基底とし,
  $D$, $B$に関する$\psi$の表現行列を$A$とする.
  このとき,
  以下の命題の真偽を判定し証明を与えよ:
  \begin{enumerate}
  \item $D'=(\varphi(v_1),\ldots,\varphi(v_n))$は$V'$の基底.
  \item $B'=(\phi(w_1),\ldots,\phi(w_m))$は$V'$の基底.
  \item $\psi'$は$D'$, $B'$に関する表現行列は$A$.
  \end{enumerate}
\end{quiz}

\begin{quiz}
  次の命題の真偽を判定し証明を与えよ:
  $V$を$\KK$-線形空間とする.
  $D=(v_1,\ldots,v_n)$,
  $B=(w_1,\ldots,w_n)$を$V$の基底とする.
  次が成り立つとき, $D$から$B$への基底の変換行列は上半三角行列である:
  \begin{align*}
    k\in\Set{1,\ldots,n}\implies \Braket{v_1,\ldots,v_k}_\KK=\Braket{w_1,\ldots,w_k}_\KK.
  \end{align*}
\end{quiz}

\begin{quiz}
  $V$, $W$を$\KK$-線形空間とする.
  \begin{align*}
    \shazo{F}{V\times V}{W}
    {(v,u)}{F(v,u)}
  \end{align*}
  が以下の条件を満たしているとする:
  \begin{align*}
    v, v', u \in V,\ c, c'\in \KK &\implies F(cv+c'v',u)=cF(v,u)+c'F(v',u).\\
    v, u, u' \in V,\ c, c'\in \KK &\implies F(v,cu+c'u')=cF(v,u')+c'F(v,u').\\
    v, u \in V,\ v=u &\implies F(v,u)=0_W.
  \end{align*}
  このとき, 次の命題の真偽を判定し証明を与えよ:
  \begin{align*}
    v, u \in V &\implies F(v,u)=-F(v,u).
  \end{align*}
\end{quiz}

\begin{quiz}
  $\KK$を体とし$1+1\neq 0$とする.
  $V$, $W$を$\KK$-線形空間とする.
  \begin{align*}
    \shazo{F}{V\times V}{W}
    {(v,u)}{F(v,u)}
  \end{align*}
  が以下の条件を満たしているとする:
  \begin{align*}
    v, v', u \in V,\ c, c'\in \KK &\implies F(cv+c'v',u)=cF(v,u)+c'F(v',u).\\
    v, u, u' \in V,\ c, c'\in \KK &\implies F(v,cu+c'u')=cF(v,u')+c'F(v,u').\\
    v, u \in V &\implies F(v,u)=-F(v,u).
  \end{align*}
  このとき, 次の命題の真偽を判定し証明を与えよ:
  \begin{align*}
    v, u \in V,\ v=u &\implies F(v,u)=0_W.
  \end{align*}
\end{quiz}

\begin{quiz}
  次の命題の真偽を判定し証明を与えよ:
  $V$, $W$, $U$を$\KK$-線形空間とする.
  \begin{align*}
    \shazo{F}{V\times V}{W}
    {(v,u)}{F(v,u)}
  \end{align*}
  多重線形かつ交代的であるとする.
  $\varphi\colon W\to U$を$\KK$線形写像とする.
  このとき, $\varphi\circ F$は
    多重線形かつ交代的である.
\end{quiz}

\begin{quiz}
  次の命題の真偽を判定し証明を与えよ:
  $V$, $W$, $U$を$\KK$-線形空間とする.
  \begin{align*}
    \shazo{F}{V\times V}{W}
    {(v,u)}{F(v,u)}
  \end{align*}
  多重線形かつ交代的であるとする.
  $\varphi\colon U\to V$を$\KK$線形写像とする.
  このとき,
  \begin{align*}
    \shazo{F'}{U\times U}{W}
    {(v,u)}{F(\varphi(v),\varphi(u))}
  \end{align*}
  は
  多重線形かつ交代的である.
\end{quiz}

%\endinput
%\newpage
\section{\Cref{chap:eigen}の後の問題}
主に固有空間についてわかっていれば解ける問題.
%\section{11/15回目: \schoolCalender{11}}
%\subsection{レポート課題 (締め切り: \schoolCalender{12})}
\subsection{問題 A}
\begin{quiz}
  \begin{align*}
    A=
    \begin{pmatrix}
      1&1&1\\
      1&1&1\\
      1&1&1
    \end{pmatrix}
  \end{align*}
  とし, $\CC^3$上の線形変換
  \begin{align*}
    \shazo{\mu_A}{\CC^3}{\CC^3}
    {\xx}{A\xx}
  \end{align*}
  を考える. また,
  \begin{align*}
    u&=\begin{pmatrix}1\\1\\1\end{pmatrix}&
    v&=\begin{pmatrix}1\\-1\\0\end{pmatrix}&
    w&=\begin{pmatrix}0\\1\\-1\end{pmatrix}
  \end{align*}
  とする.
  \begin{enumerate}
  \item $u,v,w$は$A$の固有ベクトルであることを示せ.
  \item 列ベクトル表示された行列$P=(u|v|w)$は正則であることを示せ.
  \item 基底$(u,v,w)$に関する$\mu_A$の表現行列を求めよ.
  \end{enumerate}
\end{quiz}

\subsection{問題 B}
\begin{quiz}
  $A\in \CC^{4\times 4}$を次の行列とする:
  \begin{align*}
    A=
    \begin{pmatrix}
      1&2&1&2\\
      2&1&2&1\\
      1&2&1&2\\
      2&1&2&1\\
    \end{pmatrix}
  \end{align*}
  以下の命題の真偽を判定し証明を与えよ:
  \begin{enumerate}
  \item 次のベクトルが$A$の固有ベクトルである:
      $\begin{pmatrix}1\\1\\1\\1\end{pmatrix}$
  \item 次のベクトルが$A$の固有ベクトルではない:
      $\begin{pmatrix}1\\2\\3\\4\end{pmatrix}$
  \end{enumerate}
\end{quiz}

\begin{quiz}
  $A\in \CC^{4\times 4}$を次の行列とする:
  \begin{align*}
    A=
    \begin{pmatrix}
      1&2&3&4\\
      0&1&2&3\\
      0&0&1&2\\
      0&0&0&1\\
    \end{pmatrix}
  \end{align*}
  $A$の固有空間をすべて求めよ.
\end{quiz}


\begin{quiz}
  $V$を$3$次以下の$\RR$係数多項式を集めた集合とする.
  このとき,
  $f\in V$に対し, その微分$\frac{d}{dx}f$を対応させる
  $\RR$-線形変換を$\varphi$とおく. つまり
  \begin{align*}
    \shazo{\varphi}{V}{V}
    {f(x)}{\frac{d}{dx}f}.
  \end{align*}
  このとき,
  $\varphi$の固有空間を求めよ.
\end{quiz}

\begin{quiz}
  $V_n=\Set{f\in \KK[x] |\deg(x) \leq n}\cup\Set{0}$
  とおく.
  次の$\RR$-線形変換$\varphi$の固有空間を求めよ:
  \begin{align*}
    \shazo{\varphi}{V_3}{V_3}
    {f}{x\frac{d}{dx}f}.
  \end{align*}
  ただし, $\frac{d}{dx}f$は$f$の微分を表す.
\end{quiz}

\begin{quiz}
  次の$\RR$-線形変換$\varphi$の固有空間を求めよ:
  \begin{align*}
    \shazo{\varphi}{\CC}{\CC}
    {z}{\overline{z}}.
  \end{align*}
  ただし, $\overline{z}$は$z$の複素共軛を表す.
\end{quiz}

\begin{quiz}
  $\aaa=(a_i)_{i}, \bbb=(b_i)_i \in \RR^5$に対し,
  $\Braket{\aaa,\bbb}=\sum_{i=1}^na_ib_i$とする.
  \begin{align*}
    \ccc=\frac{1}{\sqrt 5}\begin{pmatrix}1\\1\\1\\1\\1\end{pmatrix}
  \end{align*}  
  次の$\RR$-線形変換$\varphi$の固有空間を求めよ:
  \begin{align*}
    \shazo{\varphi}{\RR}{\RR}
    {\xx}{\Braket{\ccc,\xx}\ccc}.
  \end{align*}
\end{quiz}



\begin{quiz}
  $\varphi\colon V\to V$を全単射な$\KK$-線形変換とする.
  以下の命題の真偽を判定し証明を与えよ:
  \begin{enumerate}
  \item
      $0$は$\varphi$の固有値ではない.
  \item
  $\lambda$が$\varphi$の固有値であるとき,
    $\lambda^{-1}$は$\varphi^{-1}$の固有値である.
  \end{enumerate}
\end{quiz}


\begin{quiz}
  以下の線形変換の固有空間を求めよ:
  \begin{enumerate}
  \item $\CC$-線形変換$\varphi$:
    \begin{align*}
      \shazo{\varphi}{\CC^3}{\CC^3}
      {\xx}{\begin{pmatrix}0&-1&0\\1&0&0\\0&0&0\end{pmatrix}}
    \end{align*}
  \item $\RR$-線形変換$\psi$:
    \begin{align*}
      \shazo{\psi}{\RR^3}{\RR^3}
      {\xx}{\begin{pmatrix}0&-1&0\\1&0&0\\0&0&0\end{pmatrix}}
    \end{align*}
  \end{enumerate}
\end{quiz}

\begin{quiz}
  以下の線形変換の固有空間を求めよ:
  \begin{enumerate}
  \item $\CC$-線形変換$\varphi$:
    \begin{align*}
      \shazo{\varphi}{\CC^3}{\CC^3}
      {\xx}{\begin{pmatrix}0&2&0\\1&0&0\\0&0&0\end{pmatrix}}
    \end{align*}
  \item $\RR$-線形変換$\psi$:
    \begin{align*}
      \shazo{\psi}{\RR^3}{\RR^3}
      {\xx}{\begin{pmatrix}0&2&0\\1&0&0\\0&0&0\end{pmatrix}}
    \end{align*}
  \item $\QQ$-線形変換$\psi$:
    \begin{align*}
      \shazo{\phi}{\QQ^3}{\QQ^3}
      {\xx}{\begin{pmatrix}0&2&0\\1&0&0\\0&0&0\end{pmatrix}}
    \end{align*}
  \end{enumerate}
\end{quiz}

\begin{quiz}
  次の命題の真偽を判定し証明を与えよ:
  $A\in\KK^{n\times n}$に対し,
  $0$が$A$の固有値であることと,
  $\rank(A)<n$は同値である.
  また,
  $A$の固有値$0$に属する固有空間の次元は,
  $n-\rank(A)$である.
\end{quiz}

\begin{quiz}
  $\zzero_n\neq \aaa=A\in\KK^{n\times 1}=\KK^n$,
  $B\in\KK^{1\times n}$とし,
  $C=AB\in\KK^{n\times n}$, $(\lambda)=BA$とする.
  このとき,
  \begin{enumerate}
  \item $\aaa$は固有値$\lambda$に属する$C$の固有ベクトルである.
  \item 固有値$0$に属する$A$の固有空間の次元を求めよ.
  \item $A$の固有値をすべて求めよ.
  \end{enumerate}
\end{quiz}



\section{\Cref{chap:diagonalize}の後の問題}
%\section{12/15回目: \schoolCalender{12}}
%\subsection{レポート課題 (締め切り: \schoolCalender{13})}
%(次回小テストのため) レポート課題は出題しません.
\subsection{問題 A}
\begin{quiz}
  行列$A$を以下の行列とする:
  \begin{align*}
    A=\begin{pmatrix}1&2\\1&3\end{pmatrix}.
  \end{align*}
  また,
  $\det(xE_2-A)=a_0x^0+a_1x+a_2x^2$とおく.
  \begin{enumerate}
  \item $a_0,a_1,a_2$を求めよ.
  \item $\det(A)$を求めよ.
  \item $\tr(A)$を求めよ.
  \item $a_0E_3+a_1A+a_2A^2$を求めよ.
  \end{enumerate}
\end{quiz}

\begin{quiz}
  行列$A$を以下の行列とする:
  \begin{align*}
    A=\begin{pmatrix}1&2&3\\1&4&5\\1&5&6\end{pmatrix}.
  \end{align*}
  また,
  $\det(xE_3-A)=a_0x^0+a_1x+a_2x^2+a_3x^3$とおく.
  \begin{enumerate}
  \item $a_0,a_1,a_2,a_3$を求めよ.
  \item $\det(A)$を求めよ.
  \item $\tr(A)$を求めよ.
  \item $a_0E_3+a_1A+a_2A^2+a_3A^3$を求めよ.
  \end{enumerate}
\end{quiz}

\begin{quiz}
  行列$A$を以下の行列とする:
  \begin{align*}
    A=\begin{pmatrix}1&1&1&1\\1&1&1&1\\1&1&1&1\\1&1&1&1\end{pmatrix}.
  \end{align*}
  また,
  $\det(xE_4-A)=a_0x^0+a_1x+a_2x^2+a_3x^3+a_4x^4$とおく.
  \begin{enumerate}
  \item $a_0,a_1,a_2,a_3,a_4$を求めよ.
  \item $\det(A)$を求めよ.
  \item $\tr(A)$を求めよ.
  \item $a_0E_3+a_1A+a_2A^2+a_3A^3+a_4A^4$を求めよ.
  \end{enumerate}
\end{quiz}

\subsection{問題 B}
\begin{quiz}
  次の命題の真偽を判定し証明を与えよ:
  $V$を$3$次以下の$\RR$係数多項式を集めた集合とする.
  このとき,
  $f\in V$に対し, その微分$\frac{d}{dx}f$を対応させる
  $\RR$-線形変換を$\varphi$とおく. つまり
  \begin{align*}
    \shazo{\varphi}{V}{V}
    {f(x)}{\frac{d}{dx}f}.
  \end{align*}
  このとき,
  $\varphi$は対角化可能ではない.
\end{quiz}

\begin{quiz}
  $V_n=\Set{f\in \KK[x] |\deg(x) \leq n}\cup\Set{0}$
  とおく.
  \begin{align*}
    \shazo{\varphi}{V_3}{V_3}
    {f}{x\frac{d}{dx}f}.
  \end{align*}
  ただし, $\frac{d}{dx}f$は$f$の微分を表す.
  $\varphi$の表現行列が対角行列となるような$V_3$の基底と,
  そのときの表現行列を求めよ.
\end{quiz}

\begin{quiz}
  次の$\RR$-線形変換$\varphi$について考える:
  \begin{align*}
    \shazo{\varphi}{\CC}{\CC}
    {z}{\overline{z}}.
  \end{align*}
  ただし, $\overline{z}$は$z$の複素共軛を表す.
  $\varphi$の表現行列が対角行列となるような$V_3$の基底と,
  そのときの表現行列を求めよ.
\end{quiz}

\begin{quiz}
  $\aaa=(a_i)_{i}, \bbb=(b_i)_i \in \RR^5$に対し,
  $\Braket{\aaa,\bbb}=\sum_{i=1}^na_ib_i$とする.
  \begin{align*}
    \ccc=\frac{1}{\sqrt 5}\begin{pmatrix}1\\1\\1\\1\\1\end{pmatrix}
  \end{align*}  
  次の$\RR$-線形変換$\varphi$について考える:
  \begin{align*}
    \shazo{\varphi}{\RR}{\RR}
    {\xx}{\Braket{\ccc,\xx}\ccc}.
  \end{align*}
  $\varphi$の表現行列が対角行列となるような$V_3$の基底と,
  そのときの表現行列を求めよ.
\end{quiz}


\begin{quiz}
  次の$\RR$-線形変換$\varphi$について考える:
  \begin{align*}
    \shazo{\varphi}{\RR^3}{\RR^3}
    {\begin{pmatrix}x_1\\x_2\\x_3\end{pmatrix}}{\begin{pmatrix}x_2\\x_1\\x_3\end{pmatrix}}.
  \end{align*}
  $\varphi$の表現行列が対角行列となるような$V_3$の基底と,
  そのときの表現行列を求めよ.
\end{quiz}

\begin{quiz}
  $A=\transposed{A}$を満たす行列を対称行列,
  $A=-\transposed{A}$を満たす行列を交代行列と呼ぶ.
  次の$\RR$-線形変換$\varphi$について考える:
  \begin{align*}
    \shazo{\varphi}{\RR^{3\times 3}}{\RR^{3\times 3}}
    {A}{\transposed{A}}.
  \end{align*}
  \begin{enumerate}
  \item $A\in \RR^{3\times 3}$とする. このとき, $A=B+C$を満たす対称行列$B$と交代行列$C$がただ1組存在することを示せ.
  \item
      $\varphi$の表現行列が対角行列となるような$\RR^{3\times 3}$の基底と,
  そのときの表現行列を求めよ.
  \end{enumerate}

\end{quiz}


\begin{quiz}
  \begin{align*}
    A=\begin{pmatrix}0&1\\-1&0\end{pmatrix}
  \end{align*}
  とする.
  \begin{enumerate}
  \item 複素行列$A$を対角化せよ.
  \item 実行列$A$は対角化可能ではないことを示せ.
  \end{enumerate}
\end{quiz}

\begin{quiz}
  複素行列
  \begin{align*}
    J=\begin{pmatrix}0&1&0&\cdots&0&0  \\0&0&1&\\0&0&0&\\&&&\ddots&\ddots\\0&0&0&\cdots&0&1\\0&0&0&0&\cdots&0\end{pmatrix}
  \end{align*}
  は対角化可能ではないことを示せ.
\end{quiz}

\begin{quiz}
  $V$を$n$次元$\KK$-線形空間とし
  $\varphi\colon V\to V$を$\KK$上の線形変換とする.
  $\varphi\circ \varphi = \varphi$を満たすとき,
  $\varphi$は対角化可能であることを示せ.
\end{quiz}

\begin{quiz}
  $\aaa\in\KK^{n\times 1}$とし,
  $\aaa\cdot\transposed{\aaa}$も
  $\transposed{\aaa}\cdot\aaa$も零行列ではないとする.
  $A=\aaa\cdot\transposed{\aaa}$
  とする.
  このとき,
  $A$は対角化可能であることを示せ.
\end{quiz}


\begin{quiz}
  次で定義されるベクトルの列$(\xx_n)_{n\in\NN}$の一般項を求めよ:
  \begin{align*}
    \xx_n &=
    \begin{cases}
      A\xx_{n-1} & (n>0)\\
      \begin{pmatrix}1\\0\end{pmatrix} & (n=0)
    \end{cases}
    &
    A&=\begin{pmatrix}1&1\\1&0\end{pmatrix}.
  \end{align*}
\end{quiz}


