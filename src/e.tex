
\chapter{練習問題}

ここでは,
練習問題を挙げてある.
本原稿を前から順番に読み進める場合に,
どのあたりまで読んだら解けるかという目安ごとに,
問題を分けてある.
問題Aは, 次につながる問題であるので,
読んだ直後に目を通すとよいと思う.

中には, 本質的には同じ問題が何度も出てくることもある.

\section{\Cref{chap:prelim}の後の問題}
主に集合と写像, 体, それから行列に関する基本的な事項について知っていれば,
解けるであろう問題.
%\section{1/15回目: \schoolCalender{1}}
% 講義: 集合と写像, 体.
% \subsection{レポート課題 (締め切り: \schoolCalender{2})}
\subsection{問題A}
\begin{quiz}
  $\KK$を体とし, $V$を$\KK$を成分とする$(2,1)$行列の集合とする.
  つまり,
  \begin{align*}
    V=\Set{\begin{pmatrix}a_1\\a_2\end{pmatrix}|a_1,a_2\in \KK}
  \end{align*}
  とする. 通常の和とスカラー倍を考える.  つまり,
  \begin{align*}
    a&=\begin{pmatrix}a_1\\a_2\end{pmatrix}\in V,&  b&=\begin{pmatrix}b_1\\b_2\end{pmatrix} \in V,
    &\alpha\in\KK&
  \end{align*}
  に対し,
  \begin{align*}
    a+b&=\begin{pmatrix}a_1+b_1\\a_2+b_2\end{pmatrix}\in V, & \alpha a=\begin{pmatrix}\alpha a_1\\\alpha a_2\end{pmatrix} \in V
  \end{align*}
  とする. また,
  \begin{align*}
  \zzero=
  \begin{pmatrix}
    0\\0
  \end{pmatrix}
  \end{align*}
  とする.

  このとき, $\alpha,\beta,1,-1\in\KK$, $a,b,c\in V$に対し, 以下を示せ:
  \begin{enumerate}
    \item 
    \begin{enumerate}
    \item $a+b=b+a$.
    \item $(a+b)+c=a+(b+c)$.
    \item $a+\zzero=a$.
    \item
      $a+(-1 a)=\zzero$.
    \end{enumerate}
  \item
    \begin{enumerate}
    \item $(\alpha\beta)a=\alpha(\beta a)$.
    \item $1a=a$.
    \end{enumerate}
  \item
    \begin{enumerate}
    \item $\alpha(a+b)=\alpha a+\alpha b$.
    \item $(\alpha+\beta)a=\alpha a+\beta a$.
    \end{enumerate}
  \end{enumerate}
\end{quiz}



\subsection{問題B}
$\KK$を体とし, $\KK$の元を成分とする行列について考える.

\begin{quiz}
  次の命題の真偽を判定し, 証明を与えよ:
  \begin{enumerate}
  \item
    $A$, $B$がともに正則な$n$次正方行列である
    とする.
    このとき, $AB$も正則で, その逆行列は$B^{-1}A^{-1}$.
  \item
    $A$を正則な$n$次正方行列とする.
    このとき, $\transposed{A}$も正則で, その逆行列は$\transposed{(A^{-1})}$.
  \item
    $A$を正則な$n$次正方行列とし,
    $c$は逆数を持つ$\KK$の元とする.
    このとき, $cA$も正則で, その逆行列は$\frac{1}{c}A^{-1}$.
  \end{enumerate}
\end{quiz}

\begin{quiz}
  次の命題の真偽を判定し, 証明を与えよ:
  \begin{enumerate}
  \item
    単位行列$E_n$は正則で, その逆行列は$E_n$.
  \item
    $A$は正則であるとする.
    このとき,
    $A^{-1}$は正則で, その逆行列は$A$.
  \end{enumerate}
\end{quiz}


\begin{quiz}
  次の命題の真偽を判定し, 証明を与えよ:
  \begin{enumerate}
  \item
    $A$, $B$がともに正則な$n$次正方行列である
    とする.
    このとき, $A+B$は正則であるとは限らない.
  \end{enumerate}
\end{quiz}

\begin{quiz}
  $v$, $w$を$(2,1)$-行列 (2項数ベクトル) とし,
  $A$を2次正方行列とする.
  $v$, $w$を並べて得られる2次正方行列を$P=(v|w)$とおく.
  $Av=\lambda v$, $Aw=\mu w$を満たす$\lambda,\mu\in\KK$が存在するとする.
  このとき,
  次の命題の真偽を判定し, 証明を与えよ:
  \begin{enumerate}
  \item $AP=PD$, ただし,
    \begin{align*}
      D=\begin{pmatrix}\lambda&0\\0&\mu\end{pmatrix}.
    \end{align*}
  \end{enumerate}
\end{quiz}

\begin{quiz}
  $A$を$(m,n)$行列とし,
  $V$を$(n,1)$行列 ($n$項数ベクトル)を集めた集合とする.
  \begin{align*}
    \KKK=\Set{\xx\in V | A\xx=\zzero}
  \end{align*}
  とする.
  このとき,
  次の命題の真偽を判定し, 証明を与えよ:
  \begin{enumerate}
  \item
    $\aaa,\bbb\in\KKK \implies\aaa+\bbb\in\KKK$.
  \item
    $\aaa\in\KKK, \alpha\in\KK \implies \alpha\aaa\in\KKK$.
  \end{enumerate}
\end{quiz}

\begin{quiz}
  $A$を$(m,n)$行列とし,
  $\bbb$を$(m,1)$行列($m$項数ベクトル)とする.
  $V$を$(n,1)$行列 ($n$項数ベクトル)を集めた集合とし,
  \begin{align*}
    \KKK=\Set{\xx\in V | A\xx=\zzero}
  \end{align*}
  とする.
  $\vv\in V$が
  $A\vv =\bbb$を満たすとする.
  \begin{align*}
    \FFF&=\Set{\xx\in V | A\xx=\bbb}\\
    \FFF'&=\Set{\vv+\aaa | \aaa\in \KKK}
  \end{align*}
  とする.
  
  このとき,
  次の命題の真偽を判定し, 証明を与えよ:
  \begin{enumerate}
  \item
    $\FFF=\FFF'$.
  \end{enumerate}
   
\end{quiz}


\begin{quiz}
  次の行列が正則かどうか判定し正則なら逆行列を求めよ:
  \begin{enumerate}
  \item
    $A=
    \begin{pmatrix}
      1&1&0&1\\
      1&-1&1&2\\
      1&1&1&4\\
      1&-1&1&8
    \end{pmatrix}$.
  \end{enumerate}
\end{quiz}


\section{\Cref{sec:linspace:def}の後の問題}
主に線形空間の定義を知っていれば解けるであろう問題.
%\endinput
%\newpage
%\section{2/15回目: \schoolCalender{2}}
% 講義: 線形空間の定義と例.
%\subsection{レポート課題 (締め切り: \schoolCalender{3})}
\subsection{問題A}
\begin{quiz}
  %\solvelater{quiz:0:1}
  $\KK$を体とし, $V$を$\KK$を成分とする$(2,1)$行列の集合とする.
  このとき, $V$は$\KK$線形空間であった.
  $A$を$\KK$を成分とする$(2,2)$行列とし,
  \begin{align*}
    \shazo{\varphi}{V}{V}{x}{Ax}
  \end{align*}
  とする.
  このとき, 以下を示せ:
  \begin{enumerate}
    \item $a,b\in V\implies \varphi(a+b)=\varphi(a)+\varphi(b)$.
    \item $\alpha\in\KK, a\in V\implies \varphi(\alpha a)=\alpha\varphi(a)$.
  \end{enumerate}
  (ただし, 行列の演算に関する基本的な性質は既知としてよい.)
\end{quiz}

\subsection{問題B}

\begin{quiz}
  $V$を$\KK$-線形空間とする.
  $o,o'\in V$が以下を満たすとする:
  \begin{enumerate}
    \item $x\in V \implies  o+x=x+o=x$.
    \item $x\in V \implies  o'+x=x+o'=x$.
  \end{enumerate}
  このとき, $o=o'$となることを示せ.
\end{quiz}

\begin{quiz}
  $0_V$を
  $\KK$-線形空間$V$の零元(零ベクトル)とする.
  $a\in V$とする.
  このとき, $x,x'\in V$が以下を満たすとする:
  \begin{enumerate}
    \item $a+x=x+a=0_V$.
    \item $a+x'=x'+a=0_V$.
  \end{enumerate}
  このとき, $x=x'$となることを示せ.
\end{quiz}

\begin{quiz}
  $\KK$を体とし,
  $V$を$\KK$ベクトル空間とする.
  $V$の零元を$0_V$とする.
  $v\in V$, $\alpha\in \KK$に対し次が成り立つことを示せ:
  \begin{enumerate}
    \item $\alpha v=0_V\implies \alpha=0$ または$v=0_V$.
  \end{enumerate}
\end{quiz}

\begin{quiz}
  $V$, $W$を
  $\KK$-線形空間とする.
  このとき,
  $V$と$W$の外部直和$V\boxplus W$が
  $\KK$-線形空間であることを示せ.
\end{quiz}

\begin{quiz}
  $V=\Set{x\in\RR|x>0}$とする.
  $x,y\in V$に対し,
  $x \pplus y =x\cdot y$
  とし,
  $x\in V$, $a\in \RR$に対し,
  $a\aact x=x^a$とする.
  このとき,
  $V$は和$\pplus$とスカラー倍$\aact$で
  $\RR$線形空間であることを示せ.
\end{quiz}

\begin{quiz}
  次の命題の真偽を判定し証明を与えよ:
  \begin{enumerate}
  \item $\CC$は通常の和と積で$\RR$線形空間である.
  \item $\RR$は通常の和と積で$\CC$線形空間ではない.
  \end{enumerate}
\end{quiz}

\begin{quiz}
$\NN$で添字付けられた数列$a_0, a_1,\ldots$を$(a_i)_{i\in \NN}$で表す.
  $\ell(\KK) = \Set{(a_i)_{i\in \NN}|a_i\in \KK}$とする.
  $\alpha\in \KK$, $(a_i)_{i\in \NN},(b_i)_{i\in \NN}\in \ell(\KK)$
  に対し和を
  $(a_i)_{i\in \NN}+(b_i)_{i\in \NN} = (a_i+b_i)_{i\in \NN}$で定義し,
  $\alpha\in \KK$, $(a_i)_{i\in \NN}\in \ell(\KK)$
  に対しスカラー倍を
  $\alpha (a_i)_{i\in \NN} = (\alpha a_i)_{i\in \NN}$
  で定義する.
  このとき, $\ell(\KK)$は
  この和とスカラー倍で$\KK$-線形空間となることを示せ.
\end{quiz}

\begin{quiz}
  $S$を集合とし, $V$を$\KK$-線形空間とする.
  $V^S= \Set{f\colon S \to V \text{; 写像}}$とおく.
  $\alpha\in\KK$, $f,g\in V^S$に対し,
  和$f+g\in V^S$とスカラー倍$\alpha f\in V^S$を以下で定める:
  $x\in S$に対し,
  \begin{align*}
    (f+g)(x) &= f(x)+g(x),\\
    (\alpha f)(x) &= \alpha.(f(x)).
  \end{align*}
  このとき,
  $\KK^S$はこの和とスカラー倍で
  $\KK$-線形空間となることを示せ.
\end{quiz}



\section{\Cref{sec:linmap:def}のあとの問題}
主に線形写像の定義を知っていれば解けるであろう問題.
%\endinput
%\newpage
%\section{3/15回目: \schoolCalender{3}}
% 講義: 線形写像の定義と例.
%\subsection{レポート課題 (締め切り: \schoolCalender{4})}
\subsection{問題A}
\begin{quiz}
  $\KK$を体とし, $V$を$\KK$を成分とする$(2,1)$行列の集合とする.
  $A$を$\KK$を成分とする$(2,2)$行列とし,
  \begin{align*}
    \shazo{\varphi}{V}{V}{x}{Ax}
  \end{align*}
  とする. $A$が正則であるとする.
  このとき, 以下の条件を満たす線形写像$\psi\colon V\to V$が存在することを示せ:
  \begin{enumerate}
    \item $\varphi\circ \psi=\id_V$.
    \item $\psi\circ \varphi=\id_V$.
  \end{enumerate}
\end{quiz}
\begin{quiz}
  $\KK$を体とし,
  $V$を$\KK$を成分とする$(3,1)$行列のなす$\KK$-線形空間とし,
  $W$を$\KK$を成分とする$(2,1)$行列のなす$\KK$-線形空間とする.
  $A$を$\KK$を成分とする$(3,2)$行列とし,
  $\zzero$を$W$の零元とする.
  このとき, 次は$\KK$-線形空間であることを示せ:
  \begin{align*}
    \Set{x\in V|Ax=\zzero}.
  \end{align*}
\end{quiz}

\subsection{問題B}
\begin{quiz}
  次の命題の真偽を判定し証明を与えよ:
  $A\in \KK^{n\times n}$とし,
  $\varphi$を次の写像とする:
  \begin{align*}
    \shazo{\varphi_A}{\KK^{n\times n}}{\KK^{n\times n}}{X}{AX-XA}.
  \end{align*}
  このとき, $\varphi_A$は$\KK$-線形である.
\end{quiz}

\begin{quiz}
  次の命題の真偽を判定し証明を与えよ:
  $I=\Set{1,2,\ldots, n}$する.
  $\tr$を次の写像とする:
  \begin{align*}
    \shazo{\tr}{\KK^{n\times n}}{\KK}{(a_{i,j})_{i\in I, j\in I}}{\sum_{i\in I}a_{i,i}}.
  \end{align*}
  この写像は$\KK$-線形である.
\end{quiz}


\begin{quiz}
  次の命題の真偽を判定し証明を与えよ:
  $V$を$\KK$-線形空間とする.
  $a\in A$, $b\in V$とする.
  $\varphi$を次の写像とする:
  \begin{align*}
    \shazo{\varphi}{V}{V}{x}{ax+b}.
  \end{align*}
  このとき,
  $\varphi$が線形写像であることと$b$が$V$の零元であることは同値.
\end{quiz}

\begin{quiz}
  $\varphi$を次の写像とする:
  \begin{align*}
    \shazo{\varphi}{\CC}{\CC}{z}{\overline{z}},
  \end{align*}
  ただし, 
  $x,y\in\RR$に対し$\overline{x+y\sqrt{-1}}=x-y\sqrt{-1}$, つまり,
  $\overline{z}$は$z$の複素共軛とする.
  以下の命題の真偽を判定し証明を与えよ:
  \begin{enumerate}
  \item $\varphi$は$\RR$-線形である.
  \item $\varphi$は$\CC$-線形ではない.
  \end{enumerate}
\end{quiz}

\begin{quiz}
  次の命題の真偽を判定し証明を与えよ:
  $V$を$\KK$線形空間とし,
  $w_1,\ldots,w_r\in V$とする.
    \begin{align*}
      \shazo{\nu_{(w_1,\ldots,w_r)}}{\KK^r}{V}
      {\begin{pmatrix}a_1\\\vdots\\a_r\end{pmatrix}}{a_1 w_1+\cdots+ a_r w_r}
    \end{align*}
    は$\KK$-線形写像である.
\end{quiz}

\begin{quiz}
  次の命題の真偽を判定し証明を与えよ:
  $V$, $W$
  を$\KK$線形空間とし,
  $W$の零元を$0_W$とする.
    \begin{align*}
      \shazo{\underline{0_W}}{V}{W}
      {x}{0_W}
    \end{align*}
  は$\KK$-線形写像である.
\end{quiz}

\begin{quiz}
  次の命題の真偽を判定し証明を与えよ:
  $V$を$\KK$-線形空間とする.
  恒等写像$\id_V$は$\KK$-線形である.
\end{quiz}

\begin{quiz}
  次の命題の真偽を判定し証明を与えよ:
  $V$, $U$, $W$を$\KK$-線形空間とし,
  $\varphi\colon V\to U$,
  $\psi\colon U\to W$を$\KK$-線形写像とする.
  このとき, $\psi\circ\varphi\colon V\to W$は$\KK$-線形写像である.
\end{quiz}

\begin{quiz}
  次の命題の真偽を判定し証明を与えよ:
  $V$, $W$を$\KK$-線形空間とし,
  $\varphi\colon V \to W$を線形写像とする.
  $\varphi$が全単射なら,
  その逆写像
  $\varphi^{-1}$
  も線形写像である.
\end{quiz}

\begin{quiz}
  \label{quiz:hom:sc}
  次の命題の真偽を判定し証明を与えよ:
  $V$, $V$を$\KK$-線形空間とし,
  $\varphi\colon V\to W$を線形写像とする.
  このとき, $\alpha \in \KK$に対し,
  \begin{align*}
  \shazo{\alpha\varphi}{V}{W}
  {x}{\alpha (\varphi(x))}
  \end{align*}
  は$V$から$W$への線形写像.
\end{quiz}

\begin{quiz}
  \label{quiz:hom:sum}
  次の命題の真偽を判定し証明を与えよ:
  $V$, $W$を$\KK$-線形空間とし,
  $\varphi\colon V\to W$,
  $\psi\colon V\to W$
  を線形写像とする.
  このとき, 
  \begin{align*}
  \shazo{\varphi+\psi}{V}{W}
  {x}{\varphi(x) + \psi(x)}
  \end{align*}
  は$V$から$W$への線形写像.
\end{quiz}

\begin{quiz}
  $V$, $W$を$\KK$-線形空間とし,
  $0_V$, $0_W$をそれぞれの零元とする.
  $\varphi\colon V\to W$を$\KK$-線形写像とする.
  次の命題の真偽を判定し証明を与えよ:
  \begin{enumerate}
    \item $\varphi(0_V)=0_W$.
    \item $\varphi(-x)=-\varphi(x)$.
  \end{enumerate}
\end{quiz}

\begin{quiz}
  次の命題の真偽を判定し証明を与えよ:
  $A\in \KK^{m\times n}$に対し,
  $\mu_A$を次の写像とする:
  \begin{align*}
    \shazo{\mu_A}{\KK^n}{\KK^m}{w}{Aw}.
  \end{align*}
$H=\Set{\varphi\colon \KK^n\to\KK^m\text{; 線形写像}}$,
$M=\Set{\mu_A|A\in\KK^{m\times n}}$
  とするとき, $H=M$.
\end{quiz}

\begin{quiz}
  次の命題の真偽を判定し証明を与えよ:
  $\aaa,\bbb\in\RR^n$に対し,
  $\Braket{\aaa,\bbb}$を$\aaa$と$\bbb$の内積とする.
  $\aaa\in\RR^n$に対し
  \begin{align*}
    \shazo{\varphi_{\aaa}}{\RR^n}{\RR}
    {\xx}{\braket{\aaa,\xx}}
  \end{align*}
  は$\RR$-線形写像である.
\end{quiz}

\begin{quiz}
  $\aaa,\bbb\in\RR^n$に対し,
  $\Braket{\aaa,\bbb}$を$\aaa$と$\bbb$の内積とする.
  $\|\aaa\|^2=\Braket{\aaa,\aaa}=1$を満たす
  $\aaa\in\RR^n$に対し,
  \begin{align*}
    \shazo{\varphi_{\aaa}}{\RR^n}{\RR^n}
    {\xx}{\braket{\aaa,\xx}\aaa}
  \end{align*}
  とおく.
  以下の命題の真偽を判定し証明を与えよ:
  \begin{enumerate}
  \item $\varphi_{\aaa}$は$\RR$-線形写像である.
  \item $\varphi_{\aaa}\circ \varphi_{\aaa}=\varphi_{\aaa}$.
  \end{enumerate}
\end{quiz}
\begin{quiz}
  $\aaa,\bbb\in\RR^n$に対し,
  $\Braket{\aaa,\bbb}$を$\aaa$と$\bbb$の内積とする.
  $\|\aaa\|^2=\Braket{\aaa,\aaa}=1$を満たす
  $\aaa\in\RR^n$に対し,
  \begin{align*}
    \shazo{\varphi_{\aaa}}{\RR^n}{\RR^n}
    {\xx}{\xx-\braket{\aaa,\xx}\aaa}
  \end{align*}
  とおく.
  以下の命題の真偽を判定し証明を与えよ:
  \begin{enumerate}
  \item $\varphi_{\aaa}$は$\RR$-線形写像である.
  \item $\varphi_{\aaa}\circ \varphi_{\aaa}=\varphi_{\aaa}$.
  \end{enumerate}
\end{quiz}

\begin{quiz}
  $\aaa,\bbb\in\RR^n$に対し,
  $\Braket{\aaa,\bbb}$を$\aaa$と$\bbb$の内積とする.
  $\|\aaa\|^2=\Braket{\aaa,\aaa}=1$を満たす
  $\aaa\in\RR^n$に対し,
  \begin{align*}
    \shazo{\varphi_{\aaa}}{\RR^n}{\RR^n}
    {\xx}{\xx-2\braket{\aaa,\xx}\aaa}
  \end{align*}
  とおく.
  以下の命題の真偽を判定し証明を与えよ:
  \begin{enumerate}
  \item $\varphi_{\aaa}$は$\RR$-線形写像である.
  \item $\varphi_{\aaa}\circ \varphi_{\aaa}=\id_{\RR^n}$.
  \end{enumerate}
\end{quiz}

\begin{quiz}
\begin{align*}
  \aaa=\begin{pmatrix}a_1\\\vdots\\a_n\end{pmatrix}\in\RR^n
\end{align*}
に対し,
\begin{align*}
  \psi(\aaa)&=\frac{1}{n}\sum_{i=1}^n a_i\\
  \phi(\aaa)&=\frac{1}{n}\sum_{i=1}^n (a_i - \psi(\aaa))^2\\
  \varphi(\aaa)&=\sqrt{\phi(\aaa)}
\end{align*}
このとき,
以下の命題の真偽を判定し証明を与えよ:
\begin{enumerate}
  \item $\psi\colon \RR^n\to \RR$は$\RR$線形写像である.
  \item $\phi\colon \RR^n\to \RR$は$\RR$線形写像ではない.
  \item $\varphi\colon \RR^n\to \RR$は$\RR$線形写像ではない.
\end{enumerate}
\end{quiz}


\section{\Cref{sec:iso:def,sec:subspace:def,sec:subspace:example}の後の問題}
主に同型写像の定義, 部分空間の定義を知っていれば解けるであろう問題.
%\endinput
%\newpage
%\section{4/15回目: \schoolCalender{4}}
% 講義: 同型写像の定義と例. 部分空間の定義と例(1). 
%\subsection{レポート課題 (締め切り: \schoolCalender{5})}
\subsection{問題A}
\begin{quiz}
  %\solvelater{quiz:1:1}
  \begin{align*}
    V&=\Set{\begin{pmatrix}x\\x\\0\end{pmatrix}|x\in \KK}&
    V'&=\Set{\begin{pmatrix}0\\x\\x\end{pmatrix}|x\in \KK}
  \end{align*}
  とする.
  \begin{enumerate}
    \item $V\cap V'$を求めよ.
    \item $\Set{v+v'|v\in V,v'\in V}$は$\KK^3$の部分線形空間であることを示せ.
    \item $V\cup V'$は$\KK^3$の部分線形空間では無いことを示せ.
  \end{enumerate}
\end{quiz}



\subsection{問題B}

\begin{quiz}
  次の命題の真偽を判定し証明を与えよ:
  $[n]=\Set{1,\ldots,n}$とし,
  $\KK^{[n]}=\Set{f\colon [n]\to \KK}$とする.
  \begin{align*}
    \aaa=\begin{pmatrix}a_1\\\vdots\\a_n\end{pmatrix}\in
    \KK^{n}=\Set{\begin{pmatrix}x_1\\\vdots\\x_n\end{pmatrix}|x_i\in\KK}
  \end{align*}
  に対し,
  \begin{align*}
    \shazo{f}{[n]}{\KK}{i}{a_i}
  \end{align*}
  とする.  
  $\varphi$を次の写像とする:
  \begin{align*}
    \shazo{\varphi}{\KK^{n}}{\KK^{[n]}}{\aaa}{f_{\aaa}}.
  \end{align*}
  このとき, $\varphi$は同型写像である.
\end{quiz}

\begin{quiz}
  $[k]=\Set{1,\ldots,k}$とする.
  次の命題の真偽を判定し証明を与えよ:
  \begin{enumerate}
  \item
    $\KK^{m\times n}\simeq \KK^{[m]\times [n]}$.
  \item
    $\KK^\NN\simeq \ell(\KK)$.
  \end{enumerate}
\end{quiz}

\begin{quiz}
  $V=\Set{x\in\RR|x>0}$とする.
  $x,y\in V$に対し,
  $x \pplus y =x\cdot y$
  とし,
  $x\in V$, $a\in \RR$に対し,
  $a\aact x=x^a$とする.
  このとき,
  $V$は和$\pplus$とスカラー倍$\aact$で
  $\RR$線形空間であった.
  また$\RR$は通常の和と積で$\RR$線形空間であった.
  次の命題の真偽を判定し証明を与えよ:
  $\RR\simeq V$.
\end{quiz}

\begin{quiz}
  次の命題の真偽を判定し証明を与えよ:
  $A\in \KK^{m\times n}$に対し,
  $\mu_A$を次の写像とする:
  \begin{align*}
    \shazo{\mu_A}{\KK^n}{\KK^m}{w}{Aw}.
  \end{align*}
  このとき,
  \begin{align*}
    \shazo{\varphi}{\KK^{m\times n}}{\Hom_{\KK}(\KK^n,\KK^m)}
    {A}{\mu_A}
  \end{align*}
  は同型写像である.
\end{quiz}

\begin{quiz}
  次の命題の真偽を判定し証明を与えよ:
  $V$, $W$を$\KK$線形空間とし,
  $\varphi\colon V\to W$を同型写像とする.
  $c\in\KK$が逆数を持つとする.
  このとき,
  $c\varphi\colon V\to W$は同型写像.
\end{quiz}

\begin{quiz}
  $V$, $W$, $U$を$\KK$線形空間とする.
  次の命題の真偽を判定し証明を与えよ:
  \begin{enumerate}
  \item
    $\id_V\colon V\to V$は同型写像.
  \item
    $\varphi\colon V\to W$が同型写像なら
    $\varphi^{-1}\colon W\to V$も同型写像.
  \item
    $\varphi\colon V\to W$, $\psi\colon W \to U$がともに同型写像なら
    $\psi\circ\varphi\colon V\to U$も同型写像.
  \end{enumerate}
\end{quiz}



\begin{quiz}
  $\zeta = e^{\frac{2\pi\sqrt{-1}}{n}}=\cos(\frac{2\pi}{n})+\sin(\frac{2\pi}{n})\sqrt{-1}\in \CC$とする.
  $V=\Set{\sum_{i=0}^{n-1}a_i \zeta^i|a_i\in \QQ}$とする.
  このとき, $V$は通常の和と積で
  $\QQ$-線形空間であることを示せ.
\end{quiz}

\begin{quiz}
  次の命題の真偽を判定し証明を与えよ:
  \begin{enumerate}
  \item $\Set{\begin{pmatrix}x\\2x \end{pmatrix} | x\in\RR}$は,
    $\RR^2$の部分空間である.
  \item $\Set{\begin{pmatrix}x\\x+2 \end{pmatrix} | x\in\RR}$は,
    $\RR^2$の部分空間ではない.
  \item $\Set{\begin{pmatrix}x\\x^2 \end{pmatrix} | x\in\RR}$は,
    $\RR^2$の部分空間ではない.
  \end{enumerate}
\end{quiz}

\begin{quiz}
  次の命題の真偽を判定し証明を与えよ:
  \begin{align*}
    V=\Set{\begin{pmatrix}x_1\\x_2\end{pmatrix}\in\RR^2|x_1x_2>0}
  \end{align*}
  は$\RR^2$の部分空間ではない.
\end{quiz}

\begin{quiz}
  (\Cref{quiz:hom:sc,quiz:hom:sum}と本質的に同じ)
  次の命題の真偽を判定し証明を与えよ:
  $V$, $W$を$\KK$-線形空間とする.
  このとき, $\Hom_{\KK}(V,W)$は$W^V$の部分$\KK$-線形空間である.
\end{quiz}

\begin{quiz}
  次の命題の真偽を判定し証明を与えよ:
  $V$を漸化式$a_i=a_{i-1}+a_{i-2}$で定義される数列全体の集合, つまり,
  \begin{align*}
    V=\Set{(a_i)_{i\in \NN}\in \ell(\KK)|
      \begin{array}{c}
        a_0,a_1\in\KK\\
        i>1\implies a_{i}-a_{i-1}-a_{i-2}=0
      \end{array}
    }
  \end{align*}
  とする.
  $V$は$\ell(\RR)$の部分空間である.
\end{quiz}

\begin{quiz}
  次の命題の真偽を判定し証明を与えよ:
  \begin{align*}
    V=\Set{(a_i)_{i\in \NN}\in \ell(\RR)|\text{次を満たす$n$が存在する: }i>n\implies a_i=0}
  \end{align*}
  とする.
  $V$は$\ell(\RR)$の部分空間である.
\end{quiz}

\begin{quiz}
  $X\subset\RR$に対し,
  次の条件を満たす$M$を$X$の上界と呼ぶ: $x\in X\implies x<M$.
  $X$の上界が存在することを,
  $\sup(X)<\infty$と書く.
  また,
  次の条件を満たす$m$を$X$の下界と呼ぶ: $x\in X\implies m<x$.
  $X$の下界が存在することを,
  $\inf(X)>-\infty$と書く.
  \begin{align*}
    V=\Set{(a_i)_{i\in \NN}\in \ell(\RR)|
      \begin{array}{l}
        \sup\Set{a_i|i\in\NN}<\infty\\
        \inf\Set{a_i|i\in\NN}>-\infty
      \end{array}
    }
  \end{align*}
  とする.
  $V$は$\ell(\RR)$の部分空間であることを,
  $\sup(X)<\infty$, $\inf(X)>-\infty$の定義に基づき証明せよ.
\end{quiz}

\begin{quiz}
  $(a_i)_{i\in\NN}$が次の条件をみたすとき, $\lim_{i\to\infty}a_i =0$と書く:
  \begin{quote}
    $\varepsilon > 0$ ならば,
      次の条件を満たす$N\in \NN$が存在する:
      \begin{align*}
        n>N \implies -\varepsilon < a_n<\varepsilon.
      \end{align*}
  \end{quote}
  このとき,
  \begin{align*}
    V=\Set{(a_i)_{i\in \NN}\in \ell(\RR)|\lim_{i\to\infty}a_i = 0}
  \end{align*}
  とする.
  $V$は$\ell(\RR)$の部分空間であることを,
  $\lim_{i\to\infty}a_i =0$の定義に基づき証明せよ.
\end{quiz}

\begin{quiz}
  次の命題の真偽を判定し証明を与えよ:
  $U$を$\KK$-線形空間とする.
  $V\subset U$に対し,
  以下の2つは同値である:
  \begin{enumerate}
  \item 次の$2$条件を満たす:
    \begin{enumerate}
    \item $v,w\in V$, $a,b\in \KK\implies av+bw\in V$.
    \item $0_U\in V$.
    \end{enumerate}
  \item 次の$2$条件を満たす:
    \begin{enumerate}
    \item $v,w\in V$, $a,b\in \KK\implies av+bw\in V$.
    \item $V\neq \emptyset$.
    \end{enumerate}
  \end{enumerate}
\end{quiz}

\newcommand{\Mod}[1]{\mathrel{\underset{#1}{\equiv}}}
\begin{quiz}
  $U$を$\KK$-線形空間とし, $V$を$U$の部分空間とする.
  $u,u'\in U$に対し,
  $u-u'\in V$となることを$u\Mod{V} u'$と書くことにする.
  このとき,
  次の命題の真偽を判定し証明を与えよ:
  \begin{enumerate}
  \item $u\in U \implies u \Mod{V} u$.
  \item $u,u'\in U, u \Mod{V} u' \implies u' \Mod{V} u$.    
  \item $u,u',u''\in U, u \Mod{V} u', u' \Mod{V} u'' \implies u \Mod{V} u''$.
  \end{enumerate}
\end{quiz}

\begin{quiz}
  $U$を$\KK$-線形空間とし, $V$を$U$の部分空間とする.
  $u,u'\in U$に対し,
  $u-u'\in V$となることを$u\Mod{V} u'$と書くことにする.
  このとき,
  次の命題の真偽を判定し証明を与えよ:
  \begin{enumerate}
  \item $u,u', w,w'\in U, u \Mod{V} u', w \Mod{V} w' \implies u+w \Mod{V} u'+w'$.
  \item $c\in \KK,u,u'\in U, u \Mod{V} u'\implies cu \Mod{V} cu'$.
  \end{enumerate}
\end{quiz}

\section{\Cref{sec:subspace:mor,sec:subspace:sub,sec:subspace:non}の後の問題}
主に, 核, 像, 共通部分, 和空間などを知っていれば,
解けるであろう問題.

\subsection{問題A}
\begin{quiz}
  列ベクトル表示された
  $(m,n)$行列
  $A=(\aaa_1|\cdots|\aaa_n)$
  を考える.
  このとき, 以下を(連立方程式の解の自由度に帰着させることで)示せ:
  \begin{enumerate}
  \item
    $\rank(A)\geq m$ならば,
   次が成り立つ:
   \begin{enumerate}
   \item $\bbb\in\KK^m$とする.
     このとき, $\bbb=c_1\aaa_1+\cdots+c_n\aaa_n$を満たす
     $c_1,\ldots,c_n\in\KK$が存在する.
   \end{enumerate}
  \item
    $\rank(A)< n$ならば,
   次が成り立つ:
   \begin{enumerate}
   \item 
     $c_1\aaa_1+\cdots+c_n\aaa_n=\zzero$かつ
     $(c_1,\ldots,c_n)\neq (0,\ldots,0)$を満たす
     $c_1,\ldots,c_n\in\KK$が存在する.
   \end{enumerate}
  \end{enumerate}
\end{quiz}

\subsection{問題B}



\begin{quiz}
  次の命題の真偽を判定し証明を与えよ:
  $U$を$\KK$-線形空間とし,
  $V,V', W$を$U$の部分空間とする.
  このとき,
  $(V+V')\cap W=(V\cap W)+(V'\cap W)$となるとは限らない.
\end{quiz}

\begin{quiz}
  次の命題の真偽を判定し証明を与えよ:
  $U$を$\KK$-線形空間とし,
  $V,V', W$を$U$の部分空間とする.
  このとき,
  $(V\cap V')+ W=(V+W)\cap (V'+W)$となるとは限らない.
\end{quiz}


\begin{quiz}
  次の命題の真偽を判定し証明を与えよ:
  $U$を$\KK$-線形空間とし,
  $V,W$を$U$の部分空間とする.
  このとき,
  \begin{align*}
    \tilde V&=\Set{(v,-v)\in U\boxplus U|v\in V} \\
    \check W&=\Set{(w,0_U)\in U\boxplus U|w\in W} \\
    \hat U&=\Set{(0_U,u)\in U\boxplus U|u\in U}
  \end{align*}
  とおく.
  また,
  \begin{align*}
    \check X &= (\tilde V + \check W) \cap \hat U \\
    X&=\Set{x|(0_U,x) \in\check X}
  \end{align*}
  とする.
  このとき, $X=V\cap W$.
\end{quiz}

\begin{quiz}
  $V$, $W$を$\KK$-線形空間とし,
  $\varphi\colon V\to W$を$\KK$線形写像とする.
  このとき,
  以下の命題の真偽を判定し証明を与えよ:
  \begin{enumerate}
  \item
    $V$の部分空間$V'$に対し,
    $W'=\Set{\varphi(x)\in W|x\in V'}$は$W$の部分空間である.
  \item
    $W$の部分空間$W''$に対し,
    $V''=\Set{x\in V|\varphi(x)\in W''}$は$V$の部分空間である.
  \end{enumerate}
\end{quiz}

\begin{quiz}
  $\varphi\colon \RR[x]\to \RR[x]$を,
  $x^n \in \KK[x]$に対して$nx^{n-1}$を対応させる$\RR$-線形写像,
  つまり, 多項式$f\in\RR[x]$に対し$f$の微分を対応させる写像とする.
  このとき, $\Img(\varphi)$と$\Ker(\varphi)$を求めよ.
\end{quiz}

\begin{quiz}
  $V$を$\KK$-線形空間とし,
  $\varphi\colon V\to V$を$\KK$線形写像とする.
  $\varphi^{i}=\underbrace{\varphi\circ\cdots\circ\varphi}_{i}$とし,
  $\varphi^{0}=\id_V$とする.
  このとき,
  以下の命題の真偽を判定し証明を与えよ:
  \begin{enumerate}
    \item $i\in\NN$に対し, $\Ker(\varphi^i)\subset \Ker(\varphi^{i+1})$.
    \item $\Ker(\varphi^N) =\Ker(\varphi^{N+1})$かつ$i\geq N\implies \Ker(\varphi^i) =\Ker(\varphi^{i+1})$.
  \end{enumerate}
\end{quiz}

\begin{quiz}
  $V$を$\KK$-線形空間とし,
  $\varphi\colon V\to V$を$\KK$線形写像とする.
  $\varphi^{i}=\underbrace{\varphi\circ\cdots\circ\varphi}_{i}$とし,
  $\varphi^{0}=\id_V$とする.
  このとき,
  以下の命題の真偽を判定し証明を与えよ:
  \begin{enumerate}
    \item $i\in\NN$に対し, $\Img(\varphi^i)\supset \Img(\varphi^{i+1})$.
    \item $\Img(\varphi^N) =\Img(\varphi^{N+1})$かつ$i\geq N\implies \Img(\varphi^i) =\Img(\varphi^{i+1})$.
  \end{enumerate}
\end{quiz}


\begin{quiz}
  $V$, $W$を$\KK$-線形空間とし,
  $\varphi\colon V\to W$を$\KK$線形写像とする.
  \begin{align*}
    F&=\Set{(v,\varphi(v))\in V\boxplus W| v\in V }\\
    K&=\Set{(x,y) \in F | y=0_W}
  \end{align*}
  とおく.
  また
  \begin{align*}
    \shazo{\pi_1}{V\boxplus W}{V}
          {(v,w)}{v}
          &&
    \shazo{\pi_2}{V\boxplus W}{V}
          {(v,w)}{w}
  \end{align*}
  とする.
  このとき,
  以下の命題の真偽を判定し証明を与えよ:
  \begin{enumerate}
  \item
    $F$は$V\boxplus W$の部分空間である.
  \item
    $\Ker(\varphi)=\Set{\varphi_1(a) | a\in K}$.
  \item
    $\Img(\varphi)=\Set{\varphi_2(a) | a\in F}$.
  \end{enumerate}
\end{quiz}

\begin{quiz}
  $V$, $W$, $V'$, $W'$を$\KK$-線形空間とする.
  $\varphi\colon V\to V'$,
  $\phi\colon W\to W'$を同型写像とし,
  $\psi\colon V\to W$を$\KK$-線形写像とする.
  $\psi'=\phi\circ\psi\circ\varphi^{-1}$とおく.
  このとき,
  以下の命題の真偽を判定し証明を与えよ:
  \begin{enumerate}
  \item $\Ker(\psi')\simeq\Ker(\psi)$.
  \item $\Img(\psi')\simeq\Img(\psi)$.
  \end{enumerate}
\end{quiz}

\begin{quiz}
  \begin{align*}
    \aaa=\begin{pmatrix}a_1\\\vdots\\a_n\end{pmatrix},\quad
    \bbb=\begin{pmatrix}b_1\\\vdots\\b_n\end{pmatrix}\in\RR^n
  \end{align*}
  に対し,
  \begin{align*}
    B(\aaa,\bbb)=\sum_{i=1}^n a_ib_i
  \end{align*}
  とする.
  つまり, $B(\aaa,\bbb)$を$\aaa$と$\bbb$の内積とする.
  $V$を$\RR^n$の部分空間とし,
  \begin{align*}
    W=\Set{\aaa\in \RR^n|\bbb\in V\implies B(\aaa,\bbb)=0}
  \end{align*}
  とおく.
  このとき,
  以下の命題の真偽を判定し証明を与えよ:
  \begin{enumerate}
  \item
    $W$は$\RR^n$の部分空間である.
  \item
    $\RR^n=V\oplus W$と内部直和に分解できる.
  \end{enumerate}
\end{quiz}

\begin{quiz}  
  \begin{align*}
    \aaa=\begin{pmatrix}a_1\\a_2\end{pmatrix},\quad
    \bbb=\begin{pmatrix}b_1\\b_2\end{pmatrix}\in\CC^2
  \end{align*}
  に対し,
  \begin{align*}
    B(\aaa,\bbb)= a_1 \overline{b_1}+ a_2 \overline{b_2}
  \end{align*}
  とする.
  ただし, $\overline{x}$は$x$の複素共軛を表す.
  $V$を$\CC^2$の部分空間とし,
  \begin{align*}
    W=\Set{\aaa\in \CC^2|\bbb\in V\implies B(\aaa,\bbb)=0}
  \end{align*}
  とおく.
  このとき,
  以下の命題の真偽を判定し証明を与えよ:
  \begin{enumerate}
  \item
    $W$は$\CC^2$の部分空間である.
  \item
    $\CC^2=V\oplus W$と内部直和に分解できる.    
  \end{enumerate}
\end{quiz}

\begin{quiz}  
  \begin{align*}
    \aaa=\begin{pmatrix}a_1\\a_2\end{pmatrix},\quad
    \bbb=\begin{pmatrix}b_1\\b_2\end{pmatrix}\in\CC^2
  \end{align*}
  に対し,
  \begin{align*}
    B(\aaa,\bbb)= a_1 b_1+ a_2 b_2
  \end{align*}
  とする.
  $V$を$\CC^2$の部分空間とし,
  \begin{align*}
    W=\Set{\aaa\in \CC^2|\bbb\in V\implies B(\aaa,\bbb)=0}
  \end{align*}
  とおく.
  このとき,
  以下の命題の真偽を判定し証明を与えよ:
  \begin{enumerate}
  \item
    $W$は$\CC^2$の部分空間である.
  \item
    $\CC^2=V\oplus W$と内部直和に分解できないこともある.    
  \end{enumerate}
\end{quiz}

\begin{quiz}
  $\aaa,\bbb\in\RR^n$に対し,
  $\Braket{\aaa,\bbb}$を$\aaa$と$\bbb$の内積とする.
  $\|\aaa\|^2=\Braket{\aaa,\aaa}=1$を満たす
  $\aaa\in\RR^n$に対し,
  \begin{align*}
    \shazo{\varphi_{\aaa}}{\RR^n}{\RR^n}
    {\xx}{\xx-\braket{\aaa,\xx}\aaa}
  \end{align*}
  とおく.
  このとき,
  以下の命題の真偽を判定し証明を与えよ:
  $\Ker(\varphi_{\aaa})=\Set{c\aaa|c\in \RR}$.
\end{quiz}


\begin{quiz}
  次の命題の真偽を判定し証明を与えよ:
  $V$を$\KK$-線形空間とする.
  $\varphi\colon V\to V$は次を満たす$\KK$-線形写像であるとする:
  $\varphi\circ\varphi = \varphi$.
  このとき,
  $V=\Img(\varphi)\oplus \Ker(\varphi)$と内部直和分解する.
\end{quiz}

\begin{quiz}
  次の命題の真偽を判定し証明を与えよ:
  $U$を$\KK$-線形空間とし,
  $I$を集合とする.
  $i\in I$に対し, $V_i$を$U$の部分空間とする.
  このとき,
  $\bigcap_{i\in I}V_i$は$U$の部分空間である.
\end{quiz}


\begin{quiz}
  $U$を$\KK$-線形空間とする.
  $i\in \NN$に対し,
  $V_i$を$U$の部分空間とし,
  $V_{i}\subset V_{i+1}$
  を満たしているとする.
  $V=\bigcup_{i\in \NN}V_i$
  とおく.
  このとき,
  $V$は$U$の部分空間である.
\end{quiz}

\begin{quiz}
  $U$を$\KK$-線形空間とし,
  $v_1,\ldots, v_r \in U$とする.
  \begin{align*}
    V=\Set{a_1v_1+\cdots +a_rv_r|a_i\in \KK}    
  \end{align*}
  とおく. また,
  \begin{align*}
    \SSS=\Set{W|v_1,\ldots,v_r \in W, \text{$W$は$U$の部分空間}}    
  \end{align*}
  とし,
  \begin{align*}
    W_0=\bigcap_{W\in\SSS}W
  \end{align*}
  とおく.
  このとき,
  次の命題の真偽を判定し証明を与えよ:
  \begin{enumerate}
  \item $V=W_0$.
  \item $W\in \SSS \implies V\subset W$.
  \end{enumerate}
\end{quiz}


\begin{quiz}
  次の命題の真偽を判定し証明を与えよ:
  $U$を$\KK$-線形空間とし,
  $V_1, V_2$を$U$の部分空間とする.
  \begin{align*}
    V=\Set{v_1+v_2|v_1\in V_1,v_2\in V_2}    
  \end{align*}
  とおく.
  また,
  \begin{align*}
    \SSS=\Set{W|V_1\cup V_2 \subset W, \text{$W$は$U$の部分空間}}    
  \end{align*}
  とし,
  \begin{align*}
    W_0=\bigcap_{W\in\SSS}W
  \end{align*}
  とおく.
  このとき,
  次の命題の真偽を判定し証明を与えよ:
  \begin{enumerate}
  \item $V=W_0$.
  \item $W\in \SSS \implies V\subset W$.
  \end{enumerate}
\end{quiz}


\begin{quiz}
  $U$を$\KK$-線形空間とし,
  $i\in\NN$に対し,
  $v_i \in U$とする.
  \begin{align*}
    V_r=\Set{a_1v_1+\cdots +a_rv_r|a_i\in \KK}    
  \end{align*}
  とし,
  \begin{align*}
    V=\bigcup_{r\in\NN}V_r
  \end{align*}
  とおく. また,
  \begin{align*}
    \SSS=\Set{W|\Set{v_i|i\in \NN}\subset W, \text{$W$は$U$の部分空間}}    
  \end{align*}
  とし,
  \begin{align*}
    W_0=\bigcap_{W\in\SSS}W
  \end{align*}
  とおく.
  このとき,
  次の命題の真偽を判定し証明を与えよ:
  \begin{enumerate}
  \item $V=W_0$.
  \item $W\in \SSS \implies V\subset W $.
  \end{enumerate}
\end{quiz}

\begin{quiz}
  次の命題の真偽を判定し証明を与えよ:
  $U$を$\KK$-線形空間とし,
  $I$を集合とする.
  $i\in \NN$に対し,
  $V_i$を$U$の部分空間とする.
  \begin{align*}
    V=\bigcup_{i\in \NN}(V_0+V_1+\cdots+V_i)
  \end{align*}
  とおく.
  また,
  \begin{align*}
    \SSS=\Set{W|\bigcup_{i\in\NN}V_i \subset W, \text{$W$は$U$の部分空間}}    
  \end{align*}
  とし,
  \begin{align*}
    W_0=\bigcap_{W\in\SSS}W
  \end{align*}
  とおく.
  このとき,
  次の命題の真偽を判定し証明を与えよ:
  \begin{enumerate}
  \item $V=W_0$.
  \item $W\in \SSS \implies V\subset W$.
  \end{enumerate}
\end{quiz}


\begin{quiz}
  $V$と$W$を$\KK$-線形空間とし,
  $0_V$を$V$の零元,
  $0_W$を$W$の零元とする.
  $V\boxplus W=\Set{(v,w)|v\in V,w\in W}$とし,
  \begin{align*}
    \shazo{\pi}{V\boxplus W}{V}
    {(v,w)}{v}&&
    \shazo{\iota}{V}{V\boxplus W}
    {v}{(v,0_W)}\\
    \shazo{\varpi}{V\boxplus W}{W}
    {(v,w)}{w}&&
    \shazo{\nu}{W}{V\boxplus W}
    {w}{(0_V,w)}
  \end{align*}
  とする.
  以下の命題の真偽を判定し証明を与えよ:
  \begin{enumerate}
  \item
    $U$を$\KK$-線形空間とし,
    $\varphi\colon U\to V$,
    $\phi\colon U\to W$
    を$\KK$-線形写像とする.
    このとき,
    $\psi\colon U\to V\boxplus W$で以下の条件を満たすものが,
    ただ一つ存在する:
    \begin{align*}
      \varphi&=\pi\circ\psi\\
      \phi&=\varpi\circ\psi.
    \end{align*}
  \item
    $U$を$\KK$-線形空間とし,
    $\varphi\colon V\to U$,
    $\phi\colon W\to U$
    を$\KK$-線形写像とする.
    このとき,
    $\psi\colon V\boxplus W\to U$で以下の条件を満たすものが,
    ただ一つ存在する:
    \begin{align*}
      \varphi&=\psi\circ\iota\\
      \phi&=\psi\circ\nu.
    \end{align*}
  \end{enumerate}
\end{quiz}


\begin{quiz}
  次の命題の真偽を判定し証明を与えよ:
  $I$を集合とし, $i\in I$に対し$V_i$を$0_{i}$を零元とする$\KK$-線形空間とする.
  $i \in I$に対し$v_i\in V_i$とする列を$(v_i)_{i\in I}$で表し,
  \begin{align*}
    \prod_{i\in I} V_i &=\Set{(v_i)_{i\in I}|v_i\in V_i}\\
    \bigoplus_{i\in I} V_i &=\Set{(v_i)_{i\in I} \in \prod_{i\in I} V_i|\#\Set{i\in I|v_i\neq 0_{i}}<\infty}
  \end{align*}
  とおく.
  $\prod_{i\in I} V_i$は次の和とスカラー倍で$\KK$-線形空間である:
  $(v_i)_{i\in I}+(w_i)_{i\in I}=(v_i+w_i)_{i\in I}$,
  $c(v_i)_{i\in I}=(cv_i)_{i\in I}$.
  このとき,
  $\bigoplus_{i\in I} V_i$は$\prod_{i\in I} V_i$の部分空間である.
\end{quiz}


\begin{quiz}
  次の命題の真偽を判定し証明を与えよ:
  $I$を集合とし, $i\in I$に対し$V_i$を$0_{i}$を零元とする$\KK$-線形空間とする.
  $i \in I$に対し$v_i\in V_i$とする列を$(v_i)_{i\in I}$で表し,
  \begin{align*}
    \prod_{i\in I} V_i &=\Set{(v_i)_{i\in I}|v_i\in V_i}
%    \bigoplus_{i\in I} V_i &=\Set{(v_i)_{i\in I} \in \prod_{i\in I} V_i|\#\Set{i\in I|v_i\neq 0_{i}}<\infty}
  \end{align*}
  とおく.
  \begin{align*}
    \shazo{\pi_i}{\prod_{i\in I} V_i}{V_i}
    {(v_i)_{i\in I}}{v_i}
  \end{align*}

  次の命題の真偽を判定し証明を与えよ:
  \begin{enumerate}
  \item
    $U$を$\KK$-線形空間とし,
    $i\in I$に対し
    $\varphi_i\colon U\to V_i$
    を$\KK$-線形写像とする.
    このとき,
    $\psi\colon U\to \prod_{i\in I} V_i$で以下の条件を満たすものが,
    ただ一つ存在する:
    \begin{align*}
      i\in I\implies \varphi&=\pi_i\circ\psi
    \end{align*}
  \end{enumerate}
\end{quiz}


\begin{quiz}
  次の命題の真偽を判定し証明を与えよ:
  $I$を集合とし, $i\in I$に対し$V_i$を$0_{i}$を零元とする$\KK$-線形空間とする.
  $i \in I$に対し$v_i\in V_i$とする列を$(v_i)_{i\in I}$で表し,
  \begin{align*}
    \prod_{i\in I} V_i &=\Set{(v_i)_{i\in I}|v_i\in V_i}\\
    \bigoplus_{i\in I} V_i &=\Set{(v_i)_{i\in I} \in \prod_{i\in I} V_i|\#\Set{i\in I|v_i\neq 0_{i}}<\infty}
  \end{align*}
  とおく.
  $i\in I$,
  $v\in V_i$に対し,
  \begin{align*}
    \delta_{i,j}v=
    \begin{cases}
      v & (j=i)\\
      0_j& (j\neq i)
    \end{cases}
  \end{align*}
  と書くことにする.
  \begin{align*}
    \shazo{\nu_i}{V_i}{\bigoplus_{i\in I} V_i}
    {v}{(\delta_{i,j}v_i)_{j\in I}}
  \end{align*}
  とするとき,
  次の命題の真偽を判定し証明を与えよ:
  \begin{enumerate}
  \item
    $U$を$\KK$-線形空間とし,
    $i\in I$に対し
    $\varphi_i\colon  V_i \to U$
    を$\KK$-線形写像とする.
    このとき,
    $\psi\colon \bigoplus_{i\in I} V_i\to U$で以下の条件を満たすものが,
    ただ一つ存在する:
    \begin{align*}
      i\in I\implies \varphi_i&=\psi\circ\nu_i
    \end{align*}
  \end{enumerate}
\end{quiz}

\begin{quiz}
  次の命題の真偽を判定し証明を与えよ:
  $U$を$\KK$-線形空間とする.
  $i\in \NN$に対し,
  $V_i$を$U$の部分空間とし,
  $V_{i}\subset V_{i+1}$
  を満たしているとする.
  \begin{align*}
    \shazo{\nu_i}{V_i}{\bigcup_{i\in \NN}V_i}
    {v}{v}
  \end{align*}
  とするとき,
  次の命題の真偽を判定し証明を与えよ:
  \begin{enumerate}
  \item
    $U$を$\KK$-線形空間とし,
    $i\in I$に対し
    $\varphi_i\colon  V_i \to U$
    を$\KK$-線形写像とする.
    このとき,
    $\psi\colon \bigcup_{i\in \NN} V_i\to U$で以下の条件を満たすものが,
    ただ一つ存在する:
    \begin{align*}
      i\in \NN\implies \varphi_i&=\psi\circ\nu_i
    \end{align*}
  \end{enumerate}
\end{quiz}


%% \begin{quiz}
%% \begin{align*}
%%   \aaa=\begin{pmatrix}a_1\\\vdots\\a_n\end{pmatrix}\in\RR^n
%% \end{align*}
%% に対し,
%% \begin{align*}
%%   \psi(\aaa)&=\frac{1}{n}\sum_{i=1}^n a_i\\
%%   \phi(\aaa)&=\frac{1}{n}\sum_{i=1}^n (a_i - \psi(\aaa))^2\\
%%   \varphi(\aaa)&=\sqrt{\phi(\aaa)}
%% \end{align*}
%% このとき,
%% 以下の命題の真偽を判定し証明を与えよ:
%% \begin{enumerate}
%%   \item $\psi\colon \RR^n\to \RR$は$\RR$線形写像である.
%%   \item $\phi\colon \RR^n\to \RR$は$\RR$線形写像ではない.
%%   \item $\varphi\colon \RR^n\to \RR$は$\RR$線形写像ではない.
%% \end{enumerate}
%% \end{quiz}

%% \begin{quiz}
%%   次の命題の真偽を判定し証明を与えよ:
%%   \begin{align*}
%%     V=\Set{(a_i)_{i\in \NN}\in \ell(\RR)|\text{次を満たす$n$が存在する: }i>n\implies a_i=0}
%%   \end{align*}
%%   とする.
%%   $V$は$\ell(\RR)$の部分空間である.
%% \end{quiz}
